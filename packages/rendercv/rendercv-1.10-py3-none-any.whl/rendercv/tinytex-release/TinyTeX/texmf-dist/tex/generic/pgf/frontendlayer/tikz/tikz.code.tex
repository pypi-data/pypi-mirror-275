% Copyright 2019 by Till Tantau
%
% This file may be distributed and/or modified
%
% 1. under the LaTeX Project Public License and/or
% 2. under the GNU Public License.
%
% See the file doc/generic/pgf/licenses/LICENSE for more details.

\ProvidesPackageRCS{tikz.code.tex}


\def\tikzerror#1{\pgfutil@packageerror{tikz}{#1}{}}

% Always-present libraries (|\usepgflibrary| defined in
% \file{pgfutil-common.tex}).
\usepgflibrary{plothandlers}

\newdimen\tikz@lastx
\newdimen\tikz@lasty
\newdimen\tikz@lastxsaved
\newdimen\tikz@lastysaved
\newdimen\tikz@lastmovetox
\newdimen\tikz@lastmovetoy

\newdimen\tikzleveldistance
\newdimen\tikzsiblingdistance

\newbox\tikz@figbox
\newbox\tikz@figbox@bg
\newbox\tikz@tempbox
\newbox\tikz@tempbox@bg

\newcount\tikztreelevel
\newcount\tikznumberofchildren
\newcount\tikznumberofcurrentchild

\newcount\tikz@fig@count

\newif\iftikz@node@is@a@label
\newif\iftikz@snaked
\newif\iftikz@decoratepath

\let\tikz@options\pgfutil@empty
% |\tikz@addoption| adds \texttt{#1} at the end of the replacement
% text of |\tikz@options| without expansion.
\def\tikz@addoption#1{%
  \expandafter\def\expandafter\tikz@options\expandafter{\tikz@options#1}}%
% Same as |tikz@addoption| for |\tikz@mode|. Note that |\tikz@mode| is
% initially let to |\pgfutil@empty| later (see path usage options).
\def\tikz@addmode#1{%
  \expandafter\def\expandafter\tikz@mode\expandafter{\tikz@mode#1}}%
% Same as |tikz@addoption| for |\tikz@transform|. Works even if
% |\tikz@transform| is not defined. In that case, nothing is added to
% |\tikz@transform|: \texttt{#1} is expanded.
\def\tikz@addtransform#1{%
  \ifx\tikz@transform\relax
    #1%
  \else
    \expandafter\def\expandafter\tikz@transform\expandafter{\tikz@transform#1}%
  \fi
}%


% TikZ options management.

% Setting up the tikz key family (key management needs
% \file{pgfkeys.code.tex});
\pgfkeys{/tikz/.is family}%

% |\tikzset| is a shortcut to set keys that belongs to the tikz
% family.
\def\tikzset{\pgfqkeys{/tikz}}%

% Note: |\tikzoption| is supported for compatibility only. |\tikzset|
% should be used instead.
\def\tikzoption#1{%
  \pgfutil@ifnextchar[%]
    {\tikzoption@opt{#1}}{\tikzoption@noopt{#1}}}%
\def\tikzoption@opt#1[#2]#3{%
  \pgfkeysdef{/tikz/#1}{#3}%
  \pgfkeyssetvalue{/tikz/#1/.@def}{#2}}%
\def\tikzoption@noopt#1#2{%
  \pgfkeysdef{/tikz/#1}{#2}%
  \pgfkeyssetvalue{/tikz/#1/.@def}{\pgfkeysvaluerequired}}%

% Baseline options
\tikzoption{baseline}[0pt]{%
  \pgfutil@ifnextchar(%)
    {\tikz@baseline@coordinate}{\tikz@baseline@simple}#1\@nil}%
\def\tikz@baseline@simple#1\@nil{\pgfsetbaseline{#1}}%
\def\tikz@baseline@coordinate#1\@nil{%
  \pgfsetbaselinepointlater{\tikz@scan@one@point\pgfutil@firstofone#1}}%

\tikzoption{trim left}[0pt]{\pgfutil@ifnextchar({\tikz@trim@coordinate{left}}{\tikz@trim@simple{left}}#1\@nil}%)%
\tikzoption{trim right}{\pgfutil@ifnextchar({\tikz@trim@coordinate{right}}{\tikz@trim@simple{right}}#1\@nil}%)%
\def\tikz@trim@simple#1#2\@nil{\csname pgfsettrim#1\endcsname{#2}}%
\def\tikz@trim@coordinate#1#2\@nil{\csname pgfsettrim#1pointlater\endcsname{\tikz@scan@one@point\pgfutil@firstofone#2}}%

% Draw options
\tikzoption{line width}{\tikz@semiaddlinewidth{#1}}%

\def\tikz@semiaddlinewidth#1{\tikz@addoption{\pgfsetlinewidth{#1}}\pgfmathsetlength\pgflinewidth{#1}}%

\tikzoption{cap}{\tikz@addoption{\csname pgfset#1cap\endcsname}}%
\tikzoption{join}{\tikz@addoption{\csname pgfset#1join\endcsname}}%
\tikzoption{line cap}{\tikz@addoption{\csname pgfset#1cap\endcsname}}%
\tikzoption{line join}{\tikz@addoption{\csname pgfset#1join\endcsname}}%
\tikzoption{miter limit}{\tikz@addoption{\pgfsetmiterlimit{#1}}}%

\tikzoption{dash pattern}{% syntax: on 2pt off 3pt on 4pt ...
  \def\tikz@temp{#1}%
  \ifx\tikz@temp\pgfutil@empty%
    \def\tikz@dashpattern{}%
    \tikz@addoption{\pgfsetdash{}{0pt}}%
  \else%
    \def\tikz@dashpattern{}%
    \expandafter\tikz@scandashon\pgfutil@gobble#1o\@nil%
    \edef\tikz@temp{{\tikz@dashpattern}{\noexpand\tikz@dashphase}}%
    \expandafter\tikz@addoption\expandafter{\expandafter\pgfsetdash\tikz@temp}%
  \fi}%
\tikzoption{dash phase}{%
  \def\tikz@dashphase{#1}%
  \edef\tikz@temp{{\tikz@dashpattern}{\noexpand\tikz@dashphase}}%
  \expandafter\tikz@addoption\expandafter{\expandafter\pgfsetdash\tikz@temp}%
}%
\tikzoption{dash}{\tikz@parse@full@dash#1\pgf@stop}%
\def\tikz@parse@full@dash#1phase#2\pgf@stop{%
  \def\tikz@dashphase{#2}%
  \def\tikz@temp{#1}%
  \ifx\tikz@temp\pgfutil@empty%
    \def\tikz@dashpattern{}%
    \tikz@addoption{\pgfsetdash{}{0pt}}%
  \else%
    \def\tikz@dashpattern{}%
    \expandafter\tikz@scandashon\pgfutil@gobble#1o\@nil%
    \edef\tikz@temp{{\tikz@dashpattern}{\noexpand\tikz@dashphase}}%
    \expandafter\tikz@addoption\expandafter{\expandafter\pgfsetdash\tikz@temp}%
  \fi%
}%
\def\tikz@dashphase{0pt}%
\def\tikz@dashpattern{}%

\def\tikz@scandashon n#1o{%
  \expandafter\def\expandafter\tikz@dashpattern\expandafter{\tikz@dashpattern{#1}}%
  \pgfutil@ifnextchar\@nil{\pgfutil@gobble}{\tikz@scandashoff}}%
\def\tikz@scandashoff ff#1o{%
  \expandafter\def\expandafter\tikz@dashpattern\expandafter{\tikz@dashpattern{#1}}%
  \pgfutil@ifnextchar\@nil{\pgfutil@gobble}{\tikz@scandashon}}%

% use a decoration to expand the `off' section of a dash pattern.
% https://tex.stackexchange.com/a/133357
\tikzset{
    dash expand off/.code={%
        \ifcsname tikz@library@decorations@loaded\endcsname\else
            \tikzerror{You need \string\usetikzlibrary{decorations} for ``dash expand off''}%
        \fi
        \tikz@addoption{%
            \pgfgetpath\currentpath
            \pgfprocessround{\currentpath}{\currentpath}%
            \pgf@decorate@parsesoftpath{\currentpath}{\currentpath}%
            % All of \on, \off, \dashphase, \rest, and \onoff are unit-free.
            % Parse \on and \off from the current path
            \pgfmathsetmacro\on{\expandafter\pgfutil@firstoftwo\tikz@dashpattern}%
            \pgfmathsetmacro\off{\expandafter\pgfutil@secondoftwo\tikz@dashpattern}%
            % \dashphase = max(\on - \dashphase, 0)
            \pgfmathsetmacro\tikz@dashphase{\tikz@dashphase}%
            \pgfmathsubtract@{\on}{\tikz@dashphase}%
            \pgfmathmax@{\pgfmathresult,0}%
            \let\dashphase=\pgfmathresult
            % \rest = \pgf@decorate@totalpathlength - \on + 2\dashphase
            \edef\rest{\pgf@sys@tonumber\dimexpr\pgf@decorate@totalpathlength - \on pt + 2\dimexpr\dashphase pt\relax\relax}%
            % \onoff = \on + \off
            \edef\onoff{\pgf@sys@tonumber\dimexpr\on pt+\off pt\relax}%
            % \nfullonoff = max(floor(\rest/\onoff), 1)
            \pgfmathdivide@{\rest}{\onoff}%
            \pgfmathfloor@{\pgfmathresult}%
            \pgfmathmax@{\pgfmathresult,1}%
            % \offexpand = max(\rest/\nfullonoff - \on, \off)
            \pgfmathdivide@{\rest}{\pgfmathresult}%
            \pgfmathsubtract@{\pgfmathresult}{\on}%
            \pgfmathmax@{\pgfmathresult,\off}%
            \edef\tikz@marshal{\noexpand\pgfsetdash{{+\on pt}{+\pgfmathresult pt}}{+\dashphase pt}}%
            \tikz@marshal
        }%
    }
}

\tikzoption{draw opacity}{\tikz@addoption{\pgfsetstrokeopacity{#1}}}%

% Double draw options
\tikzoption{double}[]{%
  \def\tikz@temp{#1}%
  \ifx\tikz@temp\tikz@nonetext%
    \tikz@addmode{\tikz@mode@doublefalse}%
  \else%
    \ifx\tikz@temp\pgfutil@empty%
    \else%
      \pgfsetinnerstrokecolor{#1}%
    \fi%
    \tikz@addmode{\tikz@mode@doubletrue}%
    \tikzset{every double/.try}%
  \fi}%
\tikzoption{double distance}{%
  \pgfmathsetlength{\pgf@x}{#1}%
  \edef\tikz@double@setup{%
    \pgf@x=\the\pgf@x%
    \advance\pgf@x by2\pgflinewidth%
    \pgflinewidth=\pgf@x%
    \noexpand\pgfsetlinewidth{\pgflinewidth}%
    \noexpand\pgfsetinnerlinewidth{\the\pgf@x}%
  }%
  \tikzset{double}}%
\def\tikz@double@setup{%
  \pgf@x=2\pgflinewidth%
  \advance\pgf@x by0.6pt%
  \pgflinewidth=\pgf@x%
  \pgfsetlinewidth{\pgflinewidth}%
  \pgfsetinnerlinewidth{0.6pt}%
}%
\tikzset{double distance between line centers/.code={
  \pgfmathsetlength{\pgf@x}{#1}%
  \edef\tikz@double@setup{%
    \pgf@x=\pgflinewidth%
    \pgf@xa=\the\pgf@x%
    \advance\pgf@x by\pgf@xa%
    \advance\pgf@xa by-\pgflinewidth%
    \pgflinewidth=\pgf@x%
    \noexpand\pgfsetlinewidth{\pgflinewidth}%
    \noexpand\pgfsetinnerlinewidth{\pgf@xa}%
  }%
  \tikzset{double}}}%
\tikzset{double equal sign distance/.style={double distance between line centers=0.45ex}}%




% Fill options

\tikzoption{even odd rule}[]{\tikz@addoption{\pgfseteorule}}%
\tikzoption{nonzero rule}[]{\tikz@addoption{\pgfsetnonzerorule}}%

\tikzoption{fill opacity}{\tikz@addoption{\pgfsetfillopacity{#1}}}%


% Joined fill/draw options

\tikzoption{opacity}{\tikz@addoption{\pgfsetstrokeopacity{#1}\pgfsetfillopacity{#1}}}%


% Blend mode

\tikzset{blend mode/.code={\tikz@addoption{\pgfsetblendmode{#1}}}}%


% Main color options
\tikzoption{color}{%
  \tikz@addoption{%
    \ifx\tikz@fillcolor\pgfutil@empty%
      \ifx\tikz@strokecolor\pgfutil@empty%
      \else%
        \pgfsys@color@reset@inorderfalse%
        \let\tikz@strokecolor\pgfutil@empty%
        \let\tikz@fillcolor\pgfutil@empty%
      \fi%
    \else%
      \pgfsys@color@reset@inorderfalse%
      \let\tikz@strokecolor\pgfutil@empty%
      \let\tikz@fillcolor\pgfutil@empty%
    \fi%
    \pgfutil@colorlet{tikz@color}{#1}%
    \pgfutil@colorlet{.}{tikz@color}%
    %
    \pgfsetcolor{.}%
    \pgfsys@color@reset@inordertrue%
  }%
  \def\tikz@textcolor{#1}}%



% Rounding options
\tikzoption{rounded corners}[4pt]{\pgfsetcornersarced{\pgfpoint{#1}{#1}}}%
\tikzoption{sharp corners}[]{\pgfsetcornersarced{\pgfpointorigin}}%

% Radii and arc options
\tikzset{x radius/.initial=0pt}%
\tikzset{y radius/.initial=0pt}%
\tikzset{%
  radius/.code={%
    \pgfmathparse{#1}%
    \ifpgfmathunitsdeclared
      \pgfkeyssetevalue{/tikz/x radius}{\pgfmathresult pt}%
      \pgfkeyssetevalue{/tikz/y radius}{\pgfmathresult pt}%
    \else
      \pgfkeyssetevalue{/tikz/x radius}{\pgfmathresult}%
      \pgfkeyssetevalue{/tikz/y radius}{\pgfmathresult}%
    \fi
  }%
}%
\tikzset{start angle/.initial=}%
\tikzset{end angle/.initial=}%
\tikzset{delta angle/.initial=}%


% Coordinate options
\tikzoption{x}{\tikz@handle@vec{\pgfsetxvec}{\tikz@handle@x}#1\relax}%
\tikzoption{y}{\tikz@handle@vec{\pgfsetyvec}{\tikz@handle@y}#1\relax}%
\tikzoption{z}{\tikz@handle@vec{\pgfsetzvec}{\tikz@handle@z}#1\relax}%

\def\tikz@handle@vec#1#2{\pgfutil@ifnextchar({\tikz@handle@coordinate#1}{\tikz@handle@single#2}}%
\def\tikz@handle@coordinate#1{\tikz@scan@one@point#1}%
\def\tikz@handle@single#1#2\relax{#1{#2}}%
\def\tikz@handle@x#1{\pgfsetxvec{\pgfpoint{#1}{0pt}}}%
\def\tikz@handle@y#1{\pgfsetyvec{\pgfpoint{0pt}{#1}}}%
\def\tikz@handle@z#1{\pgfsetzvec{\pgfpoint{#1}{#1}}}%


% Transformation options
\tikzoption{scale}{\tikz@addtransform{\pgftransformscale{#1}}}%
\tikzoption{scale around}{\tikz@addtransform{\def\tikz@aroundaction{\pgftransformscale}\tikz@doaround{#1}}}%
\tikzoption{xscale}{\tikz@addtransform{\pgftransformxscale{#1}}}%
\tikzoption{xslant}{\tikz@addtransform{\pgftransformxslant{#1}}}%
\tikzoption{yscale}{\tikz@addtransform{\pgftransformyscale{#1}}}%
\tikzoption{yslant}{\tikz@addtransform{\pgftransformyslant{#1}}}%
\tikzoption{rotate}{\tikz@addtransform{\pgftransformrotate{#1}}}%
\tikzoption{rotate around}{\tikz@addtransform{\def\tikz@aroundaction{\pgftransformrotate}\tikz@doaround{#1}}}%
\def\tikz@doaround#1{%
  \edef\tikz@temp{#1}% get rid of active stuff
  \expandafter\tikz@doparseA\tikz@temp%
}%
\def\tikz@doparseA#1:{%
  \def\tikz@temp@rot{#1}%
  \tikz@scan@one@point\tikz@doparseB%
}%
\def\tikz@doparseB#1{%
  \pgf@process{#1}%
  \pgf@xc=\pgf@x%
  \pgf@yc=\pgf@y%
  \pgftransformshift{\pgfqpoint{\pgf@xc}{\pgf@yc}}%
  \tikz@aroundaction{\tikz@temp@rot}%
  \pgftransformshift{\pgfqpoint{-\pgf@xc}{-\pgf@yc}}%
}%

\tikzoption{shift}{\tikz@addtransform{\tikz@scan@one@point\pgftransformshift#1\relax}}%
\tikzoption{xshift}{\tikz@addtransform{\pgftransformxshift{#1}}}%
\tikzoption{yshift}{\tikz@addtransform{\pgftransformyshift{#1}}}%
\tikzoption{cm}{\tikz@addtransform{\tikz@parse@cm#1\relax}}%
\tikzoption{reset cm}[]{\tikz@addtransform{\pgftransformreset}}%
\tikzoption{shift only}[]{\tikz@addtransform{\pgftransformresetnontranslations}}%

\def\tikz@parse@cm#1,#2,#3,#4,{%
  \def\tikz@p@cm{{#1}{#2}{#3}{#4}}%
  \tikz@scan@one@point\tikz@parse@cmA}%
\def\tikz@parse@cmA#1{%
  \expandafter\pgftransformcm\tikz@p@cm{#1}%
}%

\tikzset{transform canvas/.code=%
  {%
    \tikz@addoption
    {%
      {%
        \pgftransformreset%
        \let\tikz@transform=\relax%
        \tikzset{#1}%
        \pgflowlevelsynccm%
      }%
      \pgf@relevantforpicturesizefalse%
    }%
  }%
}%

\tikzset{turn/.code={%
    \pgf@x=0pt%
    \pgf@y\pgf@x%
    \pgf@process{\tikz@tangent}%
    \advance\pgf@x by-\tikz@lastx%
    \advance\pgf@y by-\tikz@lasty%
    \pgfpointnormalised{}% x/y = normalised vector
    \pgf@x=-\pgf@x%
    \pgf@ya=-\pgf@y%
    \pgftransformcm%
    {\pgf@sys@tonumber{\pgf@x}}{\pgf@sys@tonumber{\pgf@ya}}%
    {\pgf@sys@tonumber{\pgf@y}}{\pgf@sys@tonumber{\pgf@x}}{\pgfqpoint{\tikz@lastx}{\tikz@lasty}}%
  }%
}%

\def\tikz@tangent@lookup{%
  \pgfgetpath\tikz@temp%
  \pgfprocesspathextractpoints\tikz@temp%
  \pgfpointsecondlastonpath%
}%

% Code for rotating the xyz coordinate system
% around the x, y, or z vector.
%
\def\tikz@xyz@rotate@let{%
  \let\pgf@z=\pgf@yc%
  \let\pgf@za=\pgf@xc%
}%

\def\tikz@xyz@rotate@xyz@xaxis#1#2#3#4{%
  \tikz@xyz@rotate@let%
  \pgf@x=#1\relax%
  \pgf@ya=#2\relax%
  \pgf@za=#3\relax%
  \pgfmathsin@{#4}\let\tikz@xyz@sin=\pgfmathresult%
  \pgfmathcos@{#4}\let\tikz@xyz@cos=\pgfmathresult%
  \pgf@y=\tikz@xyz@cos\pgf@ya%
  \advance\pgf@y by-\tikz@xyz@sin\pgf@za%
  \pgf@z=\tikz@xyz@sin\pgf@ya%
  \advance\pgf@z by\tikz@xyz@cos\pgf@za%
}%

\def\tikz@xyz@rotate@xyz@yaxis#1#2#3#4{%
  \tikz@xyz@rotate@let%
  \pgf@xa=#1\relax%
  \pgf@y=#2\relax%
  \pgf@za=#3\relax%
  \pgfmathsin@{#4}\let\tikz@xyz@sin=\pgfmathresult%
  \pgfmathcos@{#4}\let\tikz@xyz@cos=\pgfmathresult%
  \pgf@x=\tikz@xyz@cos\pgf@xa%
  \advance\pgf@x by\tikz@xyz@sin\pgf@za%
  \pgf@z=-\tikz@xyz@sin\pgf@xa%
  \advance\pgf@z by\tikz@xyz@cos\pgf@za%
}%

\def\tikz@xyz@rotate@xyz@zaxis#1#2#3#4{%
  \tikz@xyz@rotate@let%
  \pgf@xa=#1\relax%
  \pgf@ya=#2\relax%
  \pgf@z=#3\relax%
  \pgfmathsin@{#4}\let\tikz@xyz@sin=\pgfmathresult%
  \pgfmathcos@{#4}\let\tikz@xyz@cos=\pgfmathresult%
  \pgf@x=\tikz@xyz@cos\pgf@xa%
  \advance\pgf@x by-\tikz@xyz@sin\pgf@ya%
  \pgf@y=\tikz@xyz@sin\pgf@xa%
  \advance\pgf@y by\tikz@xyz@cos\pgf@ya%
}%

\tikzset{rotate around x/.code={%
    \tikz@xyz@rotate@let%
    \pgfmathparse{#1}\let\tikz@xyz@angle=\pgfmathresult%
    \tikz@xyz@rotate@xyz@xaxis{0pt}{1pt}{0pt}{\tikz@xyz@angle}%
    \pgfextract@process\tikz@xyz@rotate@yvec{\pgfpointxyz{\pgf@sys@tonumber{\pgf@x}}{\pgf@sys@tonumber{\pgf@y}}{\pgf@sys@tonumber{\pgf@z}}}%
    \tikz@xyz@rotate@xyz@xaxis{0pt}{0pt}{1pt}{\tikz@xyz@angle}%
    \pgfsetzvec{\pgfpointxyz{\pgf@sys@tonumber{\pgf@x}}{\pgf@sys@tonumber{\pgf@y}}{\pgf@sys@tonumber{\pgf@z}}}%
    \pgfsetyvec{\tikz@xyz@rotate@yvec}%
  },
  rotate around y/.code={%
    \tikz@xyz@rotate@let%
    \pgfmathparse{#1}\let\tikz@xyz@angle=\pgfmathresult%
    \tikz@xyz@rotate@xyz@yaxis{1pt}{0pt}{0pt}{\tikz@xyz@angle}%
    \pgfextract@process\tikz@xyz@rotate@xvec{\pgfpointxyz{\pgf@sys@tonumber{\pgf@x}}{\pgf@sys@tonumber{\pgf@y}}{\pgf@sys@tonumber{\pgf@z}}}%
    \tikz@xyz@rotate@xyz@yaxis{0pt}{0pt}{1pt}{\tikz@xyz@angle}%
    \pgfsetzvec{\pgfpointxyz{\pgf@sys@tonumber{\pgf@x}}{\pgf@sys@tonumber{\pgf@y}}{\pgf@sys@tonumber{\pgf@z}}}%
    \pgfsetxvec{\tikz@xyz@rotate@xvec}%
  },
  rotate around z/.code={%
    \tikz@xyz@rotate@let%
    \pgfmathparse{#1}\let\tikz@xyz@angle=\pgfmathresult%
    \tikz@xyz@rotate@xyz@zaxis{1pt}{0pt}{0pt}{\tikz@xyz@angle}%
    \pgfextract@process\tikz@xyz@rotate@xvec{\pgfpointxyz{\pgf@sys@tonumber{\pgf@x}}{\pgf@sys@tonumber{\pgf@y}}{\pgf@sys@tonumber{\pgf@z}}}%
    \tikz@xyz@rotate@xyz@zaxis{0pt}{1pt}{0pt}{\tikz@xyz@angle}%
    \pgfsetyvec{\pgfpointxyz{\pgf@sys@tonumber{\pgf@x}}{\pgf@sys@tonumber{\pgf@y}}{\pgf@sys@tonumber{\pgf@z}}}%
    \pgfsetxvec{\tikz@xyz@rotate@xvec}%
  },
}%

% Grid options
\tikzoption{xstep}{\def\tikz@grid@x{#1}}%
\tikzoption{ystep}{\def\tikz@grid@y{#1}}%
\tikzoption{step}{\tikz@handle@vec{\tikz@step@point}{\tikz@step@single}#1\relax}%
\def\tikz@step@single#1{\def\tikz@grid@x{#1}\def\tikz@grid@y{#1}}%
\def\tikz@step@point#1{\pgf@process{#1}\edef\tikz@grid@x{\the\pgf@x}\edef\tikz@grid@y{\the\pgf@y}}%

\def\tikz@grid@x{1cm}%
\def\tikz@grid@y{1cm}%


% Current point updates
\newif\iftikz@current@point@local
\tikzset{current point is local/.is if=tikz@current@point@local}%

% Path usage options
\newif\iftikz@mode@double
\newif\iftikz@mode@fill
\newif\iftikz@mode@draw
\newif\iftikz@mode@clip
\newif\iftikz@mode@boundary
\newif\iftikz@mode@shade
\newif\iftikz@mode@fade@path
\newif\iftikz@mode@fade@scope
\let\tikz@mode=\pgfutil@empty

\def\tikz@nonetext{none}%

\tikzoption{path only}[]{\let\tikz@mode=\pgfutil@empty}%
\tikzset{
  shade/.is choice,
  shade/.default=true,
  shade/true/.code=\tikz@addmode{\tikz@mode@shadetrue},
  shade/false/.code=\tikz@addmode{\tikz@mode@shadefalse},
  shade/none/.code=\tikz@addmode{\tikz@mode@shadefalse},
}%

\tikzoption{fill}[]{%
  \edef\tikz@temp{#1}%
  \ifx\tikz@temp\tikz@nonetext%
    \tikz@addmode{\tikz@mode@fillfalse}%
  \else%
    \ifx\tikz@temp\pgfutil@empty%
    \else%
      \tikz@addoption{\pgfsetfillcolor{#1}}%
      \def\tikz@fillcolor{#1}%
    \fi%
    \tikz@addmode{\tikz@mode@filltrue}%
  \fi%
}%
\tikzoption{draw}[]{%
  \edef\tikz@temp{#1}%
  \ifx\tikz@temp\tikz@nonetext%
    \tikz@addmode{\tikz@mode@drawfalse}%
  \else%
    \ifx\tikz@temp\pgfutil@empty%
    \else%
      \tikz@addoption{\pgfsetstrokecolor{#1}}%
      \def\tikz@strokecolor{#1}%
    \fi%
    \tikz@addmode{\tikz@mode@drawtrue}%
  \fi%
}%
\tikzoption{clip}[]{\tikz@addmode{\tikz@mode@cliptrue}}%
\tikzoption{use as bounding box}[]{\tikz@addmode{\tikz@mode@boundarytrue}}%

\tikzoption{save path}{\tikz@addmode{\pgfsyssoftpath@getcurrentpath#1\global\let#1=#1}}%
\tikzoption{use path}{\tikz@addmode{\pgfsyssoftpath@setcurrentpath#1}}%

\let\tikz@fillcolor=\pgfutil@empty
\let\tikz@strokecolor=\pgfutil@empty

% Insert a path using an option
\tikzset{insert path/.code=\tikz@scan@next@command#1\pgf@stop}%

% Pattern options
\tikzset{pattern/.code=\tikzerror{You need to say \string\usetikzlibrary{patterns}},
         pattern color/.style=pattern}%

% Path pictures
\tikzset{path picture/.code=\tikz@addmode{\def\tikz@path@picture{#1}}}%

% Fading options
\tikzset{path fading/.code={
  \def\tikz@temp{#1}%
  \ifx\tikz@temp\tikz@nonetext%
    \tikz@addmode{\tikz@mode@fade@pathfalse}%
  \else%
    \ifx\tikz@temp\pgfutil@empty%
    \else%
      \def\tikz@path@fading{#1}%
    \fi%
    \tikz@addmode{\tikz@mode@fade@pathtrue}%
  \fi%
  },
  path fading/.default=,
  scope fading/.code={
  \def\tikz@temp{#1}%
  \ifx\tikz@temp\tikz@nonetext%
    \tikz@addmode{\tikz@mode@fade@scopefalse}%
  \else%
    \ifx\tikz@temp\pgfutil@empty%
    \else%
      \def\tikz@scope@fading{#1}%
    \fi%
    \tikz@addmode{\tikz@mode@fade@scopetrue}%
  \fi%
  },
  scope fading/.default=}%
\tikzset{fit fading/.is if=tikz@fade@adjust}%
\tikzset{fading transform/.store in=\tikz@fade@transform}%
\tikzset{fading angle/.style={fading transform={rotate={#1}}}}%

\newif\iftikz@fade@adjust%
\tikz@fade@adjusttrue%
\let\tikz@fade@transform\pgfutil@empty%

\pgfutil@colorlet{transparent}{pgftransparent}%
\def\tikz@do@fade@transform{\let\tikz@transform=\relax\expandafter\tikzset\expandafter{\tikz@fade@transform}}%



% Transparency groups
\newif\iftikz@transparency@group%
\tikzset{/tikz/transparency group/.code=\tikz@transparency@grouptrue\def\tikz@transparency@group@options{isolated=true,#1}\let\tikz@blend@group\pgfutil@empty}%
\tikzset{/tikz/blend group/.code=\tikz@transparency@grouptrue\def\tikz@transparency@group@options{isolated=true}\def\tikz@blend@group{\pgfsetblendmode{#1}}}%

\let\tikz@blend@group\pgfutil@empty

% Shading options
\tikzoption{shading}{\def\tikz@shading{#1}\tikz@addmode{\tikz@mode@shadetrue}}%
\tikzoption{shading angle}{\def\tikz@shade@angle{#1}\tikz@addmode{\tikz@mode@shadetrue}}%
\tikzoption{top color}{%
  \pgfutil@colorlet{tikz@axis@top}{#1}%
  \pgfutil@colorlet{tikz@axis@middle}{tikz@axis@top!50!tikz@axis@bottom}%
  \def\tikz@shading{axis}\def\tikz@shade@angle{0}\tikz@addmode{\tikz@mode@shadetrue}}%
\tikzoption{bottom color}{%
  \pgfutil@colorlet{tikz@axis@bottom}{#1}%
  \pgfutil@colorlet{tikz@axis@middle}{tikz@axis@top!50!tikz@axis@bottom}%
  \def\tikz@shading{axis}\def\tikz@shade@angle{0}\tikz@addmode{\tikz@mode@shadetrue}}%
\tikzoption{middle color}{%
  \pgfutil@colorlet{tikz@axis@middle}{#1}%
  \def\tikz@shading{axis}\tikz@addmode{\tikz@mode@shadetrue}}%
\tikzoption{left color}{%
  \pgfutil@colorlet{tikz@axis@top}{#1}%
  \pgfutil@colorlet{tikz@axis@middle}{tikz@axis@top!50!tikz@axis@bottom}%
  \def\tikz@shading{axis}\def\tikz@shade@angle{90}\tikz@addmode{\tikz@mode@shadetrue}}%
\tikzoption{right color}{%
  \pgfutil@colorlet{tikz@axis@bottom}{#1}%
  \pgfutil@colorlet{tikz@axis@middle}{tikz@axis@top!50!tikz@axis@bottom}%
  \def\tikz@shading{axis}\def\tikz@shade@angle{90}\tikz@addmode{\tikz@mode@shadetrue}}%
\tikzoption{ball color}{\pgfutil@colorlet{tikz@ball}{#1}\def\tikz@shading{ball}\tikz@addmode{\tikz@mode@shadetrue}}%
\tikzoption{inner color}{\pgfutil@colorlet{tikz@radial@inner}{#1}\def\tikz@shading{radial}\tikz@addmode{\tikz@mode@shadetrue}}%
\tikzoption{outer color}{\pgfutil@colorlet{tikz@radial@outer}{#1}\def\tikz@shading{radial}\tikz@addmode{\tikz@mode@shadetrue}}%

\def\tikz@shading{axis}%
\def\tikz@shade@angle{0}%

\pgfdeclareverticalshading[tikz@axis@top,tikz@axis@middle,tikz@axis@bottom]{axis}{100bp}{%
  color(0bp)=(tikz@axis@bottom);
  color(25bp)=(tikz@axis@bottom);
  color(50bp)=(tikz@axis@middle);
  color(75bp)=(tikz@axis@top);
  color(100bp)=(tikz@axis@top)}%

\pgfutil@colorlet{tikz@axis@top}{gray}%
\pgfutil@colorlet{tikz@axis@middle}{gray!50!white}%
\pgfutil@colorlet{tikz@axis@bottom}{white}%

\pgfdeclareradialshading[tikz@ball]{ball}{\pgfqpoint{-10bp}{10bp}}{%
 color(0bp)=(tikz@ball!15!white);
 color(9bp)=(tikz@ball!75!white);
 color(18bp)=(tikz@ball!70!black);
 color(25bp)=(tikz@ball!50!black);
 color(50bp)=(black)}%

\pgfutil@colorlet{tikz@ball}{blue}%

\pgfdeclareradialshading[tikz@radial@inner,tikz@radial@outer]{radial}{\pgfpointorigin}{%
 color(0bp)=(tikz@radial@inner);
 color(25bp)=(tikz@radial@outer);
 color(50bp)=(tikz@radial@outer)}%

\pgfutil@colorlet{tikz@radial@inner}{gray}%
\pgfutil@colorlet{tikz@radial@outer}{white}%


% Pin options
\tikzset{
  pin distance/.store in=\tikz@pin@distance,
  pin distance=3ex,
  pin position/.store in=\tikz@pin@default@pos,
  pin position=above,
  pin edge/.store in=\tikz@pin@edge@style,
  pin edge={},
  tikz@pin@post/.code={\global\let\tikz@pin@edge@style@smuggle=\tikz@pin@edge@style},
  tikz@pre@pin@edge/.code={%
    \toks0=\expandafter{\tikz@pin@edge@style@smuggle}%
    \edef\pgf@marshal{\noexpand\tikzset{tikz@pin@options/.style={\the\toks0}}}%
    \pgf@marshal
  },%
}%

\tikzset{%
  pin/.code={%
    \begingroup
      \ifnum\the\catcode`\:=\active\relax
        \def\tikz@next{%
          \endgroup
          \tikz@parse@pin@active@i{#1}}%
      \else
        \def\tikz@next{%
          \endgroup
          \pgfutil@ifnextchar[%]
            {\tikz@parse@pin@nonactive}
            {\tikz@parse@pin@nonactive[]}#1:\pgf@nil}%
      \fi
    \tikz@next}}%

\begingroup
  \catcode`\:=\active\relax

  \gdef\tikz@parse@pin@active@i#1{%
    \pgfutil@ifnextchar[%]
      {\tikz@parse@pin@active}
      {\tikz@parse@pin@active[]}#1:\pgf@nil}%

  \long\gdef\tikz@parse@pin@active[#1]#2:#3\pgf@nil{%
    \def\tikz@temp{#3}%
    \ifx\tikz@temp\pgfutil@empty
      % no position, use default
      \tikz@@parse@pin@active[#1]\tikz@pin@default@pos:#2:\pgf@nil%
    \else
      \tikz@@parse@pin@active[#1]#2:#3\pgf@nil%
    \fi}%

  \long\gdef\tikz@@parse@pin@active[#1]#2:#3:\pgf@nil{%
    \tikzset{%
      append after command = {%
        \bgroup
          [current point is local = true]
          \pgfextra{\let\tikz@save@last@node=\tikzlastnode\tikz@node@is@a@labelfalse}%
          node [tikz@label@angle = #2,
                anchor=@auto,
                every pin,
                #1,
                append after command = {%
                  (\tikz@save@last@node)
                  edge [every pin edge,
                        tikz@pre@pin@edge,
                        tikz@pin@options]
                  (\tikzlastnode)},
                tikz@label@post = \tikz@pin@distance,
                tikz@pin@post] {#3}
        \egroup}}}%
\endgroup

\long\def\tikz@parse@pin@nonactive[#1]#2:#3\pgf@nil{%
  \def\tikz@temp{#3}%
  \ifx\tikz@temp\pgfutil@empty
    % no position, use default
    \tikz@@parse@pin@nonactive[#1]\tikz@pin@default@pos:#2:\pgf@nil%
  \else
    \tikz@@parse@pin@nonactive[#1]#2:#3\pgf@nil%
  \fi}%

\long\def\tikz@@parse@pin@nonactive[#1]#2:#3:\pgf@nil{%
  \tikzset{%
    append after command = {%
      \bgroup
        [current point is local = true]
        \pgfextra{\let\tikz@save@last@node=\tikzlastnode\tikz@node@is@a@labelfalse}%
        node [tikz@label@angle = #2,
              anchor=@auto,
              every pin,
              #1,
              append after command = {%
                (\tikz@save@last@node)
                edge [every pin edge,
                      tikz@pre@pin@edge,
                      tikz@pin@options]
                (\tikzlastnode)},
              tikz@label@post = \tikz@pin@distance,
              tikz@pin@post] {#3}
      \egroup}}}%

% Label and pin options

\tikzset{
  label distance/.store in=\tikz@label@distance,
  label distance=0pt,
  label position/.store in=\tikz@label@default@pos,
  label position=above,
  absolute/.is if=tikz@absolute,
  tikz@label@angle/.store in=\tikz@label@angle
}%

\newif\iftikz@absolute
\def\tikz@on@text{center}%

\tikzset{tikz@label@post/.code 2 args={
  \edef\tikz@label@angle{\tikz@label@angle}%
  \expandafter\pgfkeys@spdef\expandafter\tikz@label@angle\expandafter{\tikz@label@angle}%
  \csname tikz@label@angle@is@\tikz@label@angle\endcsname
  \ifx\tikz@label@angle\tikz@on@text%
    \def\tikz@node@at{\pgfpointanchor{\tikzlastnode}{center}}%
    \def\tikz@anchor{center}%
  \else%
    \iftikz@absolute%
      \pgftransformreset%
      \pgf@process{%
        \pgfpointshapeborder{\tikzlastnode}%
        {\pgfpointadd{\pgfpointanchor{\tikzlastnode}{center}}{\pgfpointpolar{\tikz@label@angle}{1pt}}}}%
      \edef\tikz@node@at{\noexpand\pgfqpoint{\the\pgf@x}{\the\pgf@y}}%
      \tikz@compute@direction{\tikz@label@angle}%
      \tikz@addtransform{\pgftransformshift{\pgfpointpolar{\tikz@label@angle}{#1}}}%
    \else%
      \pgf@process{\pgfpointanchor{\tikzlastnode}{\tikz@label@angle}}%
      \edef\tikz@node@at{\noexpand\pgfqpoint{\the\pgf@x}{\the\pgf@y}}%
      \pgf@xb=\pgf@x%
      \pgf@yb=\pgf@y%
      \pgf@process{\pgfpointanchor{\tikzlastnode}{center}}%
      \pgf@xc=\pgf@x%
      \pgf@yc=\pgf@y%
      \tikz@label@simplefalse%
      \ifdim\pgf@xc=\pgf@xb\relax%
        \ifdim\pgf@yc=\pgf@yb\relax%
          \tikz@label@simpletrue%
        \fi%
      \fi%
      \iftikz@label@simple%
        \tikz@compute@direction{\tikz@label@angle}%
        \tikz@addtransform{\pgftransformshift{\pgfpointpolar{\tikz@label@angle}{#1}}}%
      \else%
        \pgf@process{\pgfpointnormalised{%
            \pgfpointdiff{\pgfpointtransformed{\pgfqpoint{\pgf@xc}{\pgf@yc}}}{\pgfpointtransformed{\pgfqpoint{\pgf@xb}{\pgf@yb}}}}}%
        \edef\pgf@marshal{%
          \noexpand\tikz@addtransform{\noexpand\pgftransformshift{\noexpand\pgfpointscale{#1}{
                \noexpand\pgfqpoint{\the\pgf@x}{\the\pgf@y}}}}}%
        \pgf@marshal%
        \pgf@xc=\pgf@x%
        \pgf@yc=\pgf@y%
        \pgf@x=\pgf@yc%
        \pgf@y=-\pgf@xc%
        \ifx\tikz@anchor\tikz@auto@text%
          \tikz@auto@anchor%
        \fi%
      \fi%
    \fi%
  \fi}
}%

\newif\iftikz@label@simple%

\tikzset{%
  label/.code={%
    \begingroup
      \ifnum\the\catcode`\:=\active\relax
        \def\tikz@next{%
          \endgroup
          \tikz@parse@label@active@i{#1}}%
      \else
        \def\tikz@next{%
          \endgroup
          \pgfutil@ifnextchar[%]
            {\tikz@parse@label@nonactive}
            {\tikz@parse@label@nonactive[]}#1:\pgf@nil}%
      \fi
    \tikz@next}}%

\begingroup
  \catcode`\:=\active\relax

  \gdef\tikz@parse@label@active@i#1{%
    \pgfutil@ifnextchar[%]
      {\tikz@parse@label@active}
      {\tikz@parse@label@active[]}#1:\pgf@nil}%

  \gdef\tikz@parse@label@active[#1]#2:#3\pgf@nil{%
    \def\tikz@temp{#3}%
    \ifx\tikz@temp\pgfutil@empty
    % no position, use default
      \tikz@@parse@label@active[#1]\tikz@label@default@pos:#2:\pgf@nil%
    \else
      \def\tikz@temp{#2}%
      \ifx\tikz@temp\pgfutil@empty
        \tikz@@parse@label@active[#1]\tikz@label@default@pos:#3\pgf@nil%
      \else
        \tikz@@parse@label@active[#1]#2:#3\pgf@nil%
      \fi
    \fi
  }%

  \gdef\tikz@@parse@label@active[#1]#2:#3:\pgf@nil{%
    \tikzset{%
      append after command = {%
        \bgroup
          [current point is local=true]
          \pgfextra{\let\tikz@save@last@fig@name=\tikz@last@fig@name\tikz@node@is@a@labelfalse}
          node [tikz@label@angle = #2,
                anchor=@auto,
                every label,
                #1,
                tikz@label@post = \tikz@label@distance] {\iftikz@handle@active@nodes\expandafter\scantokens\else\expandafter\pgfutil@firstofone\fi{#3\noexpand}}
          \pgfextra{\global\let\tikz@last@fig@name=\tikz@save@last@fig@name}
        \egroup}}}%
\endgroup

\def\tikz@parse@label@nonactive[#1]#2:#3\pgf@nil{%
  \def\tikz@temp{#3}%
  \ifx\tikz@temp\pgfutil@empty
    % no position, use default
    \tikz@@parse@label@nonactive[#1]\tikz@label@default@pos:#2:\pgf@nil%
  \else
    \def\tikz@temp{#2}%
    \ifx\tikz@temp\pgfutil@empty
      \tikz@@parse@label@nonactive[#1]\tikz@label@default@pos:#3\pgf@nil%
    \else
      \tikz@@parse@label@nonactive[#1]#2:#3\pgf@nil%
    \fi
  \fi
}%

\def\tikz@@parse@label@nonactive[#1]#2:#3:\pgf@nil{%
  \tikzset{%
    append after command = {%
      \bgroup
        [current point is local=true]
        \pgfextra{\let\tikz@save@last@fig@name=\tikz@last@fig@name\tikz@node@is@a@labelfalse}
        node [tikz@label@angle = #2,
              anchor=@auto,
              every label,
              #1,
              tikz@label@post = \tikz@label@distance] {\iftikz@handle@active@nodes\expandafter\scantokens\else\expandafter\pgfutil@firstofone\fi{#3\noexpand}}
        \pgfextra{\global\let\tikz@last@fig@name=\tikz@save@last@fig@name}
      \egroup}}}%

\expandafter\def\csname tikz@label@angle@is@right\endcsname{\def\tikz@label@angle{0}}%
\expandafter\def\csname tikz@label@angle@is@above right\endcsname{\def\tikz@label@angle{45}}%
\expandafter\def\csname tikz@label@angle@is@above\endcsname{\def\tikz@label@angle{90}}%
\expandafter\def\csname tikz@label@angle@is@above left\endcsname{\def\tikz@label@angle{135}}%
\expandafter\def\csname tikz@label@angle@is@left\endcsname{\def\tikz@label@angle{180}}%
\expandafter\def\csname tikz@label@angle@is@below left\endcsname{\def\tikz@label@angle{225}}%
\expandafter\def\csname tikz@label@angle@is@below\endcsname{\def\tikz@label@angle{270}}%
\expandafter\def\csname tikz@label@angle@is@below right\endcsname{\def\tikz@label@angle{315}}%

\def\tikz@compute@direction#1{%
  \ifx\tikz@anchor\tikz@auto@text%
  \let\tikz@do@auto@anchor=\relax
  \pgfmathsetcount{\c@pgf@counta}{#1}%
  \ifnum\c@pgf@counta<0\relax
    \advance\c@pgf@counta by 360\relax%
  \fi%
  \ifnum\c@pgf@counta>359\relax
    \advance\c@pgf@counta by-360\relax%
  \fi%
  \ifnum\c@pgf@counta<4\relax%
    \def\tikz@anchor{west}%
  \else\ifnum\c@pgf@counta<87\relax%
    \def\tikz@anchor{south west}%
  \else\ifnum\c@pgf@counta<94\relax%
    \def\tikz@anchor{south}%
  \else\ifnum\c@pgf@counta<177\relax%
    \def\tikz@anchor{south east}%
  \else\ifnum\c@pgf@counta<184\relax%
    \def\tikz@anchor{east}%
  \else\ifnum\c@pgf@counta<267\relax%
    \def\tikz@anchor{north east}%
  \else\ifnum\c@pgf@counta<274\relax%
    \def\tikz@anchor{north}%
  \else\ifnum\c@pgf@counta<357\relax%
    \def\tikz@anchor{north west}%
  \else%
    \def\tikz@anchor{west}%
  \fi\fi\fi\fi\fi\fi\fi\fi%
  \fi%
}%
\def\tikz@auto@text{@auto}%

% General node options
\tikzset{
  name/.code={\edef\tikz@fig@name{\tikz@pp@name{#1}}\let\tikz@id@name\tikz@fig@name},%
  name prefix/.initial=,%
  name suffix/.initial=,%
  local bounding box/.style={/pgf/local bounding box/.expanded=\tikz@pp@name{#1}}
}%
\def\tikz@pp@name#1{\csname pgfk@/tikz/name prefix\endcsname#1\csname pgfk@/tikz/name suffix\endcsname}%


\tikzset{
  node contents/.code=\def\tikz@node@content{#1},
  pic type/.code=\def\tikz@node@content{#1}, % alias
}%

\tikzset{
  behind path/.code=\def\tikz@whichbox{\tikz@figbox@bg},
  in front of path/.code=\def\tikz@whichbox{\tikz@figbox}
}%
\def\tikz@whichbox{\tikz@figbox}%

\tikzoption{at}{\tikz@scan@one@point\tikz@set@at#1}%
\def\tikz@set@at#1{\def\tikz@node@at{#1}}%

\tikzoption{shape}{\edef\tikz@shape{#1}}%

\tikzoption{nodes}{\tikzset{every node/.append style={#1}}}%

\tikzset{alias/.code={%
    \tikz@fig@mustbenamed
    \begingroup
    \toks0=\expandafter{\tikz@alias}%
    \edef\pgf@temp{\noexpand\pgfnodealias{\tikz@pp@name{#1}}{\noexpand\tikz@fig@name}}%
    \toks1=\expandafter{\pgf@temp}%
    \xdef\pgf@marshal{%
        \noexpand\def\noexpand\tikz@alias{\the\toks0 \the\toks1 }%
    }%
    \endgroup
    \pgf@marshal
}}%

% deprecated:
\def\tikzaddafternodepathoption#1{#1\tikzset{prefix after command={\pgfextra{#1}}}}%
\tikzset{after node path/.style={append after command={#1}}}%


% Pic options
\tikzset{pic text/.store in=\tikzpictext}%
\let\tikzpictext\relax
\tikzset{pic text options/.store in=\tikzpictextoptions}%
\let\tikzpictextoptions\pgfutil@empty


% Anchoring

\tikzoption{anchor}{\def\tikz@anchor{#1}\let\tikz@do@auto@anchor=\relax}%

\tikzoption{left}[]{\def\tikz@anchor{east}\tikz@possibly@transform{x}{-}{#1}}%
\tikzoption{right}[]{\def\tikz@anchor{west}\tikz@possibly@transform{x}{}{#1}}%
\tikzoption{above}[]{\def\tikz@anchor{south}\tikz@possibly@transform{y}{}{#1}}%
\tikzoption{below}[]{\def\tikz@anchor{north}\tikz@possibly@transform{y}{-}{#1}}%
\tikzoption{above left}[]%
  {\def\tikz@anchor{south east}%
    \tikz@possibly@transform{x}{-}{#1}\tikz@possibly@transform{y}{}{#1}}%
\tikzoption{above right}[]%
  {\def\tikz@anchor{south west}%
    \tikz@possibly@transform{x}{}{#1}\tikz@possibly@transform{y}{}{#1}}%
\tikzoption{below left}[]%
  {\def\tikz@anchor{north east}%
    \tikz@possibly@transform{x}{-}{#1}\tikz@possibly@transform{y}{-}{#1}}%
\tikzoption{below right}[]%
  {\def\tikz@anchor{north west}%
    \tikz@possibly@transform{x}{}{#1}\tikz@possibly@transform{y}{-}{#1}}%
\tikzset{centered/.code=\def\tikz@anchor{center}}%

\tikzoption{node distance}{\def\tikz@node@distance{#1}}%
\def\tikz@node@distance{1cm}%

% The following are deprecated:
\tikzoption{above of}{\tikz@of{#1}{90}}%
\tikzoption{below of}{\tikz@of{#1}{-90}}%
\tikzoption{left of}{\tikz@of{#1}{180}}%
\tikzoption{right of}{\tikz@of{#1}{0}}%
\tikzoption{above left of}{\tikz@of{#1}{135}}%
\tikzoption{below left of}{\tikz@of{#1}{-135}}%
\tikzoption{above right of}{\tikz@of{#1}{45}}%
\tikzoption{below right of}{\tikz@of{#1}{-45}}%
\def\tikz@of#1#2{%
  \def\tikz@anchor{center}%
  \let\tikz@do@auto@anchor=\relax%
  \tikz@addtransform{%
    \expandafter\tikz@extract@node@dist\tikz@node@distance and\pgf@stop%
    \pgftransformshift{\pgfpointpolar{#2}{\tikz@extracted@node@distance}}}%
  \def\tikz@node@at{\pgfpointanchor{\tikz@pp@name{#1}}{center}}}%
\def\tikz@extract@node@dist#1and#2\pgf@stop{%
  \def\tikz@extracted@node@distance{#1}}%

\tikzset{
  transform shape nonlinear/.is choice,
  transform shape nonlinear/.default=true,
  transform shape nonlinear/true/.code=\let\tikz@nlt\relax,
  transform shape nonlinear/false/.code=\def\tikz@nlt{\pgfapproximatenonlineartranslation},
  transform shape nonlinear=false,
}%


\tikzoption{transform shape}[true]{%
  \csname tikz@fullytransformed#1\endcsname%
  \iftikz@fullytransformed%
    \pgfresetnontranslationattimefalse%
  \else%
    \pgfresetnontranslationattimetrue%
  \fi%
}%

\newif\iftikz@fullytransformed
\pgfresetnontranslationattimetrue%

\def\tikz@anchor{center}%
\def\tikz@shape{rectangle}%

\def\tikz@possibly@transform#1#2#3{%
  \let\tikz@do@auto@anchor=\relax%
  \def\tikz@test{#3}%
  \ifx\tikz@test\pgfutil@empty%
  \else%
    \pgfmathsetlength{\pgf@x}{#3}%
    \pgf@x=#2\pgf@x\relax%
    \edef\tikz@marshal{\noexpand\tikz@addtransform{%
        \expandafter\noexpand\csname  pgftransform#1shift\endcsname{\the\pgf@x}}}%
    \tikz@marshal%
  \fi%
}%


% Inter-picture options
\tikzoption{remember picture}[true]{\csname pgfrememberpicturepositiononpage#1\endcsname}
\tikzset{
    overlay/.is choice,
    overlay/true/.code={\pgf@relevantforpicturesizefalse},
    overlay/false/.code={\pgf@relevantforpicturesizetrue},
    overlay/.default=true
}



% Line/curve label placement options
\tikzoption{sloped}[true]{\csname pgfslopedattime#1\endcsname}%
\tikzoption{allow upside down}[true]{\csname pgfallowupsidedownattime#1\endcsname}%

\tikzoption{pos}{\edef\tikz@time{#1}\ifx\tikz@time\pgfutil@empty\else\pgfmathsetmacro\tikz@time{\tikz@time}\fi}%

\tikzoption{auto}[]{\csname tikz@install@auto@anchor@#1\endcsname}%
\tikzoption{swap}[]{%
  \def\tikz@temp{left}%
  \ifx\tikz@auto@anchor@direction\tikz@temp%
    \def\tikz@auto@anchor@direction{right}%
  \else%
    \def\tikz@auto@anchor@direction{left}%
  \fi%
}%
\tikzset{'/.style=swap}% shorthand


\def\tikz@install@auto@anchor@{\let\tikz@do@auto@anchor=\tikz@auto@anchor@on}%
\def\tikz@install@auto@anchor@false{\let\tikz@do@auto@anchor=\relax}%
\def\tikz@install@auto@anchor@left{\let\tikz@do@auto@anchor=\tikz@auto@anchor@on\def\tikz@auto@anchor@direction{left}}%
\def\tikz@install@auto@anchor@right{\let\tikz@do@auto@anchor=\tikz@auto@anchor@on\def\tikz@auto@anchor@direction{right}}%

\let\tikz@do@auto@anchor=\relax%

\def\tikz@auto@anchor@on{\csname tikz@auto@anchor@\tikz@auto@anchor@direction\endcsname}

\def\tikz@auto@anchor@left{\tikz@auto@pre\tikz@auto@anchor\tikz@auto@post}%
\def\tikz@auto@anchor@right{\tikz@auto@pre\tikz@auto@anchor@prime\tikz@auto@post}%

\def\tikz@auto@anchor@direction{left}%

% Text options
\tikzoption{text}{\def\tikz@textcolor{#1}}%
\tikzoption{font}{\def\tikz@textfont{#1}}%
\tikzoption{node font}{\def\tikz@node@textfont{#1}}%
\tikzoption{text opacity}{\def\tikz@textopacity{#1}}%
\tikzoption{text width}{\def\tikz@text@width{#1}}%
\tikzoption{text height}{\def\tikz@text@height{#1}}%
\tikzoption{text depth}{\def\tikz@text@depth{#1}}%
\tikzoption{text ragged}[]%
{\def\tikz@text@action{\pgfutil@raggedright\rightskip0pt plus2em \spaceskip.3333em \xspaceskip.5em\relax}}%
\tikzoption{text badly ragged}[]{\def\tikz@text@action{\pgfutil@raggedright\relax}}%
\tikzoption{text ragged left}[]%
{\def\tikz@text@action{\pgfutil@raggedleft\leftskip0pt plus2em \spaceskip.3333em \xspaceskip.5em\relax}}%
\tikzoption{text badly ragged left}[]{\def\tikz@text@action{\pgfutil@raggedleft\relax}}%
\tikzoption{text justified}[]{\def\tikz@text@action{\leftskip0pt\rightskip0pt\relax}}%
\tikzoption{text centered}[]{\def\tikz@text@action{%
  \leftskip0pt plus2em%
  \rightskip0pt plus2em%
  \spaceskip.3333em \xspaceskip.5em%
  \parfillskip=0pt%
  \iftikz@warn@for@narrow@centered\else\hbadness10000\fi%
  \let\\=\@centercr% for latex
  \relax}}%
\tikzoption{text badly centered}[]%
{\def\tikz@text@action{%
  \let\\=\@centercr% for latex
  \parfillskip=0pt%
  \rightskip\pgfutil@flushglue%
  \leftskip\pgfutil@flushglue\relax}}%
\tikzset{badness warnings for centered text/.is if=tikz@warn@for@narrow@centered}%
\newif\iftikz@warn@for@narrow@centered

\def\tikz@text@reset{%
  \let\tikz@text@width=\pgfutil@empty
  \let\tikz@text@height=\pgfutil@empty
  \let\tikz@text@depth=\pgfutil@empty
  \let\tikz@textcolor=\pgfutil@empty
  \let\tikz@textfont=\pgfutil@empty
  \let\tikz@textopacity=\pgfutil@empty
  \let\tikz@node@textfont=\pgfutil@empty
  \def\tikz@text@action{\pgfutil@raggedright\rightskip0pt plus2em \spaceskip.3333em \xspaceskip.5em\relax}%
}
\tikz@text@reset


% Alignment
\tikzset{
  node halign header/.initial=,
  align/.is choice,
  align/left/.style  ={text ragged,node halign header=\tikz@align@left@header},
  align/flush left/.style  ={text badly ragged,node halign header=\tikz@align@left@header},
  align/right/.style ={text ragged left,node halign header=\tikz@align@right@header},
  align/flush right/.style ={text badly ragged left,node halign header=\tikz@align@right@header},
  align/center/.style={text centered,node halign header=\tikz@align@center@header},
  align/flush center/.style={text badly centered,node halign header=\tikz@align@center@header},
  align/justify/.style ={text justified,node halign header=\tikz@align@left@header},
  align/none/.style ={text justified,node halign header=},
}%
\def\tikz@align@left@header{##\hfil\cr}%
\def\tikz@align@right@header{\hfil##\cr}%
\def\tikz@align@center@header{\hfil##\hfil\cr}%



% Arrow options
\tikzoption{arrows}{\tikz@processarrows{#1}}%


\tikzoption{>}{\pgfdeclarearrow{name=<->,means={#1}}}%
\pgfdeclarearrow{name=|<->|,   means={>[sep=0pt].|}}%

\tikzoption{shorten <}{\pgfsetshortenstart{#1}}%
\tikzoption{shorten >}{\pgfsetshortenend{#1}}%

\def\tikz@processarrows#1{%
  \def\tikz@current@arrows{#1}%
  \def\tikz@temp{#1}%
  \ifx\tikz@temp\pgfutil@empty%
  \else%
    \pgfsetarrows{#1}%
  \fi%
}%

\def\tikz@current@arrows{-}%

% Parabola options
\tikzoption{bend}{\tikz@scan@one@point\tikz@set@parabola@bend#1\relax}%
\tikzoption{bend pos}{\def\tikz@parabola@bend@factor{#1}}%
\tikzoption{parabola height}{%
  \def\tikz@parabola@bend@factor{.5}%
  \def\tikz@parabola@bend{\pgfpointadd{\pgfpoint{0pt}{#1}}{\tikz@last@position@saved}}}%

\def\tikz@parabola@bend{\tikz@last@position@saved}%
\def\tikz@parabola@bend@factor{0}%

\def\tikz@set@parabola@bend#1{\def\tikz@parabola@bend{#1}}%

% Axis options
\tikzoption{domain}{\edef\tikz@plot@domain{#1}\expandafter\tikz@plot@samples@recalc\tikz@plot@domain\relax}%
\tikzoption{range}{\def\tikz@plot@range{#1}}%
\tikzoption{yrange}{\def\tikz@plot@range{#1}}%
\let\tikz@plot@range=\pgfutil@empty
\tikzoption{xrange}{\def\tikz@plot@xrange{#1}}%
\let\tikz@plot@xrange=\pgfutil@empty

% Plot options
\tikzoption{smooth}[]{\let\tikz@plot@handler=\pgfplothandlercurveto}%
\tikzoption{smooth cycle}[]{\let\tikz@plot@handler=\pgfplothandlerclosedcurve}%
\tikzoption{sharp plot}[]{\let\tikz@plot@handler\pgfplothandlerlineto}%
\tikzoption{sharp cycle}[]{\let\tikz@plot@handler\pgfplothandlerpolygon}%

\tikzoption{tension}{\pgfsetplottension{#1}}%

\tikzoption{xcomb}[]{\let\tikz@plot@handler=\pgfplothandlerxcomb}%
\tikzoption{ycomb}[]{\let\tikz@plot@handler=\pgfplothandlerycomb}%
\tikzoption{polar comb}[]{\let\tikz@plot@handler=\pgfplothandlerpolarcomb}%
\tikzoption{ybar}[]{\let\tikz@plot@handler=\pgfplothandlerybar}%
\tikzoption{ybar interval}[]{\let\tikz@plot@handler=\pgfplothandlerybarinterval}%
\tikzoption{xbar interval}[]{\let\tikz@plot@handler=\pgfplothandlerxbarinterval}%
\tikzoption{xbar}[]{\let\tikz@plot@handler=\pgfplothandlerxbar}%
\tikzoption{const plot}[]{\let\tikz@plot@handler=\pgfplothandlerconstantlineto}%
\tikzoption{const plot mark left}[]{\let\tikz@plot@handler=\pgfplothandlerconstantlineto}%
\tikzoption{const plot mark right}[]{\let\tikz@plot@handler=\pgfplothandlerconstantlinetomarkright}%
\tikzoption{const plot mark mid}[]{\let\tikz@plot@handler=\pgfplothandlerconstantlinetomarkmid}%
\tikzoption{jump mark right}[]{\let\tikz@plot@handler=\pgfplothandlerjumpmarkright}%
\tikzoption{jump mark mid}[]{\let\tikz@plot@handler=\pgfplothandlerjumpmarkmid}%
\tikzoption{jump mark left}[]{\let\tikz@plot@handler=\pgfplothandlerjumpmarkleft}%

\tikzoption{raw gnuplot}[true]{\csname tikz@plot@raw@gnuplot#1\endcsname}%
\tikzoption{prefix}{\def\tikz@plot@prefix{#1}}%
\tikzoption{id}{\def\tikz@plot@id{#1}}%

\tikzoption{samples}{\pgfmathsetmacro\tikz@plot@samples{max(2,#1)}\expandafter\tikz@plot@samples@recalc\tikz@plot@domain\relax}%
\tikzoption{samples at}{\def\tikz@plot@samplesat{#1}}%
\tikzoption{parametric}[true]{\csname tikz@plot@parametric#1\endcsname}%

\tikzoption{variable}{\def\tikz@plot@var{#1}}%

\tikzoption{only marks}[]{\let\tikz@plot@handler\pgfplothandlerdiscard}%

\tikzoption{mark}{%
    \def\tikz@plot@mark{#1}%
    \def\tikz@temp{none}%
    \ifx\tikz@temp\tikz@plot@mark
        \let\tikz@plot@mark=\pgfutil@empty
    \fi
}%
\tikzset{
    no marks/.style={mark=none},%
    no markers/.style={mark=none},%
    every mark/.style={},
    mark options/.style={%
        every mark/.style={#1}%
    }}%
\tikzoption{mark size}{\pgfsetplotmarksize{#1}}%

\tikzoption{mark indices}{\def\tikz@mark@list{#1}}%
\tikzoption{mark phase}{\pgfsetplotmarkphase{#1}}%
\tikzoption{mark repeat}{\pgfsetplotmarkrepeat{#1}}%

\let\tikz@mark@list=\pgfutil@empty

\let\tikz@plot@handler=\pgfplothandlerlineto
\let\tikz@plot@mark=\pgfutil@empty

\def\tikz@plot@samples{25}%
\def\tikz@plot@domain{-5:5}%
\def\tikz@plot@var{\x}%
\def\tikz@plot@samplesat{-5,-4.5833333,...,5}%
\def\tikz@plot@samples@recalc#1:#2\relax{%
  \begingroup
  \pgfmathparse{#1}%
  \let\tikz@temp@start=\pgfmathresult%
  \pgfmathparse{#2}%
  \let\tikz@temp@end=\pgfmathresult%
  \pgfmathsetmacro\tikz@temp@diff{(\tikz@temp@end-\tikz@temp@start)/(\tikz@plot@samples-1)}%
  %
  % this particular item is for backwards compatibility.
  % Pgfplots <= 1.8 called 'samples' in a context where the 'fpu' was
  % active... and I fear there is no simple solution to replace the
  % new \ifdim below. Sorry.
  \pgfkeys{/pgf/fpu/output format/fixed/.try}%
  %
  \pgfmathsetmacro\tikz@temp@diff@abs{abs(\tikz@temp@diff)}%
  \ifdim\tikz@temp@diff@abs pt<0.0001pt\relax%
    \edef\tikz@plot@samplesat{\tikz@temp@start,\tikz@temp@end}%
  \else%
    \pgfmathparse{\tikz@temp@start+\tikz@temp@diff}%
    \edef\tikz@plot@samplesat{\tikz@temp@start,\pgfmathresult,...,\tikz@temp@end}%
  \fi%
  \pgfmath@smuggleone\tikz@plot@samplesat
  \endgroup
}%


\def\tikz@plot@prefix{\jobname.}%
\def\tikz@plot@id{pgf-plot}%

\newif\iftikz@plot@parametric
\newif\iftikz@plot@raw@gnuplot


%
% To and edge options
%
\tikzoption{to path}{\def\tikz@to@path{#1}}%

\def\tikz@to@path{-- (\tikztotarget) \tikztonodes}%

\tikzset{edge macro/.store in=\tikz@edge@macro}%
\let\tikz@edge@macro\pgfutil@empty

\tikzset{
  edge node/.code={
    \expandafter\def\expandafter\tikz@tonodes\expandafter{\tikz@tonodes #1}
  },
  edge label/.style={/tikz/edge node={node[auto]{#1}}},
  edge label'/.style={/tikz/edge node={node[auto,swap]{#1}}},
}%


% After command options
\tikzset{
  append after command/.code=\expandafter\def\expandafter\tikz@after@path\expandafter{\tikz@after@path#1},
  prefix after command/.code={%
    \def\tikz@temp{#1}%
    \expandafter\expandafter\expandafter\def%
    \expandafter\expandafter\expandafter\tikz@after@path%
    \expandafter\expandafter\expandafter{%
      \expandafter\tikz@temp\tikz@after@path}%
  },
}%
\let\tikz@after@path\pgfutil@empty


% Tree options
\newif\iftikz@child@missing
\pgfkeys{/tikz/missing/.is if=tikz@child@missing}%

\tikzset{edge from parent macro/.initial=\tikz@edge@from@parent@macro}%
\def\tikz@edge@from@parent@macro#1#2{
  [style=edge from parent, #1, /utils/exec=\tikz@node@is@a@labeltrue] \tikz@edge@to@parent@path #2}%

\tikzoption{edge from parent path}{\def\tikz@edge@to@parent@path{#1}}%

\tikzoption{parent anchor}{\def\tikzparentanchor{.#1}\ifx\tikzparentanchor\tikz@border@text\let\tikzparentanchor\pgfutil@empty\fi}%
\tikzoption{child anchor}{\def\tikzchildanchor{.#1}\ifx\tikzchildanchor\tikz@border@text\let\tikzchildanchor\pgfutil@empty\fi}%

\tikzoption{level distance}{\pgfmathsetlength\tikzleveldistance{#1}}%
\tikzoption{sibling distance}{\pgfmathsetlength\tikzsiblingdistance{#1}}%

\tikzoption{growth function}{\let\tikz@grow=#1}%

\tikzset{grow siblings on line/.style={growth function=\tikz@grow@direction}}%

\tikzoption{growth parent anchor}{\def\tikz@growth@anchor{#1}}%
\tikzoption{grow}{\tikz@set@growth{#1}\edef\tikz@special@level{\the\tikztreelevel}}%
\tikzoption{grow'}{\tikz@set@growth{#1}\tikz@swap@growth\edef\tikz@special@level{\the\tikztreelevel}}%

\def\tikz@growth@anchor{center}%

\def\tikz@special@level{-1}% never

\def\tikz@swap@growth{%
  % Swap left and right
  \let\tikz@temp=\tikz@angle@grow@right%
  \let\tikz@angle@grow@right=\tikz@angle@grow@left%
  \let\tikz@angle@grow@left=\tikz@temp%
}%

\def\tikz@set@growth#1{%
  \let\tikz@grow=\tikz@grow@direction%
  \expandafter\ifx\csname tikz@grow@direction@#1\endcsname\relax%
    \c@pgf@counta=#1\relax%
  \else%
    \c@pgf@counta=\csname tikz@grow@direction@#1\endcsname%
  \fi%
  \edef\tikz@angle@grow{\the\c@pgf@counta}%
  \advance\c@pgf@counta by-90\relax%
  \edef\tikz@angle@grow@left{\the\c@pgf@counta}%
  \advance\c@pgf@counta by180\relax%
  \edef\tikz@angle@grow@right{\the\c@pgf@counta}%
}%

\def\tikz@border@text{.border}%
\let\tikzparentanchor=\pgfutil@empty
\let\tikzchildanchor=\pgfutil@empty
\def\tikz@edge@to@parent@path{(\tikzparentnode\tikzparentanchor) -- (\tikzchildnode\tikzchildanchor)}%

\tikzleveldistance=15mm%
\tikzsiblingdistance=15mm%

\def\tikz@grow@direction@down{-90}%
\def\tikz@grow@direction@up{90}%
\def\tikz@grow@direction@left{180}%
\def\tikz@grow@direction@right{0}%

\def\tikz@grow@direction@south{-90}%
\def\tikz@grow@direction@north{90}%
\def\tikz@grow@direction@west{180}%
\def\tikz@grow@direction@east{0}%

\expandafter\def\csname tikz@grow@direction@north east\endcsname{45}%
\expandafter\def\csname tikz@grow@direction@north west\endcsname{135}%
\expandafter\def\csname tikz@grow@direction@south east\endcsname{-45}%
\expandafter\def\csname tikz@grow@direction@south west\endcsname{-135}%

\def\tikz@grow@direction{%
  \pgftransformshift{\pgfpointpolar{\tikz@angle@grow}{\tikzleveldistance}}%
  \ifnum\tikztreelevel=\tikz@special@level%
  \else%
    \pgf@xc=.5\tikzsiblingdistance%
    \c@pgf@counta=\tikznumberofchildren%
    \advance\c@pgf@counta by1\relax%
    \pgfutil@tempdima=\c@pgf@counta\pgf@xc%
    \pgftransformshift{\pgfpointpolar{\tikz@angle@grow@left}{\pgfutil@tempdima}}%
    \pgftransformshift{\pgfpointpolar{\tikz@angle@grow@right}{\tikznumberofcurrentchild\tikzsiblingdistance}}%
  \fi%
}%

\tikzset{grow=down}%


% Snakes are in a lib:
\tikzset{snake/.code=\tikzerror{You need to say \string\usetikzlibrary{snakes}}}%

% Decorations
\tikzset{decorate/.code=\tikzerror{You need to load a  decoration library}}%

% Matrix options
\usepgfmodule{matrix}%

\tikzoption{matrix}[true]{\csname tikz@is@matrix#1\endcsname}%

\tikzoption{matrix anchor}{\def\tikz@matrix@anchor{#1}}%

\tikzoption{column sep}{\def\pgfmatrixcolumnsep{#1}}%
\tikzoption{row sep}{\def\pgfmatrixrowsep{#1}}%

\tikzoption{cells}{\tikzset{every cell/.append style={#1}}}%

\tikzoption{ampersand replacement}{\def\tikz@ampersand@replacement{#1}}%

\newif\iftikz@is@matrix
\let\tikz@matrix@anchor=\pgfutil@empty
\let\tikz@ampersand@replacement=\pgfutil@empty


% Automatic shorthand management
\tikzset{%
  handle active characters in code/.is if=tikz@handle@active@code,
  handle active characters in nodes/.is if=tikz@handle@active@nodes,
}%
\newif\iftikz@handle@active@code
\newif\iftikz@handle@active@nodes


% Execute option
\tikzoption{execute at begin picture}{\expandafter\def\expandafter\tikz@atbegin@picture\expandafter{\tikz@atbegin@picture#1}}%
\tikzoption{execute at end picture}{\expandafter\def\expandafter\tikz@atend@picture\expandafter{\tikz@atend@picture#1}}%
\tikzoption{execute at begin scope}{\expandafter\def\expandafter\tikz@atbegin@scope\expandafter{\tikz@atbegin@scope#1}}%
\tikzoption{execute at end scope}{\expandafter\def\expandafter\tikz@atend@scope\expandafter{\tikz@atend@scope#1}}%
\tikzoption{execute at begin to}{\expandafter\def\expandafter\tikz@atbegin@to\expandafter{\tikz@atbegin@to#1}}%
\tikzoption{execute at end to}{\expandafter\def\expandafter\tikz@atend@to\expandafter{\tikz@atend@to#1}}%
\tikzoption{execute at begin node}{\expandafter\def\expandafter\tikz@atbegin@node\expandafter{\tikz@atbegin@node#1}}%
\tikzoption{execute at end node}{\expandafter\def\expandafter\tikz@atend@node\expandafter{\tikz@atend@node#1}}%
\tikzoption{execute at begin matrix}{\expandafter\def\expandafter\tikz@atbegin@matrix\expandafter{\tikz@atbegin@matrix#1}}%
\tikzoption{execute at end matrix}{\expandafter\def\expandafter\tikz@atend@matrix\expandafter{\tikz@atend@matrix#1}}%
\tikzoption{execute at begin cell}{\expandafter\def\expandafter\tikz@atbegin@cell\expandafter{\tikz@atbegin@cell#1}}%
\tikzoption{execute at end cell}{\expandafter\def\expandafter\tikz@atend@cell\expandafter{\tikz@atend@cell#1}}%
\tikzoption{execute at empty cell}{\expandafter\def\expandafter\tikz@at@emptycell\expandafter{\tikz@at@emptycell#1}}%

\let\tikz@atbegin@picture=\pgfutil@empty
\let\tikz@atend@picture=\pgfutil@empty
\let\tikz@atbegin@scope=\pgfutil@empty
\let\tikz@atend@scope=\pgfutil@empty
\let\tikz@atbegin@to=\pgfutil@empty
\let\tikz@atend@to=\pgfutil@empty
\let\tikz@atbegin@node=\pgfutil@empty
\let\tikz@atend@node=\pgfutil@empty
\let\tikz@atbegin@cell=\pgfutil@empty
\let\tikz@atend@cell=\pgfutil@empty
\let\tikz@at@emptycell=\pgfutil@empty
\let\tikz@atbegin@matrix=\pgfutil@empty
\let\tikz@atend@matrix=\pgfutil@empty


% Pre and post actions
\tikzset{preaction/.code=\expandafter\def\expandafter\tikz@preactions\expandafter{\tikz@preactions\tikz@extra@preaction{#1}}}%
\tikzset{postaction/.code=\expandafter\def\expandafter\tikz@postactions\expandafter{\tikz@postactions\tikz@extra@postaction{#1}}}%
\let\tikz@preactions=\pgfutil@empty
\let\tikz@postactions=\pgfutil@empty

% Styles
\tikzoption{set style}{\tikzstyle#1}%

% Handled in a special way.
\def\tikzstyle{\pgfutil@ifnextchar\bgroup\tikz@style@parseA\tikz@style@parseB}%
\def\tikz@style@parseB#1={\tikz@style@parseA{#1}=}%
\def\tikz@style@parseA#1#2=#3[#4]{% check for an optional argument
  \pgfutil@in@[{#2}%]
  \ifpgfutil@in@%
    \tikz@style@parseC{#1}#2={#4}%
  \else%
    \tikz@style@parseD{#1}#2={#4}%
  \fi%
}%

\def\tikz@style@parseC#1[#2]#3=#4{%
  \pgfkeys{/tikz/#1/.default={#2}}%
  \pgfutil@in@+{#3}%
  \ifpgfutil@in@%
    \pgfkeys{/tikz/#1/.append style={#4}}%
  \else%
    \pgfkeys{/tikz/#1/.style={#4}}%
  \fi}%
\def\tikz@style@parseD#1#2=#3{%
  \pgfutil@in@+{#2}%
  \ifpgfutil@in@%
    \pgfkeys{/tikz/#1/.append style={#3}}%
  \else%
    \pgfkeys{/tikz/#1/.style={#3}}%
  \fi}%


%
%
% Predefined styles
%
%

\tikzset{help lines/.style=              {color=gray,line width=0.2pt}}%

\tikzset{every picture/.style=           {}}%
\tikzset{every path/.style=              {}}%
\tikzset{every scope/.style=             {}}%
\tikzset{every plot/.style=              {}}%
\tikzset{every node/.style=              {}}%
\tikzset{every child/.style=             {}}%
\tikzset{every child node/.style=        {}}%
\tikzset{every to/.style=                {}}%
\tikzset{every cell/.style=              {}}%
\tikzset{every matrix/.style=            {}}%
\tikzset{every edge/.style=              {draw}}%
\tikzset{every label/.style=             {draw=none,fill=none}}%
\tikzset{every pin/.style=               {draw=none,fill=none}}%
\tikzset{every pin edge/.style=          {help lines}}%

\tikzset{ultra thin/.style=              {line width=0.1pt}}%
\tikzset{very thin/.style=               {line width=0.2pt}}%
\tikzset{thin/.style=                    {line width=0.4pt}}%
\tikzset{semithick/.style=               {line width=0.6pt}}%
\tikzset{thick/.style=                   {line width=0.8pt}}%
\tikzset{very thick/.style=              {line width=1.2pt}}%
\tikzset{ultra thick/.style=             {line width=1.6pt}}%

\tikzset{solid/.style=                   {dash pattern=}}%
\tikzset{dotted/.style=                  {dash pattern=on \pgflinewidth off 2pt}}%
\tikzset{densely dotted/.style=          {dash pattern=on \pgflinewidth off 1pt}}%
\tikzset{loosely dotted/.style=          {dash pattern=on \pgflinewidth off 4pt}}%
\tikzset{dashed/.style=                  {dash pattern=on 3pt off 3pt}}%
\tikzset{densely dashed/.style=          {dash pattern=on 3pt off 2pt}}%
\tikzset{loosely dashed/.style=          {dash pattern=on 3pt off 6pt}}%
\tikzset{dashdotted/.style=              {dash pattern=on 3pt off 2pt on \the\pgflinewidth off 2pt}}%
\tikzset{dash dot/.style=                {dash pattern=on 3pt off 2pt on \the\pgflinewidth off 2pt}}%
\tikzset{densely dashdotted/.style=      {dash pattern=on 3pt off 1pt on \the\pgflinewidth off 1pt}}%
\tikzset{densely dash dot/.style=        {dash pattern=on 3pt off 1pt on \the\pgflinewidth off 1pt}}%
\tikzset{loosely dashdotted/.style=      {dash pattern=on 3pt off 4pt on \the\pgflinewidth off 4pt}}%
\tikzset{loosely dash dot/.style=        {dash pattern=on 3pt off 4pt on \the\pgflinewidth off 4pt}}%
\tikzset{dashdotdotted/.style=           {dash pattern=on 3pt off 2pt on \the\pgflinewidth off 2pt on \the\pgflinewidth off 2pt}}%
\tikzset{densely dashdotdotted/.style=   {dash pattern=on 3pt off 1pt on \the\pgflinewidth off 1pt on \the\pgflinewidth off 1pt}}%
\tikzset{loosely dashdotdotted/.style=   {dash pattern=on 3pt off 4pt on \the\pgflinewidth off 4pt on \the\pgflinewidth off 4pt}}%
\tikzset{dash dot dot/.style=         {dash pattern=on 3pt off 2pt on \the\pgflinewidth off 2pt on \the\pgflinewidth off 2pt}}%
\tikzset{densely dash dot dot/.style=   {dash pattern=on 3pt off 1pt on \the\pgflinewidth off 1pt on \the\pgflinewidth off 1pt}}%
\tikzset{loosely dash dot dot/.style=   {dash pattern=on 3pt off 4pt on \the\pgflinewidth off 4pt on \the\pgflinewidth off 4pt}}%


\tikzset{transparent/.style=             {opacity=0}}%
\tikzset{ultra nearly transparent/.style={opacity=0.05}}%
\tikzset{very nearly transparent/.style= {opacity=0.1}}%
\tikzset{nearly transparent/.style=      {opacity=0.25}}%
\tikzset{semitransparent/.style=         {opacity=0.5}}%
\tikzset{nearly opaque/.style=           {opacity=0.75}}%
\tikzset{very nearly opaque/.style=      {opacity=0.9}}%
\tikzset{ultra nearly opaque/.style=     {opacity=0.95}}%
\tikzset{opaque/.style=                  {opacity=1}}%

\tikzset{at start/.style=                {pos=0}}%
\tikzset{very near start/.style=         {pos=0.125}}%
\tikzset{near start/.style=              {pos=0.25}}%
\tikzset{midway/.style=                  {pos=0.5}}%
\tikzset{near end/.style=                {pos=0.75}}%
\tikzset{very near end/.style=           {pos=0.875}}%
\tikzset{at end/.style=                  {pos=1}}%

\tikzset{bend at start/.style=           {bend pos=0,bend={+(0,0)}}}%
\tikzset{bend at end/.style=             {bend pos=1,bend={+(0,0)}}}%

\tikzset{edge from parent/.style=        {draw}}%



% Animation callbacks
\tikzset{
  animate/.code=\tikzerror{You need to say \string\usetikzlibrary{animations} to use animations}
}

% ID callbacks
\newif\iftikz@is@node
\let\tikz@id@name\pgfutil@empty
\let\tikz@id@hook\pgfutil@empty
\def\tikz@call@id@hook{\ifx\tikz@id@hook\pgfutil@empty\else\tikz@id@hook\pgfuseid{\tikz@id@name}\fi}%


% RDF stuff
\let\tikz@clear@rdf@options\relax
\let\tikz@do@rdf@post@options\relax
\let\tikz@do@rdf@pre@options\relax


%
% Setting keys
%

\pgfkeys{/tikz/style/.style={#1}}%

\pgfkeys{/tikz/.unknown/.code=%
  % Is it a pgf key?
  \let\tikz@key\pgfkeyscurrentname%
  \pgfkeys{/pgf/\tikz@key/.try={#1}}%
  \ifpgfkeyssuccess%
  \else%
    \expandafter\pgfutil@in@\expandafter!\expandafter{\tikz@key}%
    \ifpgfutil@in@%
      % this is a color!
      \expandafter\tikz@addoption\expandafter{\expandafter\tikz@compat@color@set\expandafter{\tikz@key}}%
      \edef\tikz@textcolor{\tikz@key}%
    \else%
      \pgfutil@doifcolorelse{\tikz@key}
      {%
        \expandafter\tikz@addoption\expandafter{\expandafter\tikz@compat@color@set\expandafter{\tikz@key}}%
        \edef\tikz@textcolor{\tikz@key}%
      }%
      {%
        % Ok, second chance: This might be an arrow specification:
        \expandafter\pgfutil@in@\expandafter-\expandafter{\tikz@key}%
        \ifpgfutil@in@%
          % Ah, an arrow spec!
          \expandafter\tikz@processarrows\expandafter{\tikz@key}%
        \else%
          % Ok, third chance: A shape!
          \expandafter\ifx\csname pgf@sh@s@\tikz@key\endcsname\relax%
            \pgfkeys{/errors/unknown key/.expand
              once=\expandafter{\expandafter/\expandafter t\expandafter i\expandafter k\expandafter z\expandafter/\tikz@key}{#1}}%
          \else%
            \edef\tikz@shape{\tikz@key}%
          \fi%
        \fi%
      }%
    \fi%
  \fi%
}%
\def\tikz@compat@color@set#1{%
  \pgfutil@color{#1}\pgfutil@colorlet{pgffillcolor}{#1}%
  \expandafter\let\expandafter\pgf@temp\csname\string\color@pgffillcolor\endcsname%
  % for arrow tips:
  \global\let\pgf@strokecolor@global=\pgf@temp
  \global\let\pgf@fillcolor@global=\pgf@temp
}%

\def\tikz@startup@env{%
  \ifnum\the\catcode`\;=\active\relax\expandafter\let\expandafter\tikz@origsemi\expandafter=\tikz@activesemicolon\fi%
  \ifnum\the\catcode`\:=\active\relax\expandafter\let\expandafter\tikz@origcolon\expandafter=\tikz@activecolon\fi%
  \ifnum\the\catcode`\|=\active\relax\expandafter\let\expandafter\tikz@origbar\expandafter=\tikz@activebar\fi%
  \tikz@deactivatthings%
  \iftikz@handle@active@code%
    \tikz@switchoff@shorthands%
  \fi%
}%

%
% Main TikZ Environment
%
\newif\iftikz@inside@picture
\tikz@inside@picturefalse
\def\tikz@check@inside@picture{%
  \iftikz@inside@picture%
    \pgfwarning{Nesting tikzpictures is NOT supported}%
  \fi%
  \tikz@inside@picturetrue%
}

\def\tikzpicture{%
  \begingroup%
    \tikz@startup@env%
    \tikz@collect@scope@anims\tikz@picture}%
\def\tikz@picture[#1]{%
  %\tikz@check@inside@picture%
  \pgfpicture%
  \let\tikz@atbegin@picture=\pgfutil@empty%
  \let\tikz@atend@picture=\pgfutil@empty%
  \let\tikz@transform=\relax%
  \def\tikz@time{.5}%
  \tikz@installcommands%
  \scope[every picture,#1]%
  \iftikz@handle@active@code%
    \tikz@switchoff@shorthands%
  \fi%
  \expandafter\tikz@atbegin@picture%
  \tikz@lib@scope@check%
}%
\def\endtikzpicture{%
    \tikz@atend@picture%
    \global\let\pgf@shift@baseline@smuggle=\pgf@baseline%
    \global\let\pgf@trimleft@final@smuggle=\pgf@trimleft%
    \global\let\pgf@trimright@final@smuggle=\pgf@trimright%
    \global\let\pgf@remember@smuggle=\ifpgfrememberpicturepositiononpage%
    \pgf@remember@layerlist@globally
    \endscope%
    \let\pgf@baseline=\pgf@shift@baseline@smuggle%
    \let\pgf@trimleft=\pgf@trimleft@final@smuggle%
    \let\pgf@trimright=\pgf@trimright@final@smuggle%
    \let\ifpgfrememberpicturepositiononpage=\pgf@remember@smuggle%
    \pgf@restore@layerlist@from@global
  \endpgfpicture\endgroup}%


% Inlined picture
%
% #1 - some code to be put in a tikzpicture environment.
%
% If the command is not followed by braces, everything up to the next
% semicolon is used as argument.
%
% Example:
%
% The rectangle \tikz{\draw (0,0) rectangle (1em,1ex)} has width 1em and
% height 1ex.


\def\tikz{%
  \begingroup%
    \tikz@startup@env%
    \tikz@collect@scope@anims\tikz@opt}%
\def\tikz@opt[#1]{\tikzpicture[#1]\pgfutil@ifnextchar\bgroup{\tikz@}{\tikz@@single}}%
\def\tikz@{\bgroup\tikz@auto@end@pathtrue\aftergroup\endtikzpicture\aftergroup\endgroup\let\pgf@temp=}%
\def\tikz@@single#1{%
  \expandafter\ifx\csname tikz@protected@command\string#1\endcsname\relax%
    \expandafter\tikz@@%
  \else%
    \begingroup\def\tikz@path@do@at@end{\endgroup\endtikzpicture\endgroup}%
  \fi%
  #1%
}%

\expandafter\let\csname tikz@protected@command\string\draw\endcsname\pgfutil@empty%
\expandafter\let\csname tikz@protected@command\string\pattern\endcsname\pgfutil@empty%
\expandafter\let\csname tikz@protected@command\string\fill\endcsname\pgfutil@empty%
\expandafter\let\csname tikz@protected@command\string\filldraw\endcsname\pgfutil@empty%
\expandafter\let\csname tikz@protected@command\string\shade\endcsname\pgfutil@empty%
\expandafter\let\csname tikz@protected@command\string\shadedraw\endcsname\pgfutil@empty%
\expandafter\let\csname tikz@protected@command\string\clip\endcsname\pgfutil@empty%
\expandafter\let\csname tikz@protected@command\string\graph\endcsname\pgfutil@empty%
\expandafter\let\csname tikz@protected@command\string\useasboundingbox\endcsname\pgfutil@empty%
\expandafter\let\csname tikz@protected@command\string\node\endcsname\pgfutil@empty%
\expandafter\let\csname tikz@protected@command\string\coordinate\endcsname\pgfutil@empty%
\expandafter\let\csname tikz@protected@command\string\matrix\endcsname\pgfutil@empty%
\expandafter\let\csname tikz@protected@command\string\datavisualization\endcsname\pgfutil@empty%
\expandafter\let\csname tikz@protected@command\string\path\endcsname\pgfutil@empty%
\expandafter\let\csname tikz@protected@command\string\pic\endcsname\pgfutil@empty%

% Comment by TT: I hope I fixed the \tikz \foreach problem. The new
% version will take a conservative approach and will only do fancy
% stuff when the next keyword after \tikz is one of the following:
% \draw, \fill, \filldraw, \graph, \matrix,
\def\tikz@@{%
  \let\tikz@next=\tikz@collectnormalsemicolon%
  \ifnum\the\catcode`\;=\active\relax%
    \let\tikz@next=\tikz@collectactivesemicolon%
  \fi%
  \tikz@next}%
\def\tikz@collectnormalsemicolon#1;{#1;\endtikzpicture\endgroup}
{
  \catcode`\;=\active
  \gdef\tikz@collectactivesemicolon#1;{#1;\endtikzpicture\endgroup}
}%
% End old code

% Invokes '#1' if the command is invoked within a tikzpicture and
% '#2' if not.
\def\tikzifinpicture#1#2{%
    \pgfutil@ifundefined{filldraw}{#2}{#1}% TT: This is a wrong
                                % test! Who uses this?...
}%


\def\tikz@collect@scope@anims#1{%
  \pgfutil@ifnextchar[#1{#1[]}%]
}%

%
% Environment for scoping graphic state settings
%
\def\tikz@scope@env{%
  \pgfscope%
  \begingroup%
  \let\tikz@atbegin@scope=\pgfutil@empty%
  \let\tikz@atend@scope=\pgfutil@empty%
  \let\tikz@options=\pgfutil@empty%
  \tikz@clear@rdf@options%
  \let\tikz@mode=\pgfutil@empty%
  \let\tikz@id@name=\pgfutil@empty%
  \tikz@transparency@groupfalse%
  \tikzset{every scope/.try}%
  \tikz@collect@scope@anims\tikz@scope@opt%
}%
\def\tikz@scope@opt[#1]{%
  \tikzset{#1}%
  \tikz@options%
  \tikz@do@rdf@pre@options%
  \iftikz@transparency@group\expandafter\pgftransparencygroup\expandafter[\tikz@transparency@group@options]\tikz@blend@group\fi%
  \tikz@is@nodefalse%
  \tikz@call@id@hook%
  \pgfidscope%
    \tikz@do@rdf@post@options%
    \begingroup%
      \let\tikz@id@name\pgfutil@empty%
      \expandafter\tikz@atbegin@scope%
      \expandafter\pgfclearid%
      \tikz@lib@scope@check%
}%
\def\endtikz@scope@env{%
      \tikz@atend@scope%
    \endgroup%
  \endpgfidscope%
  \iftikz@transparency@group\endpgftransparencygroup\fi%
  \endgroup%
  \endpgfscope%
  \tikz@lib@scope@check%
}%


\def\tikz@scoped{\tikz@collect@scope@anims\tikz@scoped@opt}%
\def\tikz@scoped@opt[#1]{\scope[#1]\pgfutil@ifnextchar\bgroup{\tikz@scoped@}{\tikz@scoped@@single}}%
\def\tikz@scoped@{\bgroup\tikz@auto@end@pathtrue\aftergroup\endscope\let\pgf@temp=}%
\def\tikz@scoped@@single#1{%
  \expandafter\ifx\csname tikz@scoped@protected@command\string#1\endcsname\relax%
    \expandafter\tikz@scoped@@%
  \else%
    \begingroup\def\tikz@scoped@path@do@at@end{\endgroup\endscope}%
  \fi%
  #1%
}%
\def\tikz@scoped@@{%
  \let\tikz@scoped@next=\tikz@scoped@collectnormalsemicolon%
  \ifnum\the\catcode`\;=\active\relax%
    \let\tikz@scoped@next=\tikz@scoped@collectactivesemicolon%
  \fi%
  \tikz@scoped@next}%
\def\tikz@scoped@collectnormalsemicolon#1;{#1;\endscope}
{
  \catcode`\;=\active
  \gdef\tikz@scoped@collectactivesemicolon#1;{#1;\endscope}
}%


% Install a shortcut command which is only valid inside of a
% tikzpicture.
%
% It works in the same way as the '\path' shortcut does: it introduces
% a \let#1=#2 at the beginning of tikzpicture.
%
% #1: shortcut command inside of tikzpicture
% #2: real command name
\def\tikzaddtikzonlycommandshortcutlet#1#2{%
    \expandafter\def\expandafter\tikz@installcommands\expandafter{\tikz@installcommands
        \let#1=#2%
    }%
}%

% Has the same effect as \tikzaddtikzonlycommandshortcutlet but uses
% \def#1{#2} instead of \let.
\def\tikzaddtikzonlycommandshortcutdef#1#2{%
    \expandafter\def\expandafter\tikz@installcommands\expandafter{\tikz@installcommands
        \def#1{#2}%
    }%
}%

%
% Install the abbreviated commands
%
\def\tikz@installcommands{%
  \let\tikz@origscope=\scope%
  \let\tikz@origscoped=\scoped%
  \let\tikz@origendscope=\endscope%
  \let\tikz@origstartscope=\startscope%
  \let\tikz@origstopscope=\stopscope%
  \let\tikz@origpath=\path%
  \let\tikz@origagainpath=\againpath%
  \let\tikz@origdraw=\draw%
  \let\tikz@origpattern=\pattern%
  \let\tikz@origfill=\fill%
  \let\tikz@origfilldraw=\filldraw%
  \let\tikz@origshade=\shade%
  \let\tikz@origshadedraw=\shadedraw%
  \let\tikz@origclip=\clip%
  \let\tikz@origuseasboundingbox=\useasboundingbox%
  \let\tikz@orignode=\node%
  \let\tikz@origpic=\pic%
  \let\tikz@origcoordinate=\coordinate%
  \let\tikz@origmatrix=\matrix%
  \let\tikz@origcalendar=\calendar%
  \let\tikz@origdv=\datavisualization%
  \let\tikz@origgraph=\graph%
  %
  \let\scope=\tikz@scope@env%
  \let\scoped=\tikz@scoped%
  \let\endscope=\endtikz@scope@env%
  \let\startscope=\scope%
  \let\stopscope=\endscope%
  \let\path=\tikz@command@path%
  \let\againpath=\tikz@command@againpath%
  %
  \def\draw{\path[draw]}%
  \def\pattern{\path[pattern]}%
  \def\fill{\path[fill]}%
  \def\filldraw{\path[fill,draw]}%
  \def\shade{\path[shade]}%
  \def\shadedraw{\path[shade,draw]}%
  \def\clip{\path[clip]}%
  \def\graph{\path graph}%
  \def\useasboundingbox{\path[use as bounding box]}%
  \def\node{\tikz@path@overlay{node}}%
  \def\pic{\tikz@path@overlay{pic}}%
  \def\coordinate{\tikz@path@overlay{coordinate}}%
  \def\matrix{\tikz@path@overlay{node[matrix]}}%
  \def\calendar{\tikz@lib@cal@calendar}%
  \def\datavisualization{\tikz@lib@datavisualization}%
}%
\ifx\tikz@lib@cal@calendar\@undefined
\def\tikz@lib@cal@calendar{\tikzerror{You need to say \string\usetikzlibrary{calendar} to use the \string\calendar{} command}}%
\fi
\ifx\tikz@lib@datavisualization\@undefined
\def\tikz@lib@datavisualization{\tikzerror{You need to say \string\usetikzlibrary{datavisualization} to use the \string\datavisualization{} command}}%
\fi

\def\tikz@path@overlay#1{%
  \let\tikz@signal@path=\tikz@signal@path% for detection at begin of matrix cell
  \pgfutil@ifnextchar<{\tikz@path@overlayed{#1}}{\path #1}}%
\def\tikz@path@overlayed#1<#2>{\path<#2> #1}%

\def\tikz@uninstallcommands{%
  \let\scope=\tikz@origscope%
  \let\scoped=\tikz@origscoped%
  \let\endscope=\tikz@origendscope%
  \let\startscope=\tikz@origstartscope%
  \let\stopscope=\tikz@origstopscope%
  \let\path=\tikz@origpath%
  \let\againpath=\tikz@origagainpath%
  \let\draw=\tikz@origdraw%
  \let\pattern=\tikz@origpattern%
  \let\fill=\tikz@origfill%
  \let\filldraw=\tikz@origfilldraw%
  \let\shade=\tikz@origshade%
  \let\shadedraw=\tikz@origshadedraw%
  \let\clip=\tikz@origclip%
  \let\useasboundingbox=\tikz@origuseasboundingbox%
  \let\node=\tikz@orignode%
  \let\pic=\tikz@origpic%
  \let\coordinate=\tikz@origcoordinate%
  \let\matrix=\tikz@origmatrix%
  \let\calendar=\tikz@origcalendar%
  \let\datavisualization=\tikz@origdv%
  \let\graph=\tikz@origgraph%
}%


{%
  \catcode`\;=12
  \gdef\tikz@nonactivesemicolon{;}%
  \catcode`\:=12
  \gdef\tikz@nonactivecolon{:}%
  \catcode`\|=12
  \gdef\tikz@nonactivebar{|}%
  \catcode`\!=12
  \gdef\tikz@nonactiveexlmark{!}%
  \catcode`\;=\active
  \catcode`\:=\active
  \catcode`\|=\active
  \catcode`\"=\active
  \catcode`\!=\active
  \gdef\tikz@activesemicolon{;}%
  \gdef\tikz@activecolon{:}%
  \gdef\tikz@activebar{|}%
  \gdef\tikz@activequotes{"}%
  \global\let\tikz@active@quotes@token="%
  \gdef\tikz@activeexlmark{!}%
  \gdef\tikz@deactivatthings{%
    \def;{\tikz@nonactivesemicolon}%
    \def:{\tikz@nonactivecolon}%
    \def|{\tikz@nonactivebar}%
    \def!{\tikz@nonactiveexlmark}%
  }%
}%

\let\tikz@orig@shorthands\pgfutil@empty
\def\tikz@switchoff@shorthands{%
  \ifx\tikz@orig@shorthands\pgfutil@empty%
    \edef\tikz@orig@shorthands{%
      \catcode\noexpand`\noexpand\;\the\catcode`\;\relax%
      \catcode\noexpand`\noexpand\:\the\catcode`\:\relax%
      \catcode\noexpand`\noexpand\|\the\catcode`\|\relax%
      \catcode\noexpand`\noexpand\!\the\catcode`\!\relax%
      \catcode\noexpand`\noexpand\,\the\catcode`\,\relax%
      \catcode\noexpand`\noexpand\<\the\catcode`\<\relax%
      \catcode\noexpand`\noexpand\>\the\catcode`\>\relax%
      \catcode\noexpand`\noexpand\"\the\catcode`\"\relax%
      \catcode\noexpand`\noexpand\'\the\catcode`\'\relax%
      \catcode\noexpand`\noexpand\-\the\catcode`\-\relax%
      \catcode\noexpand`\noexpand\=\the\catcode`\=\relax%
      \catcode\noexpand`\noexpand\.\the\catcode`\.\relax%
      \catcode\noexpand`\noexpand\$\the\catcode`\$\relax%
    }%
    \catcode`\;12\relax%
    \catcode`\:12\relax%
    \catcode`\|12\relax%
    \catcode`\!12\relax%
    \catcode`\,12\relax%
    \catcode`\<12\relax%
    \catcode`\>12\relax%
    \catcode`\"12\relax%
    \catcode`\'12\relax%
    \catcode`\-12\relax%
    \catcode`\=12\relax%
    \catcode`\.12\relax%
    \catcode`\$3\relax%
  \fi%
}%



% Constructs a path and draws/fills them according to the current
% settings.

\def\tikz@command@path{%
  \let\tikz@signal@path=\tikz@signal@path% for detection at begin of matrix cell
  \pgfutil@ifnextchar[{\tikz@check@earg}%]
  {\pgfutil@ifnextchar<{\tikz@doopt}{\tikz@@command@path}}}%
\pgfutil@protected\def\tikz@signal@path{\tikz@signal@path}%
\def\tikz@check@earg[#1]{%
  \pgfutil@ifnextchar<{\tikz@swap@args[#1]}{\tikz@@command@path[#1]}}
\def\tikz@swap@args[#1]<#2>{\tikz@command@path<#2>[#1]}%

\def\tikz@doopt{%
  \let\tikz@next=\tikz@eargnormalsemicolon%
  \ifnum\the\catcode`\;=\active\relax%
    \let\tikz@next=\tikz@eargactivesemicolon%
  \fi%
  \tikz@next}%
\long\def\tikz@eargnormalsemicolon<#1>#2;{\alt<#1>{\tikz@@command@path#2;}{\tikz@path@do@at@end}}%
{
  \catcode`\;=\active
  \long\global\def\tikz@eargactivesemicolon<#1>#2;{\alt<#1>{\tikz@@command@path#2;}{\tikz@path@do@at@end}}%
}

\def\tikz@@command@path{%
  \edef\tikzscope@linewidth{\the\pgflinewidth}%
  \begingroup%
    \setbox\tikz@figbox=\box\pgfutil@voidb@x%
    \setbox\tikz@figbox@bg=\box\pgfutil@voidb@x%
    \let\tikz@path@do@at@end=\tikz@lib@scope@check%
    \let\tikz@options=\pgfutil@empty%
    \tikz@clear@rdf@options%
    \let\tikz@mode=\pgfutil@empty%
    \let\tikz@moveto@waiting=\relax%
    \let\tikz@timer=\relax%
    \let\tikz@tangent=\relax%
    \let\tikz@collected@onpath=\pgfutil@empty%
    \let\tikz@preactions=\pgfutil@empty%
    \let\tikz@postactions=\pgfutil@empty%
    \tikz@snakedfalse%
    \tikz@decoratepathfalse%
    \tikz@node@is@a@labelfalse%
    \tikz@resetexpandcount
    \pgf@path@lastx=0pt%
    \pgf@path@lasty=0pt%
    \tikz@lastx=0pt%
    \tikz@lasty=0pt%
    \tikz@lastxsaved=0pt%
    \tikz@lastysaved=0pt%
    \tikzset{every path/.try}%
    \tikz@scan@next@command%
}%
\def\tikz@scan@next@command{%
  \ifx\tikz@collected@onpath\pgfutil@empty%
  \else%
    \tikz@invoke@collected@onpath%
  \fi%
  \afterassignment\tikz@handle\let\pgf@let@token=%
}%
\newcount\tikz@expandcount
\def\tikz@resetexpandcount{\tikz@expandcount=100\relax}
\let\tikz@collected@onpath=\pgfutil@empty%

\edef\tikz@frozen@relax@token{\ifnum0=0\fi}

% Central dispatcher for commands
\def\tikz@handle{%
  \pgfutil@switch\pgfutil@ifx\pgf@let@token{%
    {(}{\let\pgfutil@next\tikz@movetoabs}%)
    {+}{\let\pgfutil@next\tikz@movetorel}%
    {-}{\let\pgfutil@next\tikz@lineto}%
    {.}{\let\pgfutil@next\tikz@dot}%
    {r}{\let\pgfutil@next\tikz@rect}%
    {n}{\let\pgfutil@next\tikz@fig}%
    {[}{\let\pgfutil@next\tikz@parse@options}%]
    {c}{\let\pgfutil@next\tikz@cchar}%
    {\bgroup}{\let\pgfutil@next\tikz@beginscope}%
    {\egroup}{\let\pgfutil@next\tikz@endscope}%
    {;}{\let\pgfutil@next\tikz@finish}%
    {a}{\let\pgfutil@next\tikz@a@char}%
    {e}{\let\pgfutil@next\tikz@e@char}%
    {g}{\let\pgfutil@next\tikz@g@char}%
    {s}{\let\pgfutil@next\tikz@schar}%
    {|}{\let\pgfutil@next\tikz@vh@lineto}%
    {p}{\pgfsetmovetofirstplotpoint\let\pgfutil@next\tikz@pchar}%
    {t}{\let\pgfutil@next\tikz@to}%
    {\pgfextra}{\let\pgfutil@next\tikz@extra}%
    {\foreach}{\let\pgfutil@next\tikz@foreach}%
    {f}{\let\pgfutil@next\tikz@fchar}%
    {\pgf@stop}{\let\pgfutil@next\relax}%
    {\par}{\let\pgfutil@next\tikz@scan@next@command}%
    {d}{\let\pgfutil@next\tikz@decoration}%
    {l}{\let\pgfutil@next\tikz@l@char}%
    {:}{\let\pgfutil@next\tikz@colon@char}%
    {\relax}{\relax\let\pgfutil@next\tikz@scan@next@command}%
  }{\tikz@resetexpandcount\pgfutil@next}{\tikz@expand}%
}%

\def\tikz@l@char{%
  \pgfutil@ifnextchar e{\tikz@let@command}{%
    \pgfutil@ifnextchar i{\tikz@lsystem}{%
      \pgfutil@ifnextchar-{\tikz@@lsystem}{\tikz@expand}%
    }%
  }%
}%

\def\tikz@lsystem{%
  \tikzerror{You need to say \string\usetikzlibrary{lindenmayersystems} to draw Lindenmayer systems}
}%

\def\tikz@@lsystem{%
  \tikzerror{You need to say \string\usetikzlibrary{lindenmayersystems} to draw L-systems}
}%

\def\tikz@pchar{\pgfutil@ifnextchar l{\tikz@plot}{\pgfutil@ifnextchar i{\tikz@subpicture}{\tikz@parabola}}}%
\def\tikz@cchar{%
  \pgfutil@ifnextchar i{\tikz@circle}%
  {\pgfutil@ifnextchar h{\tikz@children}{\tikz@cochar}}}%
\def\tikz@cochar o{%
  \pgfutil@ifnextchar o{\tikz@coordinate}{\tikz@cosine}}%
\def\tikz@e@char{%
  \pgfutil@ifnextchar l{\tikz@ellipse}{\tikz@@e@char}}%
\def\tikz@a@char{%
  \pgfutil@ifnextchar r{\tikz@arcA}{\tikzerror{Arc expected}}}%
\def\tikz@@e@char dge{%
  \pgfutil@ifnextchar f{\tikz@edgetoparent}{\tikz@edge@plain}}%

\def\tikz@schar{\pgfutil@ifnextchar i{\tikz@sine}{\tikz@svg@path}}%

\def\tikz@g@char r{\pgfutil@ifnextchar i{\tikz@grid}{\tikz@graph}}%

% svg syntax
% svg[options] {...}

\def\tikz@svg@path{%
  \tikzerror{You need to say \string\usetikzlibrary{svg.path} to use the svg path command}
}%


\def\tikz@finish{%
  % Rendering pipeline
  %
  % Step 1: The path background box
  %
  \box\tikz@figbox@bg%
  %
  % Step 2: Decorate path
  %
  \iftikz@decoratepath%
    \tikz@lib@dec@decorate@path%
  \fi%
  %
  % Step 3: Preactions
  %
  \pgfsyssoftpath@getcurrentpath\tikz@actions@path%
  \edef\tikz@restorepathsize{%
    \global\pgf@pathmaxx=\the\pgf@pathmaxx%
    \global\pgf@pathmaxy=\the\pgf@pathmaxy%
    \global\pgf@pathminx=\the\pgf@pathminx%
    \global\pgf@pathminy=\the\pgf@pathminy%
  }%
  \tikz@preactions%
  %
  % Step 4: Reset modes
  %
  \let\tikz@path@picture=\pgfutil@empty%
  \tikz@mode@fillfalse%
  \tikz@mode@drawfalse%
  \tikz@mode@doublefalse%
  \tikz@mode@clipfalse%
  \tikz@mode@boundaryfalse%
  \tikz@mode@fade@pathfalse%
  \tikz@mode@fade@scopefalse%
  \edef\tikz@pathextend{%
    {\noexpand\pgfqpoint{\the\pgf@pathminx}{\the\pgf@pathminy}}%
    {\noexpand\pgfqpoint{\the\pgf@pathmaxx}{\the\pgf@pathmaxy}}%
  }%
  \tikz@mode% installs the mode settings
  % Path fading counts as an option:
  \iftikz@mode@fade@path%
    \tikz@addoption{%
      \iftikz@fade@adjust%
        \iftikz@mode@draw%
          \pgfsetfadingforcurrentpathstroked{\tikz@path@fading}{\tikz@do@fade@transform}%
        \else%
          \pgfsetfadingforcurrentpath{\tikz@path@fading}{\tikz@do@fade@transform}%
        \fi%
      \else%
        \pgfsetfading{\tikz@path@fading}{\tikz@do@fade@transform}%
      \fi%
      \tikz@mode@fade@pathfalse% no more fading...
    }%
  \fi%
  %
  % Step 5: Install scope fading
  %
  \iftikz@mode@fade@scope%
    \iftikz@fade@adjust%
      \iftikz@mode@draw%
        \pgfsetfadingforcurrentpathstroked{\tikz@scope@fading}{\tikz@do@fade@transform}%
      \else%
        \pgfsetfadingforcurrentpath{\tikz@scope@fading}{\tikz@do@fade@transform}%
      \fi%
    \else%
      \pgfsetfading{\tikz@scope@fading}{\tikz@do@fade@transform}%
    \fi%
    \tikz@mode@fade@scopefalse%
  \fi%
  %
  % Step 5': Setup options
  %
  \ifx\tikz@options\pgfutil@empty%
  \else%
    \pgfsys@beginscope%
      \let\pgfscope@stroke@color=\pgf@strokecolor@global%
      \let\pgfscope@fill@color=\pgf@fillcolor@global%
      \begingroup%
        \tikz@options%
  \fi%
  \tikz@do@rdf@pre@options%
  %
  % Step 5'': Setup animations
  %
  \tikz@is@nodefalse%
  \tikz@call@id@hook%
  \iftikz@mode@clip\else%
    \pgfidscope%
      \tikz@do@rdf@post@options%
      \begingroup%
  \fi% open an animation scope here, unless clipping is done
  %
  % Step 6: Do a fill if shade or a path picture follows.
  %
  \iftikz@mode@fill%
    \iftikz@mode@shade%
      \pgfsyssoftpath@getcurrentpath\tikz@temppath
      \pgfprocessround{\tikz@temppath}{\tikz@temppath}% change the path
      \pgfsyssoftpath@setcurrentpath\tikz@temppath%
      \pgfsyssoftpath@invokecurrentpath%
      \pgfpushtype%
      \pgfusetype{.path fill}%
      \pgfsys@fill%
      \pgfpoptype%
      \tikz@mode@fillfalse% no more filling...
    \else%
      \ifx\tikz@path@picture\pgfutil@empty%
      \else%
        \pgfsyssoftpath@getcurrentpath\tikz@temppath
        \pgfprocessround{\tikz@temppath}{\tikz@temppath}% change the path
        \pgfsyssoftpath@setcurrentpath\tikz@temppath%
        \pgfsyssoftpath@invokecurrentpath%
        \pgfpushtype%
        \pgfusetype{.path fill}%
        \pgfsys@fill%
        \pgfpoptype%
        \tikz@mode@fillfalse% no more filling...
      \fi%
    \fi%
  \fi%
  %
  % Step 7: Do a shade if necessary.
  %
  \iftikz@mode@shade%
    \pgfsyssoftpath@getcurrentpath\tikz@temppath
    \pgfprocessround{\tikz@temppath}{\tikz@temppath}% change the path
    \pgfsyssoftpath@setcurrentpath\tikz@temppath%
    \pgfpushtype%
    \pgfusetype{.path shade}%
    \pgfshadepath{\tikz@shading}{\tikz@shade@angle}%
    \pgfpoptype%
    \tikz@mode@shadefalse% no more shading...
  \fi%
  %
  % Step 8: Do a path picture if necessary.
  %
  \ifx\tikz@path@picture\pgfutil@empty%
  \else%
    \begingroup%
      \pgfusetype{.path picture}%
      \pgfidscope%
      \pgfsys@beginscope%
        \let\tikz@id@name\pgfutil@empty%
        \pgfclearid%
        \pgfsyssoftpath@getcurrentpath\tikz@temppath
        \pgfprocessround{\tikz@temppath}{\tikz@temppath}% change the path
        \pgfsyssoftpath@setcurrentpath\tikz@temppath%
        \pgfsyssoftpath@invokecurrentpath%
        \pgfsys@clipnext%
        \pgfsys@discardpath%
        \pgf@relevantforpicturesizefalse%
        \expandafter\def\csname pgf@sh@ns@path picture bounding box\endcsname{rectangle}
        \expandafter\edef\csname pgf@sh@np@path picture bounding box\endcsname{%
          \def\noexpand\southwest{\noexpand\pgfqpoint{\the\pgf@pathminx}{\the\pgf@pathminy}}%
          \def\noexpand\northeast{\noexpand\pgfqpoint{\the\pgf@pathmaxx}{\the\pgf@pathmaxy}}%
        }
        \expandafter\def\csname pgf@sh@nt@path picture bounding box\endcsname{{1}{0}{0}{1}{0pt}{0pt}}
        \expandafter\def\csname pgf@sh@pi@path picture bounding box\endcsname{\pgfpictureid}
        \pgfinterruptpath%
          \tikz@path@picture%
        \endpgfinterruptpath%
      \pgfsys@endscope%
      \endpgfidscope%
    \endgroup%
    \let\tikz@path@picture=\pgfutil@empty%
  \fi%
  %
  % Step 9: Double stroke, if necessary
  %
  \iftikz@mode@draw%
    \iftikz@mode@double%
      % Change line width
      \begingroup%
        \pgfsys@beginscope%
          \tikz@double@setup%
    \fi%
  \fi%
  %
  % Step 10: Do stroke/fill/clip as needed
  %
  \pgfpushtype%
  \edef\tikz@temp{\noexpand\pgfusepath{%
    \iftikz@mode@fill fill,\fi%
    \iftikz@mode@draw draw,\fi%
    \iftikz@mode@clip clip\fi%
    }}%
  \pgfusetype{.path}%
  \tikz@temp%
  \pgfpoptype%
  \tikz@mode@fillfalse% no more filling
  %
  % Step 11: Double stroke, if necessary
  %
  \iftikz@mode@draw%
    \iftikz@mode@double%
        \pgfsys@endscope%
      \endgroup%
    \fi%
  \fi%
  \tikz@mode@drawfalse% no more stroking
  %
  % Step 12: Postactions
  %
  \tikz@postactions%
  %
  % Step 13: Add labels and nodes
  %
  \box\tikz@figbox%
  %
  % Step 14: Close animations
  %
  \iftikz@mode@clip\else\endgroup\endpgfidscope\fi%
  %
  % Step 14: Close option brace
  %
  \ifx\tikz@options\pgfutil@empty%
  \else%
      \endgroup%
      \global\let\pgf@strokecolor@global=\pgfscope@stroke@color%
      \global\let\pgf@fillcolor@global=\pgfscope@fill@color%
    \pgfsys@endscope%
    \iftikz@mode@clip%
      \tikzerror{Extra options not allowed for clipping path command.}%
    \fi%
  \fi%
  \iftikz@mode@clip%
    \aftergroup\pgf@relevantforpicturesizefalse%
  \fi%
  \iftikz@mode@boundary%
    \aftergroup\pgf@relevantforpicturesizefalse%
  \fi%
  \endgroup%
  \global\pgflinewidth=\tikzscope@linewidth%
  \tikz@path@do@at@end%
}%
\let\tikz@lib@scope@check\pgfutil@empty% this is a hook for the scopes library
\def\tikz@path@do@at@end{\tikz@lib@scope@check}%
\def\tikz@@pathtext{@path}%

\def\pgf@outer@auto@adjust@hook{%
  {%
    \tikz@mode@drawfalse%
    \tikz@mode%
    \expandafter%
  }%
  \iftikz@mode@draw\else%
    \pgfkeyslet{/pgf/outer xsep}\pgf@zero@text
    \pgfkeyslet{/pgf/outer ysep}\pgf@zero@text
  \fi%
}%

% Extra actions

\def\tikz@extra@preaction#1{%
  {%
    \pgfsys@beginscope%
      \setbox\tikz@figbox=\box\pgfutil@voidb@x%
      \setbox\tikz@figbox@bg=\box\pgfutil@voidb@x%
      \path[#1];% do extra path
      \pgfsyssoftpath@setcurrentpath\tikz@actions@path% restore
      \tikz@restorepathsize%
    \pgfsys@endscope%
  }%
}%

\def\tikz@extra@postaction#1{%
  {%
    \pgfsys@beginscope%
      \setbox\tikz@figbox=\box\pgfutil@voidb@x%
      \setbox\tikz@figbox@bg=\box\pgfutil@voidb@x%
      \tikz@restorepathsize%
      \path[#1]\pgfextra{\pgfsyssoftpath@setcurrentpath\tikz@actions@path};% do extra path
      \pgf@resetpathsizes%
    \pgfsys@endscope%
  }%
}%



\def\tikz@skip#1{\tikz@scan@next@command#1}%
\def\tikz@expand{%
  \advance\tikz@expandcount by -1
  \ifnum\tikz@expandcount<0\relax%
    \expandafter\pgfutil@firstoftwo
  \else
    \expandafter\pgfutil@secondoftwo
  \fi
  {%
    \tikzerror{Giving up on this path. Did you forget a semicolon?}%
    % since the last token caused an error we should reinsert it and therefore save it
    \global\let\tikz@expand@last@token=\pgf@let@token
    \tikz@finish%
    %
    % To be combatible with `scopes` lib, which uses a redefined 
    % \tikz@lib@scope@check to check the next token, the reinsertion is done
    % here, not at the end of (every) \tikz@finish.
    %
    \expandafter\let\expandafter\tikz@expand@last@token@\csname tikz@expand@last@token\endcsname
    \global\let\tikz@expand@last@token=\relax
    \tikz@expand@last@token@
  }{%
    \tikz@@expand
  }%
}

\def\tikz@@expand{%
  \expandafter\tikz@scan@next@command\pgf@let@token}%



% Syntax for scopes:
% {scoped path commands}

\newif\iftikz@auto@end@path

\def\tikz@beginscope{\begingroup\tikz@auto@end@pathfalse\tikz@scan@next@command}%
\def\tikz@endscope{%
  \iftikz@auto@end@path\expandafter\tikz@finish\expandafter\egroup\else\expandafter\tikz@@endscope\fi%
}%
\def\tikz@@endscope{%
    \global\setbox\tikz@tempbox=\box\tikz@figbox%
    \global\setbox\tikz@tempbox@bg=\box\tikz@figbox@bg%
    \global\let\tikz@tangent@temp\tikz@tangent%
    \edef\tikz@marshal{%
      \tikz@lastx=\the\tikz@lastx%
      \tikz@lasty=\the\tikz@lasty%
      \iftikz@current@point@local%
      \else%
        \tikz@lastxsaved=\the\tikz@lastxsaved%
        \tikz@lastysaved=\the\tikz@lastysaved%
        \ifx\tikz@moveto@waiting\relax%
          \let\tikz@moveto@waiting\relax%
        \else%
          \def\noexpand\tikz@moveto@waiting{\tikz@moveto@waiting}%
        \fi%
        \iftikz@shapeborder%
          \noexpand\tikz@shapebordertrue%
          \def\noexpand\tikz@shapeborder@name{\tikz@shapeborder@name}%
        \else%
          \noexpand\tikz@shapeborderfalse%
        \fi%
      \fi%
    }%
    \expandafter%
  \endgroup\tikz@marshal%
  \setbox\tikz@figbox=\box\tikz@tempbox%
  \setbox\tikz@figbox@bg=\box\tikz@tempbox@bg%
  \let\tikz@tangent\tikz@tangent@temp%
  \tikz@scan@next@command}%


% Syntax for pgfextra:
% \pgfextra {normal tex text}
% \pgfextra normal tex text \endpgfextra

\def\tikz@extra{\pgfutil@ifnextchar\bgroup\tikz@@extra\relax}%
\long\def\tikz@@extra#1{#1\tikz@scan@next@command}%
\let\endpgfextra=\tikz@scan@next@command

\def\pgfextra{pgfextra}%


% Syntax for foreach:
%
% foreach \var in {list} {path text}
%
% or
%
% \foreach \var in {list} {path text}
%
% Example:
%
% \draw (0,0) \foreach \x in {1,2,3} {-- (\x,0) circle (1cm)} -- (5,5);

\def\tikz@fchar oreach{\tikz@foreach}%

\def\tikz@foreach{%
  \def\pgffor@beginhook{%
    \tikz@lastx=\tikz@foreach@save@lastx%
    \tikz@lasty=\tikz@foreach@save@lasty%
    \tikz@lastxsaved=\tikz@foreach@save@lastxsaved%
    \tikz@lastysaved=\tikz@foreach@save@lastysaved%
    \setbox\tikz@figbox=\box\tikz@tempbox%
    \setbox\tikz@figbox@bg=\box\tikz@tempbox@bg%
    \expandafter\tikz@scan@next@command\pgfutil@firstofone}%
  \def\pgffor@endhook{\pgfextra{%
      \xdef\tikz@foreach@save@lastx{\the\tikz@lastx}%
      \xdef\tikz@foreach@save@lasty{\the\tikz@lasty}%
      \xdef\tikz@foreach@save@lastxsaved{\the\tikz@lastxsaved}%
      \xdef\tikz@foreach@save@lastysaved{\the\tikz@lastysaved}%
      \global\setbox\tikz@tempbox=\box\tikz@figbox%
      \global\setbox\tikz@tempbox@bg=\box\tikz@figbox@bg%
      \pgfutil@gobble}}%
  \def\pgffor@afterhook{%
    \tikz@lastx=\tikz@foreach@save@lastx%
    \tikz@lasty=\tikz@foreach@save@lasty%
    \tikz@lastxsaved=\tikz@foreach@save@lastxsaved%
    \tikz@lastysaved=\tikz@foreach@save@lastysaved%
    \let\pgffor@beginhook\relax%
    \let\pgffor@endhook\relax%
    \let\pgffor@afterhook\relax%
    \setbox\tikz@figbox=\box\tikz@tempbox%
    \setbox\tikz@figbox@bg=\box\tikz@tempbox@bg%
    \tikz@scan@next@command}%
  \global\setbox\tikz@tempbox=\box\tikz@figbox%
  \global\setbox\tikz@tempbox@bg=\box\tikz@figbox@bg%
  \xdef\tikz@foreach@save@lastx{\the\tikz@lastx}%
  \xdef\tikz@foreach@save@lasty{\the\tikz@lasty}%
  \xdef\tikz@foreach@save@lastxsaved{\the\tikz@lastxsaved}%
  \xdef\tikz@foreach@save@lastysaved{\the\tikz@lastysaved}%
  \foreach}%


% Syntax for againpath:
% \againpath \somepathname

\def\tikz@command@againpath#1{%
  \pgfextra{%
    \pgfsyssoftpath@getcurrentpath\tikz@temp%
    \expandafter\pgfutil@g@addto@macro\expandafter\tikz@temp\expandafter{#1}%
    \pgfsyssoftpath@setcurrentpath\tikz@temp%
  }%
}%


% animation syntax
% :attribute = {...}

\def\tikz@colon@char#1=#2{%
  \tikz@scan@next@command{[animate={myself:{#1}={#2}}]}%
}%



%
% When this if is set, a just-scanned point is a shape and its border
% position still needs to be determined, depending on subsequent
% commands.
%

\newif\iftikz@shapeborder


% Syntax for moveto:
% <point>
\def\tikz@movetoabs{\tikz@moveto(}%
\def\tikz@movetorel{\tikz@moveto+}%
\def\tikz@moveto{%
  \tikz@scan@one@point{\tikz@@moveto}}%
\def\tikz@@moveto#1{%
    \tikz@make@last@position{#1}%
  \iftikz@shapeborder%
    % ok, the moveto will have to wait. flag that we have a moveto in
    % waiting:
    \edef\tikz@moveto@waiting{\tikz@shapeborder@name}%
  \else%
    \tikz@@movetosave{\tikz@last@position}%
    \let\tikz@moveto@waiting=\relax%
  \fi%
  \tikz@scan@next@command%
}%

  % Wrapper around \pgfpathmoveto that adds a save
\def\tikz@@movetosave#1{%
  {\pgftransformreset
    \pgf@process{#1}%
    \xdef\tikz@marshal{%
      \tikz@lastmovetox=\the\pgf@x\relax%
      \tikz@lastmovetoy=\the\pgf@y\relax%
    }%
  }%
  \tikz@marshal
  \pgfpathmoveto{#1}%
}%

  
\let\tikz@moveto@waiting=\relax % normally, nothing is waiting...

\def\tikz@flush@moveto{%
  \ifx\tikz@moveto@waiting\relax%
  \else%
    \tikz@@movetosave{\tikz@last@position}%
  \fi%
  \let\tikz@moveto@waiting=\relax%
}%


\def\tikz@flush@moveto@toward#1#2#3{%
  % #1 = a point towards which the last moveto should be corrected
  % #2 = a dimension to which the corrected x-coordinate should be stored
  % #3 = a dimension for the corrected y-coordinate
  \ifx\tikz@moveto@waiting\relax%
    % do nothing
  \else%
    \pgf@process{\pgfpointshapeborder{\tikz@moveto@waiting}{#1}}%
    #2=\pgf@x%
    #3=\pgf@y%
    \edef\tikz@timer@start{\noexpand\pgfqpoint{\the\pgf@x}{\the\pgf@y}}%
    \tikz@@movetosave{\pgfqpoint{\pgf@x}{\pgf@y}}%
  \fi%
  \let\tikz@moveto@waiting=\relax%
}%


%
% Collecting labels on the path
%

\def\tikz@collect@coordinate@onpath#1c{%
  \pgfutil@ifnextchar y{\tikz@cycle@expander@add#1}{\tikz@collect@coordinate@onpath@{#1}}}%
\def\tikz@collect@coordinate@onpath@#1oordinate%
\def\tikz@@collect@coordinate@opt#1[#2]{%
  \pgfutil@ifnextchar({\tikz@@collect@coordinate#1[#2]}%
\def\tikz@@collect@coordinate#1[#2](#3){%
  \tikz@collect@label@onpath#1node[shape=coordinate,#2](#3){}}%

\newif\iftikz@collect@pic

\def\tikz@collect@label@onpath#1node{%
  \expandafter\def\expandafter\tikz@collected@onpath\expandafter{\tikz@collected@onpath node}%
  \let\tikz@collect@cont#1%
  \tikz@collect@picfalse%
  \tikz@collect@label@scan}%

\def\tikz@collect@pic@onpath#1pic{%
  \expandafter\def\expandafter\tikz@collected@onpath\expandafter{\tikz@collected@onpath pic}%
  \let\tikz@collect@cont#1
  \tikz@collect@pictrue%
  \tikz@collect@label@scan}%

\def\tikz@collect@label@scan{%
  \pgfutil@ifnextchar f{\tikz@collect@nodes}{%
    \pgfutil@ifnextchar({\tikz@collect@paran}%
    {\pgfutil@ifnextchar[{\tikz@collect@options}%
      {\pgfutil@ifnextchar:{\tikz@collect@animation}%
        {\pgfutil@ifnextchar\bgroup{\tikz@collect@arg}%
          {\tikz@collect@cont}}}}}%
}%}}%

\def\tikz@collect@nodes foreach#1in{%
  \expandafter\def\expandafter\tikz@collected@onpath\expandafter{\tikz@collected@onpath foreach#1in}%
  \pgfutil@ifnextchar\bgroup\tikz@collect@nodes@group\tikz@collect@nodes@one%
}%
\def\tikz@collect@nodes@one#1{%
  \expandafter\def\expandafter\tikz@collected@onpath\expandafter{\tikz@collected@onpath #1}%
  \tikz@collect@label@scan%
}%
\def\tikz@collect@nodes@group#1{%
  \expandafter\def\expandafter\tikz@collected@onpath\expandafter{\tikz@collected@onpath{#1}}%
  \tikz@collect@label@scan%
}%

\def\tikz@collect@animation#1=#2{%
  \expandafter\def\expandafter\tikz@collected@onpath\expandafter{\tikz@collected@onpath#1={#2}}%
  \tikz@collect@label@scan%
}%
\def\tikz@collect@paran#1){%
  \expandafter\def\expandafter\tikz@collected@onpath\expandafter{\tikz@collected@onpath#1)}%
  \tikz@collect@label@scan%
}%
\def\tikz@collect@options#1]{%
  \expandafter\def\expandafter\tikz@collected@onpath\expandafter{\tikz@collected@onpath#1]}%
  \tikz@collect@label@scan%
}%
\def\tikz@collect@arg#1{%
  \iftikz@handle@active@nodes%
    \iftikz@collect@pic%
      \expandafter\def\expandafter\tikz@collected@onpath\expandafter{\tikz@collected@onpath{#1}}%
    \else%
      \expandafter\def\expandafter\tikz@collected@onpath\expandafter{\tikz@collected@onpath{\scantokens{#1}}}%
    \fi%
  \else%
    \expandafter\def\expandafter\tikz@collected@onpath\expandafter{\tikz@collected@onpath{#1}}%
  \fi%
  \tikz@collect@cont%
}%

\def\tikz@invoke@collected@onpath{%
  \tikz@node@is@a@labeltrue%
  \let\tikz@temp=\tikz@collected@onpath%
  \let\tikz@collected@onpath=\pgfutil@empty%
  \expandafter\tikz@scan@next@command\tikz@temp\pgf@stop%
  \tikz@node@is@a@labelfalse%
}%


%
% Macros for the cycle command
%

\def\tikz@cycle@expander#1{\pgfutil@ifnextchar c{\tikz@cycle@expander@{#1}}{#1}}%
\def\tikz@cycle@expander@#1c{\pgfutil@ifnextchar y{\tikz@cycle@expander@add{#1}}{#1c}}%
\def\tikz@cycle@expander@add#1ycle{#1(current subpath start)--cycle}%




% Syntax for lineto:
% -- <point>

\def\tikz@lineto{%
  \pgfutil@ifnextchar |%
  {\expandafter\tikz@hv@lineto\pgfutil@gobble}%
  {\expandafter\pgfutil@ifnextchar\tikz@activebar{\expandafter\tikz@hv@lineto\pgfutil@gobble}%
    {\expandafter\tikz@lineto@mid\pgfutil@gobble}}}%
\def\tikz@lineto@mid{%
  \pgfutil@ifnextchar n{\tikz@collect@label@onpath\tikz@lineto@mid}%
  {%
    \pgfutil@ifnextchar c{\tikz@close}{%
      \pgfutil@ifnextchar p{\tikz@lineto@plot@or@pic}{\tikz@scan@one@point{\tikz@@lineto}}}}}%
\def\tikz@lineto@plot@or@pic p{%
  \pgfutil@ifnextchar i{\tikz@collect@pic@onpath\tikz@lineto@mid p}{%
    \pgfsetlinetofirstplotpoint\tikz@plot}%
}%
\def\tikz@@lineto#1{%
  % Record the starting point for later labels on the path:
  \edef\tikz@timer@start{\noexpand\pgfqpoint{\the\tikz@lastx}{\the\tikz@lasty}}
  \iftikz@shapeborder%
    % ok, target is a shape. recalculate end
    \pgf@process{\pgfpointshapeborder{\tikz@shapeborder@name}{\tikz@last@position}}%
    \tikz@make@last@position{\pgfqpoint{\pgf@x}{\pgf@y}}%
    \tikz@flush@moveto@toward{\tikz@last@position}\pgf@x\pgf@y%
    \tikz@path@lineto{\tikz@last@position}%
    \edef\tikz@timer@end{\noexpand\pgfqpoint{\the\tikz@lastx}{\the\tikz@lasty}}%
    \tikz@make@last@position{#1}%
    \edef\tikz@moveto@waiting{\tikz@shapeborder@name}%
  \else%
    % target is a reasonable point...
    % Record the starting point for later labels on the path:
    \tikz@make@last@position{#1}%
    \tikz@flush@moveto@toward{\tikz@last@position}\pgf@x\pgf@y%
    \tikz@path@lineto{\tikz@last@position}%
    \edef\tikz@timer@end{\noexpand\pgfqpoint{\the\tikz@lastx}{\the\tikz@lasty}}%
  \fi%
  \let\tikz@timer=\tikz@timer@line%
  \let\tikz@tangent\tikz@timer@start%
  \tikz@scan@next@command%
}%

% snake or lineto?
\def\tikz@path@lineto#1{%
  \iftikz@snaked%
    {
      \pgfsyssoftpathmovetorelevantfalse%
      \pgfpathsnakesto{\tikz@presnake,{\tikz@snake}{\tikz@mainsnakelength}{\noexpand\tikz@snake@install@trans}{},\tikz@postsnake}{#1}%
    }
  \else%
    \pgfpathlineto{#1}%
  \fi%
}%

% snake or lineto?
\def\tikz@path@close#1{%
  \iftikz@snaked%
    {%
      \pgftransformreset%
      \pgfpathsnakesto{\tikz@presnake,{\tikz@snake}{\tikz@mainsnakelength}{\noexpand\tikz@snake@install@trans}{},\tikz@postsnake}{#1}%
    }%
  \fi%
  \pgfpathclose%
}%


% Syntax for lineto horizontal/vertical:
% -| <point>

\def\tikz@hv@lineto{%
  \pgfutil@ifnextchar n{\tikz@collect@label@onpath\tikz@hv@lineto}{%
  \pgfutil@ifnextchar p{\tikz@collect@pic@onpath\tikz@hv@lineto}%
  {\pgfutil@ifnextchar c{\tikz@collect@coordinate@onpath\tikz@hv@lineto}%
    {\tikz@scan@one@point{\tikz@@hv@lineto}}}}}%
\def\tikz@@hv@lineto#1{%
  \edef\tikz@timer@start{\noexpand\pgfqpoint{\the\tikz@lastx}{\the\tikz@lasty}}%
  \pgf@yc=\tikz@lasty%
  \tikz@make@last@position{#1}%
  \edef\tikz@tangent{\noexpand\pgfqpoint{\the\tikz@lastx}{\the\pgf@yc}}%
  \tikz@flush@moveto@toward{\pgfqpoint{\tikz@lastx}{\pgf@yc}}\pgf@x\pgf@yc%
  \iftikz@shapeborder%
    % ok, target is a shape. have to work now:
    {%
      \pgf@process{\pgfpointshapeborder{\tikz@shapeborder@name}{\pgfqpoint{\tikz@lastx}{\pgf@yc}}}%
      \tikz@make@last@position{\pgfqpoint{\pgf@x}{\pgf@y}}%
      \tikz@path@lineto{\pgfqpoint{\tikz@lastx}{\pgf@yc}}%
      \tikz@path@lineto{\tikz@last@position}%
      \xdef\tikz@timer@end@temp{\noexpand\pgfqpoint{\the\tikz@lastx}{\the\tikz@lasty}}% move out of group
    }%
    \let\tikz@timer@end=\tikz@timer@end@temp%
    \edef\tikz@moveto@waiting{\tikz@shapeborder@name}%
  \else%
    \tikz@path@lineto{\pgfqpoint{\tikz@lastx}{\pgf@yc}}%
    \tikz@path@lineto{\tikz@last@position}%
    \edef\tikz@timer@end{\noexpand\pgfqpoint{\the\tikz@lastx}{\the\tikz@lasty}}% move out of group
  \fi%
  \let\tikz@timer=\tikz@timer@hvline%
  \tikz@scan@next@command%
}%

% Syntax for lineto vertical/horizontal:
% |- <point>

\def\tikz@vh@lineto-{\tikz@vh@lineto@next}%
\def\tikz@vh@lineto@next{%
  \pgfutil@ifnextchar n{\tikz@collect@label@onpath\tikz@vh@lineto@next}{%
  \pgfutil@ifnextchar p{\tikz@collect@pic@onpath\tikz@vh@lineto@next}%
  {\pgfutil@ifnextchar c{\tikz@collect@coordinate@onpath\tikz@vh@lineto@next}%
    {\tikz@scan@one@point\tikz@@vh@lineto}}}}%
\def\tikz@@vh@lineto#1{%
  \edef\tikz@timer@start{\noexpand\pgfqpoint{\the\tikz@lastx}{\the\tikz@lasty}}%
  \pgf@xc=\tikz@lastx%
  \tikz@make@last@position{#1}%
  \edef\tikz@tangent{\noexpand\pgfqpoint{\the\pgf@xc}{\the\tikz@lasty}}%
  \tikz@flush@moveto@toward{\pgfqpoint{\pgf@xc}{\tikz@lasty}}\pgf@xc\pgf@y%
  \iftikz@shapeborder%
    % ok, target is a shape. have to work now:
    {%
      \pgf@process{\pgfpointshapeborder{\tikz@shapeborder@name}{\pgfqpoint{\pgf@xc}{\tikz@lasty}}}%
      \tikz@make@last@position{\pgfqpoint{\pgf@x}{\pgf@y}}%
      \tikz@path@lineto{\pgfqpoint{\pgf@xc}{\tikz@lasty}}%
      \tikz@path@lineto{\tikz@last@position}%
      \xdef\tikz@timer@end@temp{\noexpand\pgfqpoint{\the\tikz@lastx}{\the\tikz@lasty}}% move out of group
    }%
    \let\tikz@timer@end=\tikz@timer@end@temp%
    \edef\tikz@moveto@waiting{\tikz@shapeborder@name}%
  \else%
    \tikz@path@lineto{\pgfqpoint{\pgf@xc}{\tikz@lasty}}%
    \tikz@path@lineto{\tikz@last@position}%
    \edef\tikz@timer@end{\noexpand\pgfqpoint{\the\tikz@lastx}{\the\tikz@lasty}}%
  \fi%
  \let\tikz@timer=\tikz@timer@vhline%
  \tikz@scan@next@command%
}%

% Syntax for cycle:
% -- cycle
\def\tikz@close c{%
  \pgfutil@ifnextchar o{\tikz@collect@coordinate@onpath\tikz@lineto@mid c}% oops, a coordinate
  {\tikz@@close c}}%
\def\tikz@@close cycle{%
  \tikz@flush@moveto%
  \edef\tikz@timer@start{\noexpand\pgfqpoint{\the\tikz@lastx}{\the\tikz@lasty}}
  \tikz@make@last@position{\expandafter\pgfpoint\pgfsyssoftpath@lastmoveto}%
  \tikz@path@close{\expandafter\pgfpoint\pgfsyssoftpath@lastmoveto}%
  \def\pgfstrokehook{}%
  \edef\tikz@timer@end{%\noexpand\pgfqpoint{\the\tikz@lastx}{\the\tikz@lasty}}%
      \noexpand\pgfqpoint{\the\tikz@lastmovetox}{\the\tikz@lastmovetoy}}%
  \let\tikz@timer=\tikz@timer@line%
  \let\tikz@tangent\tikz@timer@start%
  \tikz@scan@next@command%
}%


% Syntax for options:
% [options]
\def\tikz@parse@options#1]{%
  \tikzset{#1}%
  \tikz@scan@next@command%
}%

% Syntax for edges:
% edge [options] (coordinate)
% edge [options] node {node text} (coordinate)
% edge :attribute={...} [options] node {node text} (coordinate)
\def\tikz@edge@plain{%
  \begingroup%
    \ifx\tikz@to@use@whom\pgfutil@undefined\else\tikz@to@use@whom\fi
    \let\tikz@to@or@edge@function=\tikz@do@edge%
    \let\tikz@@to@local@options\pgfutil@empty%
    \let\tikz@collected@onpath=\pgfutil@empty%
    \tikz@to@or@edge}%

% Syntax for to paths:
% to [options] (coordinate)
% to [options] node {node text} (coordinate)
% to :attribute={...} [options] node {node text} (coordinate)
\def\tikz@to o{%
  \tikz@to@use@last@coordinate%
  \let\tikz@to@or@edge@function=\tikz@do@to%
  \let\tikz@@to@local@options\pgfutil@empty%
  \let\tikz@collected@onpath=\pgfutil@empty%
  \tikz@to@or@edge}%

\def\tikz@to@or@edge{%
  \pgfutil@ifnextchar[{\tikz@to@or@edge@option}{%
    \pgfutil@ifnextchar:{\tikz@to@or@edge@animation}{%
      \tikz@@to@collect}}%]
}%
\def\tikz@to@or@edge@option[#1]{%
  \expandafter\def\expandafter\tikz@@to@local@options\expandafter{\tikz@@to@local@options,#1}%
  \tikz@to@or@edge%
}%
\def\tikz@to@or@edge@animation:#1=#2{%
  \expandafter\def\expandafter\tikz@@to@local@options\expandafter{\tikz@@to@local@options,%
    animate={myself:{#1}={#2}}}%
  \tikz@to@or@edge%
}%
\def\tikz@@to@collect{%
  \pgfutil@ifnextchar(\tikz@@to@or@edge@coordinate%)
  {\pgfutil@ifnextchar n{\tikz@collect@label@onpath\tikz@@to@collect}%
    {\pgfutil@ifnextchar p{\tikz@collect@pic@onpath\tikz@@to@collect}%
      {\pgfutil@ifnextchar c{\tikz@collect@coordinate@onpath\tikz@@to@collect}%
        {\pgfutil@ifnextchar +{\tikz@scan@one@point\tikz@@to@or@edge@math}%
          {\tikzerror{(, +, coordinate, pic, or node expected}%)
            \tikz@@to@or@edge@coordinate()}}}}}%
}%

\def\tikz@@to@or@edge@coordinate({%
  \pgfutil@ifnextchar${%$
    % Ok, parse directly
    \tikz@scan@one@point\tikz@@to@or@edge@math(%
  }{%
    \pgfutil@ifnextchar[{%]
      \tikz@scan@one@point\tikz@@to@or@edge@math(%
    }{%
      \tikz@@to@or@edge@@coordinate(%
    }%
  }%
}%
\def\tikz@@to@or@edge@math#1{%
  \pgf@process{#1}%
  \iftikz@updatecurrent\else
    \tikz@updatenextfalse
  \fi
  \edef\tikztotarget{\the\pgf@x,\the\pgf@y}%
  \tikz@to@or@edge@function%
}%

\def\tikz@@to@or@edge@@coordinate(#1){%
  \def\tikztotarget{#1}%
  \tikz@to@or@edge@function%
}%

\def\tikz@do@edge{%
  \ifx\tikz@edge@macro\pgfutil@empty%
    \setbox\tikz@whichbox=\hbox\bgroup%
      \unhbox\tikz@whichbox%
      \hbox\bgroup
        \bgroup%
          \pgfinterruptpath%
            \pgfscope%
              \let\tikz@transform=\pgfutil@empty%
              \let\tikz@options=\pgfutil@empty%
              \tikz@clear@rdf@options%
              \let\tikz@tonodes=\tikz@collected@onpath%
              \def\tikztonodes{{\pgfextra{\tikz@node@is@a@labeltrue}\tikz@tonodes}}%
              \let\tikz@collected@onpath=\pgfutil@empty%
              \tikz@options%
              \tikz@do@rdf@pre@options%
              \pgfidscope%
              \tikz@do@rdf@post@options%
              \tikz@transform%
              \let\tikz@transform=\relax%
              % Typeset node:
              \let\tikz@after@path\pgfutil@empty%
              \tikz@atbegin@to%
                \tikz@enable@edge@quotes%
                \path[style=every edge]\expandafter[\tikz@@to@local@options](\tikztostart)\tikz@to@path
                \pgfextra{\global\let\tikz@after@path@smuggle=\tikz@after@path};%
              \tikz@atend@to%
              \endpgfidscope%
            \endpgfscope%
          \endpgfinterruptpath%
        \egroup
      \egroup%
    \egroup%
    \global\setbox\tikz@tempbox=\box\tikz@whichbox%
    \expandafter\endgroup%
    \expandafter\setbox\tikz@whichbox=\box\tikz@tempbox%
  \else%
      \expandafter\expandafter\expandafter\tikz@edge@macro%
      \expandafter\expandafter\expandafter{\expandafter\tikz@@to@local@options\expandafter}\expandafter{\tikz@collected@onpath}%
    \endgroup%
    \let\tikz@after@path@smuggle=\pgfutil@empty%
  \fi%
  \expandafter\tikz@scan@next@command\tikz@after@path@smuggle%
}%

\def\tikz@do@to{%
  \let\tikz@tonodes=\tikz@collected@onpath%
  \def\tikztonodes{{\pgfextra{\tikz@node@is@a@labeltrue}\tikz@tonodes}}%
  \let\tikz@collected@onpath=\pgfutil@empty%
  \tikz@scan@next@command%
  {%
    \pgfextra{\let\tikz@after@path\pgfutil@empty}%
    \pgfextra{\tikz@atbegin@to}%
    \pgfextra{\tikz@enable@edge@quotes}%
    [style=every to]\expandafter[\tikz@@to@local@options]\tikz@to@path%
    \pgfextra{\tikz@atend@to}%
    \pgf@stop%
    \expandafter\tikz@scan@next@command\expandafter%
  }\tikz@after@path%
  \pgfextra{\tikz@updatenexttrue\tikz@updatecurrenttrue}%
}%


\def\tikz@to@use@last@coordinate{%
  \iftikz@shapeborder%
    \edef\tikztostart{\tikz@shapeborder@name}%
  \else%
    \edef\tikztostart{\the\tikz@lastx,\the\tikz@lasty}%
  \fi%
}%
\def\tikz@to@use@last@fig@name{%
  \edef\tikztostart{\tikz@to@last@fig@name}%
}%



% Syntax for graph path command:
% graph [options] {...}
% See the graph library for details

\def\tikz@graph aph{\tikz@lib@graph@parser}%

\def\tikz@lib@graph@parser{\pgfutil@ifnextchar[\tikz@graph@error{\tikz@graph@error[]}}%]%
\def\tikz@graph@error[#1]#2{%
  \tikzerror{You need to say \string\usetikzlibrary{graphs} in order to use the graph syntax}%
  \tikz@lib@graph@parser@done%
}%

\def\tikz@lib@graph@parser@done{%
  \tikz@scan@next@command%
}%




% Syntax for edge from parent:
% edge from parent [options]
\def\tikz@edgetoparent from parent{\pgfutil@ifnextchar[\tikz@@edgetoparent{\tikz@@edgetoparent[]}}%}%
\def\tikz@@edgetoparent[#1]{%
  \let\tikz@edge@to@parent@needed=\pgfutil@empty%
  \def\tikz@edgetoparent@options{#1}%
  \begingroup%
    \let\tikz@collected@onpath=\pgfutil@empty%
    \tikz@edgetoparentcollect%
}%
\def\tikz@edgetoparentcollect{%
  \pgfutil@ifnextchar n{\tikz@collect@label@onpath\tikz@edgetoparentcollect}%
  {%
      \expandafter%
    \endgroup%
    \expandafter\tikz@edgetoparent@rollout\expandafter{\tikz@collected@onpath}%
  }%
}%

\def\tikz@edgetoparent@rollout#1{%
  \pgfkeysgetvalue{/tikz/edge from parent macro}\tikz@etop@temp
  \expandafter\tikz@scan@next@command\expandafter\tikz@etop@temp\expandafter{\tikz@edgetoparent@options}{#1}%
}%


% Syntax for bezier curves
% .. controls(point) and (point) .. (target)
% .. controls(point) .. (target)
% .. (target) % currently not supported

\def\tikz@dot.{\tikz@@dot}%
\def\tikz@@dot{%
  \pgfutil@ifnextchar n{\tikz@collect@label@onpath\tikz@@dot}{%
  \pgfutil@ifnextchar p{\tikz@collect@pic@onpath\tikz@@dot}%
  {\pgfutil@ifnextchar c{\tikz@curveto@double}{\tikz@curveto@auto}}}%
}%

\def\tikz@curveto@double co{%
  \pgfutil@ifnextchar o{\tikz@collect@coordinate@onpath\tikz@@dot co}
  {\tikz@cureveto@@double}}%
\def\tikz@cureveto@@double ntrols#1{%
  \tikz@scan@one@point\tikz@curveA#1%
}%
\def\tikz@curveA#1{%
  \edef\tikz@timer@start{\noexpand\pgfqpoint{\the\tikz@lastx}{\the\tikz@lasty}}%
  {%
    \tikz@lastxsaved=\tikz@lastx%
    \tikz@lastysaved=\tikz@lasty%
    \tikz@make@last@position{#1}%
    \xdef\tikz@curve@first{\noexpand\pgfqpoint{\the\tikz@lastx}{\the\tikz@lasty}}%
  }%
  \pgfutil@ifnextchar a
  {\tikz@curveBand}%
  {\let\tikz@curve@second\tikz@curve@first\tikz@curveCdots}%
}%
\def\tikz@curveBand and{%
  \tikz@scan@one@point\tikz@curveB%
}%
\def\tikz@curveB#1{%
  \def\tikz@curve@second{#1}%
  \tikz@curveCdots}
\def\tikz@curveCdots{%
  \afterassignment\tikz@curveCdot\let\pgfutil@next=}%
\def\tikz@curveCdot.{%
  \ifx\pgfutil@next.%
  \else%
    \tikzerror{Dot expected}%
  \fi%
  \iftikz@updatenext
    \tikz@updatecurrenttrue%
  \fi
  \tikz@curveCcheck%
}%
\def\tikz@curveCcheck{%
  \pgfutil@ifnextchar n{\tikz@collect@label@onpath\tikz@curveCcheck}{%
  \pgfutil@ifnextchar p{\tikz@collect@pic@onpath\tikz@curveCcheck}%
  {\pgfutil@ifnextchar c{\tikz@collect@coordinate@onpath\tikz@curveCcheck}
    {\tikz@scan@one@point\tikz@curveC}}}%
}%
\def\tikz@curveC#1{%
  \tikz@make@last@position{#1}%
  \edef\tikz@curve@third{\noexpand\pgfqpoint{\the\tikz@lastx}{\the\tikz@lasty}}%
  {%
    \tikz@lastxsaved=\tikz@lastx%
    \tikz@lastysaved=\tikz@lasty%
    \tikz@make@last@position{\tikz@curve@second}%
    \xdef\tikz@curve@second{\noexpand\pgfqpoint{\the\tikz@lastx}{\the\tikz@lasty}}%
  }%
  %
  % Start recalculating things in case start and end are shapes.
  %
  % First, the start:
  \ifx\tikz@moveto@waiting\relax%
  \else%
    \pgf@process{\pgfpointshapeborder{\tikz@moveto@waiting}{\tikz@curve@first}}%
    \edef\tikz@timer@start{\noexpand\pgfqpoint{\the\pgf@x}{\the\pgf@y}}%
  \tikz@@movetosave{\pgfqpoint{\pgf@x}{\pgf@y}}%
  \fi%
  \let\tikz@timer@cont@one=\tikz@curve@first%
  \let\tikz@timer@cont@two=\tikz@curve@second%
  % Second, the end:
  \iftikz@shapeborder%
    % ok, target is a shape. recalculate third
    {%
      \pgf@process{\pgfpointshapeborder{\tikz@shapeborder@name}{\tikz@curve@second}}%
      \tikz@make@last@position{\pgfqpoint{\pgf@x}{\pgf@y}}%
      \edef\tikz@curve@third{\noexpand\pgfqpoint{\the\tikz@lastx}{\the\tikz@lasty}}%
      \pgfpathcurveto{\tikz@curve@first}{\tikz@curve@second}{\tikz@curve@third}%
      \global\let\tikz@timer@end@temp=\tikz@curve@third% move out of group
    }%
    \let\tikz@timer@end=\tikz@timer@end@temp%
    \edef\tikz@moveto@waiting{\tikz@shapeborder@name}%
  \else%
    \pgfpathcurveto{\tikz@curve@first}{\tikz@curve@second}{\tikz@curve@third}%
    \let\tikz@timer@end=\tikz@curve@third
    \let\tikz@moveto@waiting=\relax%
  \fi%
  \let\tikz@timer=\tikz@timer@curve%
  \let\tikz@tangent=\tikz@curve@second%
  \tikz@scan@next@command%
}%


% Syntax for rectangles:
% rectangle <corner point>
\def\tikz@rect ectangle{%
  \tikz@flush@moveto%
  \edef\tikz@timer@start{\noexpand\pgfqpoint{\the\tikz@lastx}{\the\tikz@lasty}}%
  \tikz@@rect}%
\def\tikz@@rect{%
  \pgfutil@ifnextchar n{\tikz@collect@label@onpath\tikz@@rect}{%
  \pgfutil@ifnextchar p{\tikz@collect@pic@onpath\tikz@@rect}%
  {\pgfutil@ifnextchar c{\tikz@collect@coordinate@onpath\tikz@@rect}%
    {
      \pgf@xa=\tikz@lastx\relax%
      \pgf@ya=\tikz@lasty\relax%
      \tikz@scan@one@point\tikz@rectB}}}}%
\def\tikz@rectB#1{%
  \tikz@make@last@position{#1}%
  \edef\tikz@timer@end{\noexpand\pgfqpoint{\the\tikz@lastx}{\the\tikz@lasty}}%
  \let\tikz@timer=\tikz@timer@line%
          \tikz@@movetosave{\pgfqpoint{\pgf@xa}{\pgf@ya}}%
  \tikz@path@lineto{\pgfqpoint{\pgf@xa}{\tikz@lasty}}%
  \tikz@path@lineto{\pgfqpoint{\tikz@lastx}{\tikz@lasty}}%
  \tikz@path@lineto{\pgfqpoint{\tikz@lastx}{\pgf@ya}}%
  \iftikz@snaked%
    \tikz@path@lineto{\pgfqpoint{\pgf@xa}{\pgf@ya}}%
  \fi%
  \pgfpathclose%
          \tikz@@movetosave{\pgfqpoint{\tikz@lastx}{\tikz@lasty}}%
  \def\pgfstrokehook{}%
  \let\tikz@tangent\relax%
  \tikz@scan@next@command%
}%



% Syntax for grids:
% grid <corner point>
\def\tikz@grid id{%
  \tikz@flush@moveto%
  \pgf@xa=\tikz@lastx\relax%
  \pgf@ya=\tikz@lasty\relax%
  \pgfutil@ifnextchar[{\tikz@gridA}{\tikz@gridA[]}}%}%
\def\tikz@gridA[#1]{%
  \def\tikz@grid@options{#1}%
  \tikz@cycle@expander{\tikz@scan@one@point\tikz@gridB}}%
\def\tikz@gridB#1{%
  \tikz@make@last@position{#1}%
  \let\tikz@tangent\relax%
  {%
    \let\tikz@after@path\pgfutil@empty%
    \expandafter\tikzset\expandafter{\tikz@grid@options}
    \tikz@checkunit{\tikz@grid@x}%
    \iftikz@isdimension%
      \pgf@process{\pgfpoint{\tikz@grid@x}{0pt}}%
    \else%
      \pgf@process{\pgfpointxy{\tikz@grid@x}{0}}%
    \fi%
    \pgf@xb=\pgf@x%
    \pgf@yb=\pgf@y%
    \tikz@checkunit{\tikz@grid@y}%
    \iftikz@isdimension%
      \pgf@process{\pgfpoint{0pt}{\tikz@grid@y}}%
    \else%
      \pgf@process{\pgfpointxy{0}{\tikz@grid@y}}%
    \fi%
    \advance\pgf@xb by\pgf@x%
    \advance\pgf@yb by\pgf@y%
    \pgfpathgrid[stepx=\pgf@xb,stepy=\pgf@yb]%
      {\pgfqpoint{\pgf@xa}{\pgf@ya}}{\pgfqpoint{\tikz@lastx}{\tikz@lasty}}%
  \expandafter}%
  \expandafter\tikz@scan@next@command\tikz@after@path%
}%



% Syntax for plot:
% plot [local options] ...    % starts with a moveto
% -- plot [local options] ... % starts with a lineto
\def\tikz@plot lot{%
  \tikz@flush@moveto%
  \pgfutil@ifnextchar[{\tikz@@plot}{\tikz@@plot[]}}%}%
\def\tikz@@plot[#1]{%
  \let\tikz@tangent\tikz@tangent@lookup%
  \begingroup%
    \let\tikz@after@path\pgfutil@empty%
    \let\tikz@options=\pgfutil@empty%
    \tikzset{every plot/.try}%
    \tikzset{#1}%
    \pgfutil@ifnextchar f{\tikz@plot@f}%
    {\pgfutil@ifnextchar c{\tikz@plot@scan@points}%
      {\pgfutil@ifnextchar ({\tikz@plot@expression}{%
      \tikzerror{Cannot parse this plotting data}%
       \endgroup}}}}%
\def\tikz@plot@f f{\pgfutil@ifnextchar i{\tikz@plot@file}{\tikz@plot@function}}%

\def\tikz@plot@file ile#1{\def\tikz@plot@data{\pgfplotxyfile{#1}}\tikz@@@plot}%
\def\tikz@plot@scan@points coordinates#1{%
  \pgfplothandlerrecord\tikz@plot@data%
  \pgfplotstreamstart%
  \pgfutil@ifnextchar\pgf@stop{\pgfplotstreamend\expandafter\tikz@@@plot\pgfutil@gobble}
  {\tikz@scan@one@point\tikz@plot@next@point}%
  #1\pgf@stop%
}%
\def\tikz@plot@next@point#1{%
  \pgfplotstreampoint{#1}%
  \pgfutil@ifnextchar\pgf@stop{\pgfplotstreamend\expandafter\tikz@@@plot\pgfutil@gobble}%
  {\tikz@scan@one@point\tikz@plot@next@point}%
}%
\def\tikz@plot@function unction#1{%
  \def\tikz@plot@filename{\tikz@plot@prefix\tikz@plot@id}%
  \iftikz@plot@raw@gnuplot%
    \def\tikz@plot@data{\pgfplotgnuplot[\tikz@plot@filename]{#1}}%
  \else%
    \iftikz@plot@parametric%
      \def\tikz@plot@data{\pgfplotgnuplot[\tikz@plot@filename]{%
          set samples \tikz@plot@samples;
          set parametric;
          plot [t=\tikz@plot@domain]
          [\tikz@plot@xrange]
          [\tikz@plot@range]
          #1}}%
    \else%
      \def\tikz@plot@data{\pgfplotgnuplot[\tikz@plot@filename]{%
          set samples \tikz@plot@samples;
          plot [x=\tikz@plot@domain]
          \ifx\tikz@plot@range\pgfutil@empty\else[\tikz@plot@range]\fi
          #1}}%
    \fi%
  \fi%
  \tikz@@@plot%
}%

\def\tikz@plot@no@resample{%
  \pgfutil@IfFileExists{\tikz@plot@filename.table}%
  {\def\tikz@plot@data{\pgfplotxyfile{\tikz@plot@filename.table}}}%
  {}%
}%

\def\tikz@plot@expression(#1){%
  \edef\tikz@plot@data{\noexpand\pgfplotfunction{\expandafter\noexpand\tikz@plot@var}{\tikz@plot@samplesat}}%
  \expandafter\def\expandafter\tikz@plot@data\expandafter{\tikz@plot@data{\tikz@scan@one@point\pgfutil@firstofone(#1)}}%
  \tikz@@@plot%
}%

\def\tikz@@@plot{%
    \def\pgfplotlastpoint{\pgfpointorigin}%
    \tikz@plot@handler%
    \tikz@plot@data%
    \global\let\tikz@@@temp=\pgfplotlastpoint%
    \ifx\tikz@plot@mark\pgfutil@empty%
    \else%
      % Marks are drawn after the path.
      \setbox\tikz@whichbox=\hbox{%
        \unhbox\tikz@whichbox%
        \hbox{{%
          \pgfinterruptpath%
            \pgfscope%
              \let\tikz@options=\pgfutil@empty%
              \let\tikz@transform=\pgfutil@empty%
              \tikzset{every mark}%
              \tikz@options%
              \ifx\tikz@mark@list\pgfutil@empty%
                \pgfplothandlermark{\tikz@transform\pgfuseplotmark{\tikz@plot@mark}}%
              \else
                \pgfplothandlermarklisted{\tikz@transform\pgfuseplotmark{\tikz@plot@mark}}{\tikz@mark@list}%
              \fi
              \tikz@plot@data%
            \endpgfscope
          \endpgfinterruptpath%
        }}%
      }%
    \fi%
    \global\setbox\tikz@tempbox=\box\tikz@whichbox%
    \global\let\tikz@after@path@smuggle=\tikz@after@path
  \expandafter\endgroup%
  \expandafter\setbox\tikz@whichbox=\box\tikz@tempbox%
  \tikz@make@last@position{\tikz@@@temp}%
  \expandafter\tikz@scan@next@command\tikz@after@path@smuggle%
}%


\pgfdeclareplotmark{ball}
{%
  \def\tikz@shading{ball}%
  \shade (0pt,0pt) circle (\pgfplotmarksize);%
}%




% Syntax for cosine curves:
% cos <end of quarter-period>
\def\tikz@cosine s{\tikz@cycle@expander{\tikz@scan@one@point\tikz@@cosine}}
\def\tikz@@cosine#1{%
  \let\tikz@tangent\tikz@tangent@lookup%
  \tikz@flush@moveto%
  \pgf@process{#1}%
  \pgf@xc=\pgf@x%
  \pgf@yc=\pgf@y%
  \advance\pgf@xc by-\tikz@lastx%
  \advance\pgf@yc by-\tikz@lasty%
  \advance\tikz@lastx by\pgf@xc%
  \advance\tikz@lasty by\pgf@yc%
  \tikz@lastxsaved=\tikz@lastx%
  \tikz@lastysaved=\tikz@lasty%
  \tikz@updatecurrenttrue%
  \pgfpathcosine{\pgfqpoint{\pgf@xc}{\pgf@yc}}%
  \tikz@scan@next@command%
}%

% Syntax for sine curves:
% sin <end of quarter-period>
\def\tikz@sine in{\tikz@cycle@expander{\tikz@scan@one@point\tikz@@sine}}
\def\tikz@@sine#1{%
  \let\tikz@tangent\tikz@tangent@lookup%
  \tikz@flush@moveto%
  \pgf@process{#1}%
  \pgf@xc=\pgf@x%
  \pgf@yc=\pgf@y%
  \advance\pgf@xc by-\tikz@lastx%
  \advance\pgf@yc by-\tikz@lasty%
  \advance\tikz@lastx by\pgf@xc%
  \advance\tikz@lasty by\pgf@yc%
  \tikz@lastxsaved=\tikz@lastx%
  \tikz@lastysaved=\tikz@lasty%
  \tikz@updatecurrenttrue%
  \pgfpathsine{\pgfqpoint{\pgf@xc}{\pgf@yc}}%
  \tikz@scan@next@command%
}%

% Syntax for parabolas:
% parabola[options] bend <coordinate> <coordinate>
\def\tikz@parabola arabola{%
  \let\tikz@tangent\tikz@tangent@lookup%
  \pgfutil@ifnextchar[{\tikz@parabola@options}{\tikz@parabola@options[]}}%}%

\def\tikz@parabola@options[#1]{%
  \def\tikz@parabola@option{#1}%
  \pgfutil@ifnextchar b{\tikz@parabola@scan@bend}{\tikz@cycle@expander{\tikz@scan@one@point\tikz@parabola@semifinal}}}%
\def\tikz@parabola@scan@bend bend{\tikz@scan@one@point\tikz@parabola@scan@bendB}%
\def\tikz@parabola@scan@bendB#1{%
  \def\tikz@parabola@bend{#1}%
  \tikz@cycle@expander{\tikz@scan@one@point\tikz@parabola@semifinal}%
}%
\def\tikz@parabola@semifinal#1{%
  \tikz@flush@moveto%
  % Save original start:
  \pgf@xb=\tikz@lastx%
  \pgf@yb=\tikz@lasty%
  \tikz@make@last@position{#1}%
  \pgf@xc=\tikz@lastx%
  \pgf@yc=\tikz@lasty%
  \begingroup% now calculate bend:
    \let\tikz@after@path\pgfutil@empty%
    \expandafter\tikzset\expandafter{\tikz@parabola@option}%
    \tikz@lastxsaved=\tikz@parabola@bend@factor\tikz@lastx%
    \tikz@lastysaved=\tikz@parabola@bend@factor\tikz@lasty%
    \advance\tikz@lastxsaved by\pgf@xb%
    \advance\tikz@lastysaved by\pgf@yb%
    \advance\tikz@lastxsaved by-\tikz@parabola@bend@factor\pgf@xb%
    \advance\tikz@lastysaved by-\tikz@parabola@bend@factor\pgf@yb%
    \expandafter\tikz@make@last@position\expandafter{\tikz@parabola@bend}%
    % Calculate delta from bend
    \advance\pgf@xc by-\tikz@lastx%
    \advance\pgf@yc by-\tikz@lasty%
    % Ok, now calculate delta to bend
    \advance\tikz@lastx by-\pgf@xb%
    \advance\tikz@lasty by-\pgf@yb%
    \xdef\tikz@parabola@b{{\noexpand\pgfqpoint{\the\tikz@lastx}{\the\tikz@lasty}}{\noexpand\pgfqpoint{\the\pgf@xc}{\the\pgf@yc}}}%
  \expandafter\endgroup%
  \expandafter\expandafter\expandafter\pgfpathparabola\expandafter\tikz@parabola@b%
  \expandafter\tikz@scan@next@command\tikz@after@path%
}%


% Syntax for circles:
% circle [options] % where options should set, at least, radius
% circle (radius) % deprecated
%
% Syntax for ellipses:
% ellipse [options] % identical to circle.
% ellipse (x-radius and y-radius) % deprecated
%
% radii can be dimensionless, then they are in the xy-system
\def\tikz@circle ircle{\tikz@flush@moveto\tikz@@circle}%
\def\tikz@ellipse llipse{\tikz@flush@moveto\tikz@@circle}%
\def\tikz@@circle{%
  \let\tikz@tangent\relax%
  \pgfutil@ifnextchar(\tikz@@@circle
  {\pgfutil@ifnextchar[\tikz@circle@opt{%])
    \advance\tikz@expandcount by -10\relax% go down quickly
    \ifnum\tikz@expandcount<0\relax%
      \let\pgfutil@next=\tikz@@circle@normal%
    \else%
      \let\pgfutil@next=\tikz@@circle@scanexpand%
    \fi%
    \pgfutil@next%
  }}%
}%
\def\tikz@@circle@scanexpand{\expandafter\tikz@@circle}%
\def\tikz@@circle@normal{\tikz@circle@opt[]}%

\def\tikz@circle@opt[#1]{%
  {%
    \def\tikz@node@at{\tikz@last@position}%
    \let\tikz@transform=\pgfutil@empty%
    \tikzset{every circle/.try,#1}%
    \pgftransformshift{\tikz@node@at}%
    \tikz@transform%
    \tikz@do@ellipse{\pgfkeysvalueof{/tikz/x radius}}{\pgfkeysvalueof{/tikz/y radius}}
  }%
  \tikz@scan@next@command%
}%

\def\tikz@@@circle(#1){%
  {%
    \pgftransformshift{\tikz@last@position}%
    \pgfutil@in@{ and }{#1}%
    \ifpgfutil@in@%
      \tikz@@ellipseB(#1)%
    \else%
      \tikz@do@circle{#1}%
    \fi%
  }%
  \tikz@scan@next@command%
}%
\def\tikz@@ellipseB(#1 and #2){%
  \tikz@do@ellipse{#1}{#2}%
}%
\def\tikz@do@circle#1{%
  \pgfmathparse{#1}%
  \let\tikz@ellipse@x=\pgfmathresult
  \ifpgfmathunitsdeclared
    \pgfpathellipse{\pgfpointorigin}%
      {\pgfqpoint{\tikz@ellipse@x pt}{0pt}}%
      {\pgfpoint{0pt}{\tikz@ellipse@x pt}}%
  \else
    \pgfpathellipse{\pgfpointorigin}%
      {\pgfpointxy{\tikz@ellipse@x}{0}}%
      {\pgfpointxy{0}{\tikz@ellipse@x}}%
  \fi
}
\def\tikz@do@ellipse#1#2{
  \pgfmathparse{#1}%
  \let\tikz@ellipse@x=\pgfmathresult%
  \ifpgfmathunitsdeclared%
    \pgfmathparse{#2}%
    \let\tikz@ellipse@y=\pgfmathresult%
    \ifpgfmathunitsdeclared%
      \pgfpathellipse{\pgfpointorigin}{%
        \pgfqpoint{\tikz@ellipse@x pt}{0pt}}{\pgfpoint{0pt}{\tikz@ellipse@y pt}}%
    \else%
      \tikzerror{You cannot mix dimensions and dimensionless values in an ellipse}%
    \fi%
  \else%
    \pgfmathparse{#2}%
    \let\tikz@ellipse@y=\pgfmathresult%
    \ifpgfmathunitsdeclared%
      \tikzerror{You cannot mix dimensions and dimensionless values in an ellipse}%
    \else%
      \pgfpathellipse{\pgfpointorigin}{%
        \pgfpointxy{\tikz@ellipse@x}{0}}{\pgfpointxy{0}{\tikz@ellipse@y}}%
    \fi%
  \fi%
}%

% Syntax for arcs:
% arc [options]
%
% (The syntax with parentheses is deprecated.)
\def\tikz@arcA rc{\tikz@flush@moveto\tikz@arc@cont}%
\def\tikz@arc@cont{%
  \pgfutil@ifnextchar(%)
  {\tikz@@arcto}{%
    \pgfutil@ifnextchar[%]
    {\tikz@arc@opt}%
    {%
      \advance\tikz@expandcount by -10\relax% go down quickly
      \ifnum\tikz@expandcount<0\relax%
        \let\pgfutil@next=\tikz@@arc@normal%
      \else%
        \let\pgfutil@next=\tikz@@arc@scanexpand%
      \fi%
      \pgfutil@next%
    }%
  }%
}%
\def\tikz@@arc@scanexpand{\expandafter\tikz@arc@cont}%
\def\tikz@@arc@normal{\tikz@arc@opt[]}%


\def\tikz@arc@opt[#1]{%
  {%
    \tikzset{every arc/.try,#1}%
    \pgfkeysgetvalue{/tikz/start angle}\tikz@s
    \pgfkeysgetvalue{/tikz/end angle}\tikz@e
    \pgfkeysgetvalue{/tikz/delta angle}\tikz@d
    \ifx\tikz@s\pgfutil@empty%
      \pgfmathsetmacro\tikz@s{\tikz@e-\tikz@d}
    \else
      \ifx\tikz@e\pgfutil@empty%
        \pgfmathsetmacro\tikz@e{\tikz@s+\tikz@d}
      \fi%
    \fi%
    \xdef\pgf@marshal{\noexpand%
    \tikz@do@arc{\tikz@s}{\tikz@e}
      {\pgfkeysvalueof{/tikz/x radius}}
      {\pgfkeysvalueof{/tikz/y radius}}}%
  }%
  \pgf@marshal%
  \tikz@arcfinal%
}%

\def\tikz@@arcto(#1){%
  \edef\tikz@temp{(#1)}%
  \expandafter\tikz@@@arcto@check@slashand\tikz@temp%
}%

\def\tikz@@@arcto@check@slashand(#1:#2:#3){%
  \pgfutil@in@{ and }{#3}%
  \ifpgfutil@in@%
    \tikz@parse@arc@and(#1:#2:#3)%
  \else%
    \tikz@parse@arc@and(#1:#2:{#3} and {#3})%
  \fi%
  \tikz@arcfinal%
}%

\def\tikz@parse@arc@and(#1:#2:#3 and #4){%
  \tikz@do@arc{#1}{#2}{#3}{#4}%
}%
\def\tikz@do@arc#1#2#3#4{%
  \let\tikz@tangent\tikz@tangent@lookup%
  \edef\tikz@timer@start{\noexpand\pgfqpoint{\the\tikz@lastx}{\the\tikz@lasty}}%
  \pgfmathsetmacro\tikz@timer@start@angle{#1}%
  \pgfmathsetmacro\tikz@timer@end@angle{#2}%
  \pgfmathparse{#3}%
  \let\tikz@arc@x=\pgfmathresult%
  \ifpgfmathunitsdeclared%
    \pgfmathparse{#4}%
    \let\tikz@arc@y=\pgfmathresult%
    \ifpgfmathunitsdeclared%
      \tikz@@@arcfinal{\pgfpatharc{\tikz@timer@start@angle}{\tikz@timer@end@angle}{\tikz@arc@x pt and \tikz@arc@y pt}}
      {\pgfpointpolar{\tikz@timer@start@angle}{\tikz@arc@x pt and \tikz@arc@y pt}}
      {\pgfpointpolar{\tikz@timer@end@angle}{\tikz@arc@x pt and \tikz@arc@y pt}}%
      \edef\tikz@timer@zero@axis{\noexpand\pgfqpoint{\tikz@arc@x pt}{0pt}}
      \edef\tikz@timer@ninety@axis{\noexpand\pgfqpoint{0pt}{\tikz@arc@y pt}}
    \else%
      \tikzerror{You cannot mix dimensions and dimensionless values in an arc}%
    \fi%
  \else%
    \pgfmathparse{#4}%
    \let\tikz@arc@y=\pgfmathresult%
    \ifpgfmathunitsdeclared%
      \tikzerror{You cannot mix dimensions and dimensionless values in an arc}%
    \else%
      \tikz@@@arcfinal{\pgfpatharcaxes{\tikz@timer@start@angle}{\tikz@timer@end@angle}{\pgfpointxy{\tikz@arc@x}{0}}{\pgfpointxy{0}{\tikz@arc@y}}}
      {\pgfpointpolarxy{\tikz@timer@start@angle}{\tikz@arc@x and \tikz@arc@y}}{\pgfpointpolarxy{\tikz@timer@end@angle}{\tikz@arc@x and \tikz@arc@y}}%
      \pgf@process{\pgfpointxy{\tikz@arc@x}{0}}
      \edef\tikz@timer@zero@axis{\noexpand\pgfqpoint{\the\pgf@x}{\the\pgf@y}}
      \pgf@process{\pgfpointxy{0}{\tikz@arc@y}}
      \edef\tikz@timer@ninety@axis{\noexpand\pgfqpoint{\the\pgf@x}{\the\pgf@y}}
     \fi%
  \fi%
}%

\def\tikz@@@arcfinal#1#2#3{%
  #1%
  \pgf@process{#2}
  \xdef\tikz@arc@save@first{\pgfqpoint{\the\pgf@x}{\the\pgf@y}}%
  \pgf@process{#3}
  \xdef\tikz@arc@save@second{\pgfqpoint{\the\pgf@x}{\the\pgf@y}}%
}%

\def\tikz@arcfinal{%
  \pgf@process{\tikz@arc@save@first}%
  \advance\tikz@lastx by-\pgf@x%
  \advance\tikz@lasty by-\pgf@y%
  \pgf@process{\tikz@arc@save@second}%
  \advance\tikz@lastx by\pgf@x%
  \advance\tikz@lasty by\pgf@y%
  \tikz@lastxsaved=\tikz@lastx%
  \tikz@lastysaved=\tikz@lasty%
  \let\tikz@timer=\tikz@timer@arc%
  \tikz@scan@next@command%
}%


% Syntax for coordinates:
% coordinate[options] (coordinate name) at (point)
% where ``at (point)'' is optional
\def\tikz@coordinate ordinate{%
  \pgfutil@ifnextchar[{\tikz@@coordinate@opt}{\tikz@@coordinate@opt[]}}%
\def\tikz@@coordinate@opt[#1]%
\def\tikz@@coordinate[#1](#2){%
  \pgfutil@ifnextchar a{\tikz@@coordinate@at[#1](#2)}
  {\tikz@fig ode[shape=coordinate,#1](#2){}}}%
\def\tikz@@coordinate@at[#1](#2)a{%
  \pgfutil@ifnextchar t{\tikz@@coordinate@@at[#1](#2)a}%
  {\tikz@fig ode[shape=coordinate,#1](#2){}a}%
}%
\def\tikz@@coordinate@@at[#1](#2)at#3({%
  \def\tikz@coordinate@caller{\tikz@fig ode[shape=coordinate,#1](#2)at}%
  \tikz@scan@one@point\tikz@@coordinate@at@math(%
}%
\def\tikz@@coordinate@at@math#1{%
  \pgf@process{#1}%
  \edef\tikz@temp{(\the\pgf@x,\the\pgf@y)}%
  \expandafter\tikz@coordinate@caller\tikz@temp{}%
}%



% Syntax for nodes:
% node foreach \var in {list} ... :attribute={...} [options] (node name) at (pos) {label text}
%
% all of :attribute, [options], (node name), at(pos), and foreach are
% optional. There can be multiple options and the ordering is not
% important as in node[draw] (a) [rotate=10] {text}, *except* that all
% foreach statements must come first.
%
% A label text always ``ends'' the node.
%
\def\tikz@fig ode{%
  \pgfutil@ifnextchar a\tikz@test@also{%
    \pgfutil@ifnextchar f{\tikz@nodes@start}\tikz@normal@fig}}%
\def\tikz@test@also a{\pgfutil@ifnextchar l\tikz@node@also{\tikz@normal@fig a}}%
\def\tikz@normal@fig{%
  \edef\tikz@save@line@width{\the\pgflinewidth}%
  \begingroup%
  \let\tikz@fig@name=\pgfutil@empty%
    \begingroup%
      \tikz@is@matrixfalse%
      \let\nodepart=\tikz@nodepart%
      \let\tikz@atbegin@scope=\pgfutil@empty%
      \let\tikz@atend@scope=\pgfutil@empty%
      \let\tikz@do@after@node=\tikz@scan@next@command%
      \let\tikz@options=\pgfutil@empty%
      \tikz@clear@rdf@options%
      \let\tikz@id@name=\pgfutil@empty%
      \let\tikz@after@path=\pgfutil@empty%
      \let\tikz@transform=\pgfutil@empty%
      \let\tikz@mode=\pgfutil@empty%
      \tikz@decoratepathfalse%
      \let\tikz@preactions=\pgfutil@empty%
      \let\tikz@postactions=\pgfutil@empty%
      \let\tikz@alias=\pgfutil@empty%
      \def\tikz@node@at{\pgfqpoint{\the\tikz@lastx}{\the\tikz@lasty}}%
      \let\tikz@time@for@matrix\tikz@time%
      \let\tikz@node@content\relax%
      \pgfgetpath\tikzpathuptonow%
      \iftikz@node@is@a@label%
      \else%
        \let\tikz@time\pgfutil@empty%
      \fi%
      \tikz@node@reset@hook%
      \tikzset{every node/.try}%
      \tikz@@scan@fig}%
\def\tikz@@scan@fig{%
  \pgfutil@ifnextchar a{\tikz@fig@scan@at}
  {\pgfutil@ifnextchar({\tikz@fig@scan@name}
    {\pgfutil@ifnextchar[{\tikz@fig@scan@options}%
      {\pgfutil@ifnextchar:{\tikz@fig@scan@animation}%
        {\pgfutil@ifnextchar\bgroup{\tikz@fig@main}%
          {\tikzerror{A node must have a (possibly empty) label text}%
            \tikz@fig@main{}}}}}}}%}}%
\def\tikz@fig@scan@at at{%
  \tikz@scan@one@point\tikz@@fig@scan@at}%
\def\tikz@@fig@scan@at#1{%
  \def\tikz@node@at{#1}\tikz@@scan@fig}%
\def\tikz@fig@scan@name(#1){%
    \pgfkeysvalueof{/tikz/name/.@cmd}#1\pgfeov% CF : this is now ALWAYS consistent with 'name=' option; allows overrides.
    \tikz@@scan@fig}%
% make it \long to allow \par in "pin" options etc:
\long\def\tikz@fig@scan@options[#1]{\iftikz@node@is@pic\tikz@enable@pic@quotes\else\tikz@enable@node@quotes\fi\tikzset{#1}\ifx\tikz@node@content\relax\expandafter\tikz@@scan@fig\else\tikz@expand@node@contents\fi}%
\def\tikz@fig@scan@animation:#1=#2{\tikzset{animate={myself:{#1}={#2}}}\tikz@@scan@fig}%
\def\tikz@expand@node@contents{%
  \expandafter\tikz@@scan@fig\expandafter{\tikz@node@content}%
}%
\let\tikz@node@reset@hook=\pgfutil@empty%
\let\tikz@node@begin@hook=\pgfutil@empty%
\def\tikz@fig@main{%
  \iftikz@node@is@pic%
    \tikz@node@is@picfalse%
    \expandafter\tikz@subpicture@handle%
  \else%
    \afterassignment\tikz@@fig@main\expandafter\let\expandafter\next\expandafter=%
  \fi}%
\def\tikz@@fig@main{%
    \pgfutil@ifundefined{pgf@sh@s@\tikz@shape}%
    {\tikzerror{Unknown shape ``\tikz@shape.'' Using ``rectangle'' instead}%
      \def\tikz@shape{rectangle}}%
    {}%
    \expandafter\xdef\csname tikz@dcl@coord@\tikz@fig@name\endcsname{%
      \csname tikz@scan@point@coordinate\endcsname}%
    \tikzset{every \tikz@shape\space node/.try}%
    \tikz@node@textfont%
    \tikz@node@begin@hook%
    \iftikz@is@matrix%
      \let\tikz@next=\tikz@do@matrix%
    \else%
      \let\tikz@next=\tikz@do@fig%
    \fi%
    \tikz@next%
}%
\let\tikz@nodepart@list\pgfutil@empty
\def\tikz@do@fig{%
    % Ok, reset all node part boxes
    \pgfutil@for\tikz@temp:=\tikz@nodepart@list\do{%
      \expandafter\setbox\csname pgfnodepart\tikz@temp box\endcsname=\box\pgfutil@voidb@x%
    }%
    \setbox\pgfnodeparttextbox=\hbox%
      \bgroup%
        \pgfinterruptpicture%
        \pgfsys@begin@text%
          \pgfsys@text@to@black@hook%
          \tikzset{every text node part/.try}%
          \ifx\tikz@textopacity\pgfutil@empty%
          \else%
            \pgfsetfillopacity{\tikz@textopacity}%
            \pgfsetstrokeopacity{\tikz@textopacity}%
          \fi%
          \ifx\tikz@text@width\pgfutil@empty%
            \tikz@textfont%
          \else%
            \begingroup%
                \pgfmathsetlength{\pgf@x}{\tikz@text@width}%
              \pgfutil@minipage[t]{\pgf@x}\leavevmode\hbox{}%
                \tikz@textfont%
                \tikz@text@action%
          \fi%
          \tikz@atbegin@node%
          \bgroup%
            \aftergroup\unskip%
            % Some color stuff has been moved from here to outside; this is
            % necessary for support of dvisvgm and of animation
            % snapshots.
            \ifx\tikz@textcolor\pgfutil@empty%
            \else%
              \pgfutil@colorlet{.}{\tikz@textcolor}%
              \pgfutil@color{\tikz@textcolor}%
            \fi%
            \setbox\tikz@figbox=\box\pgfutil@voidb@x%
            \setbox\tikz@figbox@bg=\box\pgfutil@voidb@x%
            \tikz@uninstallcommands%
            \iftikz@handle@active@code%
              \tikz@orig@shorthands%
              \let\tikz@orig@shorthands\pgfutil@empty%
            \fi%
            \ifnum\the\catcode`\;=\active\relax\expandafter\let\tikz@activesemicolon=\tikz@origsemi\fi%
            \ifnum\the\catcode`\:=\active\relax\expandafter\let\tikz@activecolon=\tikz@origcolon\fi%
            \ifnum\the\catcode`\|=\active\relax\expandafter\let\tikz@activebar=\tikz@origbar\fi%
            \aftergroup\tikz@fig@collectresetcolor%
            \tikz@signal@halign@check%
            \tikz@text@reset%
            \tikz@halign@check%
            \ignorespaces%
}%
\def\tikz@fig@collectresetcolor{%
  % Hacks for special packages that mess with \aftergroup
  \pgfutil@ifnextchar\reset@color% hack for color package
  {\reset@color\afterassignment\tikz@fig@collectresetcolor\let\tikz@temp=}\tikz@fig@boxdone%
}%
\def\tikz@fig@boxdone{%
            \tikz@atend@node%
          \ifx\tikz@text@width\pgfutil@empty%
          \else%
              \pgfutil@endminipage%
            \endgroup%
          \fi%
        \pgfsys@end@text%
        \endpgfinterruptpicture%
      \egroup%
    \pgfutil@ifnextchar c{\tikz@fig@mustbenamed\tikz@fig@continue}%
    {\pgfutil@ifnextchar[{\tikz@fig@mustbenamed\tikz@fig@continue}%
      {\pgfutil@ifnextchar t{\tikz@fig@mustbenamed\tikz@fig@continue}
        {\pgfutil@ifnextchar e{\tikz@fig@mustbenamed\tikz@fig@continue}
          {\ifx\tikz@after@path\pgfutil@empty\expandafter\tikz@fig@continue\else\expandafter\tikz@fig@mustbenamed\expandafter\tikz@fig@continue\fi}}}}}%}%

\tikzset{
  matrix/inner style/every cell/.code={%
    \tikzset{every cell/.try={\the\pgfmatrixcurrentrow}{\the\pgfmatrixcurrentcolumn}}%
  },
  matrix/inner style/column/.code={%
    \tikzset{column \the\pgfmatrixcurrentcolumn/.try}%
  },
  matrix/inner style/even odd column/.code={
    \ifodd\pgfmatrixcurrentcolumn%
      \tikzset{every odd column/.try}%
    \else%
      \tikzset{every even column/.try}%
    \fi
  },
  matrix/inner style/row/.code={%
    \tikzset{row \the\pgfmatrixcurrentrow/.try}%
  },
  matrix/inner style/even odd row/.code={%
    \ifodd\pgfmatrixcurrentrow%
      \tikzset{every odd row/.try}%
    \else%
      \tikzset{every even row/.try}%
    \fi
  },
  matrix/inner style/cell/.code={%
    \tikzset{row \the\pgfmatrixcurrentrow\space column \the\pgfmatrixcurrentcolumn/.try}%
  },
  matrix/inner style order/.store in=\tikz@common@matrix@code@styleorder,
  matrix/inner style order={
    every cell,
    column,
    even odd column,
    row,
    even odd row,
    cell
  },
}%

\def\tikz@do@matrix{%
    \tikzset{every matrix/.try}%
    \tikz@node@transformations%
    \tikz@fig@mustbenamed%
    \setbox\tikz@whichbox=\hbox\bgroup%
      \unhbox\tikz@whichbox%
      \hbox\bgroup\bgroup%
          \pgfinterruptpath%
            \pgfscope%
              \ifx\tikz@time\pgfutil@empty\let\tikz@time\tikz@time@for@matrix\fi%
              \tikz@options%
              \tikz@do@rdf@pre@options%
              \tikz@is@nodefalse%
              \tikz@call@id@hook%
              \pgfidscope%
                \tikz@do@rdf@post@options%
                \begingroup%
                \let\tikz@id@name\pgfutil@empty%
                \pgfclearid%
                \setbox\tikz@figbox=\box\pgfutil@voidb@x%
                \setbox\tikz@figbox@bg=\box\pgfutil@voidb@x%
                \let\tikzmatrixname=\tikz@fig@name%
                \edef\tikz@m@anchor{\ifx\tikz@matrix@anchor\pgfutil@empty\tikz@anchor\else\tikz@matrix@anchor\fi}%
                \expandafter\pgfutil@in@\expandafter{\expandafter.\expandafter}\expandafter{\tikz@m@anchor}%
                \ifpgfutil@in@%
                  \expandafter\tikz@matrix@split\tikz@m@anchor\relax%
                \else%
                  \def\tikz@matrix@shift{\pgfpointorigin}%
                \fi%
                \let\tikz@transform=\relax%
                \pgfmatrixbeforeassemblenode{\tikzset{every outer matrix/.try}}%
                \pgfmatrix%
                {\tikz@shape}%
                {\tikz@m@anchor}%
                {\tikz@fig@name}%
                {%
                  \pgfutil@tempdima=\pgflinewidth%
                  {\begingroup\tikz@finish}%
                  \global\pgflinewidth=\pgfutil@tempdima%
                }%
                {\tikz@matrix@shift}%
                {%
                  \tikz@matrix@make@active@ampersand%
                  \def\pgfmatrixbegincode{%
                    \pgfsys@beginscope%
                    \tikz@common@matrix@code%
                    \tikz@atbegin@cell%
                  }%
                  \def\tikz@common@matrix@code{%
                    \let\tikz@options=\pgfutil@empty%
                    \let\tikz@mode=\pgfutil@empty%
                    \pgfutil@for\pgf@temp:=\tikz@common@matrix@code@styleorder\do{%
                      \toks0=\expandafter{\romannumeral-`0\expandafter\pgfutil@trimspaces\expandafter{\pgf@temp}}%
                      \def\pgf@marshal{}%
                      \pgfutil@ifxempty\pgf@temp{}{%
                        \edef\pgf@marshal{\noexpand\tikzset{matrix/inner style/.cd,\the\toks0}}%
                      }%
                      \pgf@marshal
                    }%
                    \tikz@options%
                  }%
                  \def\pgfmatrixendcode{%
                    \tikz@atend@cell%
                    \pgfsys@endscope%
                  }%
                  \def\pgfmatrixemptycode{%
                    \pgfsys@beginscope%
                    \tikz@common@matrix@code%
                    \tikz@at@emptycell%
                    \pgfsys@endscope%
                  }%
                  \tikz@atbegin@matrix%
                  \aftergroup\tikz@do@matrix@cont}%
                \bgroup%
}%
\def\tikz@do@matrix@cont{%
                \tikz@atend@matrix%
              \endgroup%
              \endpgfidscope%
            \endpgfscope
          \endpgfinterruptpath%
      \egroup\egroup%
    \egroup%
    %
    \tikz@node@finish%
}%
{%
  \catcode`\&=13
  \gdef\tikz@matrix@make@active@ampersand{%
    \ifx\tikz@ampersand@replacement\pgfutil@empty%
      \catcode`\&=13%
      \let&=\pgfmatrixnextcell%
    \else%
      \expandafter\let\tikz@ampersand@replacement=\pgfmatrixnextcell%
    \fi%
  }%
}%


\def\tikz@matrix@split#1.#2\relax{%
  \def\tikz@m@anchor{text}%
  \def\tikz@matrix@shift{\pgfpointanchor{#1}{#2}}%
}%

\def\tikz@fig@continue{%
    \ifx\tikz@text@width\pgfutil@empty%
    \else%
      \pgfmathsetlength{\pgf@x}{\tikz@text@width}%
      \wd\pgfnodeparttextbox=\pgf@x%
    \fi%
    \ifx\tikz@text@height\pgfutil@empty%
    \else%
      \pgfmathsetlength{\pgf@x}{\tikz@text@height}%
      \ht\pgfnodeparttextbox=\pgf@x%
    \fi%
    \ifx\tikz@text@depth\pgfutil@empty%
    \else%
      \pgfmathsetlength{\pgf@x}{\tikz@text@depth}%
      \dp\pgfnodeparttextbox=\pgf@x%
    \fi%
    %
    % Node transformation
    %
    \tikz@node@transformations%
    \tikz@nlt%
    %
    \setbox\tikz@whichbox=\hbox{%
      \unhbox\tikz@whichbox%
      \hbox{{%
          \pgfinterruptpath%
            \pgfscope%
              \tikz@options%
              \tikz@do@rdf@pre@options%
              \tikz@is@nodetrue%
              \tikz@call@id@hook%
              \pgfidscope%
                \tikz@do@rdf@post@options%
                \let\tikz@id@name\pgfutil@empty%
                \setbox\tikz@figbox=\box\pgfutil@voidb@x%
                \setbox\tikz@figbox@bg=\box\pgfutil@voidb@x%
                % Add color modifications to text box
                \setbox\pgfnodeparttextbox=\hbox{{%
                    \pgfsys@begin@text% Colors moved here...
                    \ifx\tikz@textcolor\pgfutil@empty%
                    \else%
                      \pgfutil@colorlet{.}{\tikz@textcolor}%
                    \fi%
                    \pgfsetcolor{.}%
                    \pgfusetype{.text}%
                    \pgfidscope%
                      \box\pgfnodeparttextbox%
                    \endpgfidscope%
                    \pgfsys@end@text%
                  }}%
                \pgfmultipartnode{\tikz@shape}{\tikz@anchor}{\tikz@fig@name}{%
                  \pgfutil@tempdima=\pgflinewidth%
                  {\begingroup\tikz@finish}%
                  \global\pgflinewidth=\pgfutil@tempdima%
                }%
              \endpgfidscope%
            \endpgfscope%
          \endpgfinterruptpath%
      }}%
    }%
    %
    \tikz@alias%
    \tikz@node@finish%
}%


\def\tikz@fig@mustbenamed{%
  \ifx\tikz@fig@name\pgfutil@empty%
    % Assign a dummy name
    \global\advance\tikz@fig@count by1\relax
    \edef\tikz@fig@name{tikz@f@\the\tikz@fig@count}%
    \let\tikz@id@name\tikz@fig@name%
  \fi%
}%

\def\tikz@node@transformations{%
  %
  % Possibly, we are ``online''
  %
  \ifx\tikz@time\pgfutil@empty%
    \pgftransformshift{\tikz@node@at}%
    \iftikz@fullytransformed%
    \else%
      \pgftransformresetnontranslations%
    \fi%
  \else%
    \tikz@do@auto@anchor%
    \tikz@timer%
  \fi%
  % Invoke local transformations
  \tikz@transform%
}%

\def\tikz@node@finish{%
    \global\let\tikz@last@fig@name=\tikz@fig@name%
    \global\let\tikz@after@path@smuggle=\tikz@after@path%
    % shift box outside group
    \global\setbox\tikz@tempbox=\box\tikz@figbox%
    \global\setbox\tikz@tempbox@bg=\box\tikz@figbox@bg%
  \endgroup\endgroup%
  \setbox\tikz@figbox=\box\tikz@tempbox%
  \setbox\tikz@figbox@bg=\box\tikz@tempbox@bg%
  \global\pgflinewidth=\tikz@save@line@width%
  \tikz@do@after@path@smuggle%
  \tikz@node@is@picfalse
  \tikz@do@after@node%
}%
\let\tikz@fig@continue@orig=\tikz@fig@continue

\def\tikz@do@after@node{\tikz@scan@next@command}%

\def\tikz@do@after@path@smuggle{%
  \let\tikz@to@last@fig@name=\tikz@last@fig@name%
  \let\tikz@to@use@whom=\tikz@to@use@last@fig@name%
  \let\tikzlastnode=\tikz@last@fig@name%
  \ifx\tikz@after@path@smuggle\pgfutil@empty%
  \else%
    \ifpgflatenodepositioning%
      \expandafter\expandafter\expandafter\tikz@call@late%
      \expandafter\expandafter\expandafter{\expandafter\tikz@last@fig@name\expandafter}\expandafter{\tikz@after@path@smuggle}%
    \else%
      \tikz@scan@next@command{\tikz@after@path@smuggle}\pgf@stop%
    \fi%
  \fi%
}%

\def\tikz@call@late#1#2{\pgfnodepostsetupcode{#1}{\path[late options={name={#1},append after command={#2}}];}}%

\newif\iftikz@do@align

% Alignment handling
\def\tikz@signal@halign@check{%
  \tikz@do@alignfalse
  \ifx\tikz@text@width\pgfutil@empty%
    \pgfkeysgetvalue{/tikz/node halign header}\tikz@align@header%
    \ifx\tikz@align@header\pgfutil@empty%
    \else%
      \tikz@do@aligntrue%
    \fi%
  \fi%
}
\def\tikz@halign@check{%
  \iftikz@do@align%
    % Bingo
    \setbox\tikz@align@aligned@box=\box\pgfutil@voidb@x% void
    \let\\=\tikz@align@newline%
    \expandafter\tikz@start@align%
  \fi%
}%
\def\tikz@align@newline{\pgfutil@protect\tikz@align@newline@}%
\def\tikz@align@newline@{\unskip\pgfutil@ifnextchar[\tikz@@align@newline{\tikz@@align@newline[0pt]}}%}%
\def\tikz@@align@newline[#1]{\egroup\tikz@align@continue\pgfmathparse{#1}\let\tikz@align@temp=\pgfmathresult\tikz@start@align}%
% Two safe boxes for alignment:
\let\tikz@align@aligned@box=\pgfnodeparttextbox
\let\tikz@align@line@box=\tikz@figbox

\def\tikz@start@align{%
  % Start collecting text:
  \setbox\tikz@align@line@box=\hbox\bgroup\bgroup%
    \aftergroup\tikz@align@collectresetcolor\ignorespaces%
}%
\def\tikz@align@collectresetcolor{%
  \pgfutil@ifnextchar\reset@color%
  {\reset@color\afterassignment\tikz@align@collectresetcolor\let\tikz@temp=}%
  {\tikz@align@end@check}%
}%
\def\tikz@align@end@check{%
  \egroup%
  \ifvoid\tikz@align@aligned@box%
    \setbox\tikz@align@aligned@box=\box\tikz@align@line@box%
  \else%
    \setbox\tikz@align@aligned@box=\vbox{%
      \expandafter\expandafter\expandafter\halign\expandafter\expandafter\expandafter{\tikz@align@header%
        \box\tikz@align@aligned@box\cr%
        \noalign{\vskip\tikz@align@temp pt}%
        \unhbox\tikz@align@line@box\unskip\cr}}%
  \fi%
  \pgfutil@ifnextchar\tikz@align@continue{}
  {%
    % Main continue
      \box\tikz@align@aligned@box%
    \egroup%
  }%
}%
\def\tikz@align@continue{\tikz@@align@continue}%
\let\tikz@@align@continue=\pgfutil@empty


\def\tikz@node@also lso{\pgfutil@ifnextchar[\tikz@node@also@opt{\tikz@node@also@opt[]}}%
\def\tikz@node@also@opt[#1]{
  \pgfutil@ifnextchar(%)
  {\tikz@node@also@opt@cont[#1]}%
  {\tikzerror{Syntax error in node also: ``('' expected.}%
    \tikz@scan@next@command}%
}%
\def\tikz@node@also@opt@cont[#1](#2){\tikzset{late options={name=#2,#1}}\tikz@scan@next@command}%



% Syntax for parts of  nodes:
% node ... {... \nodepart[options]{name} ... \nodepart{name} ...}

\def\tikz@nodepart{\pgfutil@ifnextchar[\tikz@@nodepart{\tikz@@nodepart[]}}%}%
\def\tikz@@nodepart[#1]#2{%
  \tikz@atend@node%
  \unskip%
  \gdef\tikz@nodepart@options{#1}%
  \gdef\tikz@nodepart@name{#2}%
  \global\let\tikz@fig@continue=\tikz@nodepart@continue%
  \pgfutil@ifnextchar x{\egroup\relax}{\egroup\relax}% gobble spaces
}%
\def\tikz@nodepart@continue{%
  \global\let\tikz@fig@continue=\tikz@fig@continue@orig%
  \ifx\tikz@nodepart@list\pgfutil@empty%
    \let\tikz@nodepart@list\tikz@nodepart@name%
  \else%
    \edef\tikz@nodepart@list{\tikz@nodepart@list,\tikz@nodepart@name}%
  \fi%
  % Now start new box:
   \expandafter\setbox\csname pgfnodepart\tikz@nodepart@name box\endcsname=\hbox%
      \bgroup%
        \pgfinterruptpicture%
        \pgfsys@begin@text%
        \pgfsys@text@to@black@hook%
        \tikzset{every \tikz@nodepart@name\space node part/.try}%
        \expandafter\tikzset\expandafter{\tikz@nodepart@options}%
        \ifx\tikz@textopacity\pgfutil@empty%
        \else%
          \pgfsetfillopacity{\tikz@textopacity}%
          \pgfsetstrokeopacity{\tikz@textopacity}%
        \fi%
        % Colors moved here...
        \ifx\tikz@textcolor\pgfutil@empty%
        \else%
          \pgfutil@colorlet{.}{\tikz@textcolor}%
        \fi%
        \pgfsetcolor{.}%
          \ifx\tikz@text@width\pgfutil@empty%
            \tikz@textfont%
          \else%
            \begingroup%
              \pgfmathsetlength{\pgf@x}{\tikz@text@width}%
              \pgfutil@minipage[t]{\pgf@x}\leavevmode\hbox{}%
                \tikz@textfont%
                \tikz@text@action%
          \fi%
          \bgroup%
            \aftergroup\unskip%
            \setbox\tikz@figbox=\box\pgfutil@voidb@x%
            \setbox\tikz@figbox@bg=\box\pgfutil@voidb@x%
            \tikz@uninstallcommands%
            \iftikz@handle@active@code%
              \tikz@orig@shorthands%
              \let\tikz@orig@shorthands\pgfutil@empty%
            \fi%
            \ifnum\the\catcode`\;=\active\relax\expandafter\let\tikz@activesemicolon=\tikz@origsemi\fi%
            \ifnum\the\catcode`\:=\active\relax\expandafter\let\tikz@activecolon=\tikz@origcolon\fi%
            \ifnum\the\catcode`\|=\active\relax\expandafter\let\tikz@activebar=\tikz@origbar\fi%
            \tikz@atbegin@node%
            \aftergroup\tikz@fig@collectresetcolor%
            \tikz@signal@halign@check%
            \tikz@text@reset%
            \tikz@halign@check%
            \ignorespaces%
}%


%
% Node foreach
%

\def\tikz@nodes@start{%
  \let\tikz@nodes@list\pgfutil@empty%
  \iftikz@node@is@pic%
    \def\tikz@nodes@collect{pic }%
  \else%
    \def\tikz@nodes@collect{node }%
  \fi%
  \tikz@nodes%
}%
\def\tikz@nodes foreach{\pgfutil@ifnextchar x\tikz@nodes@\tikz@nodes@}% get rid of spaces
\def\tikz@nodes@#1in{%
  \expandafter\def\expandafter\tikz@nodes@list\expandafter{\tikz@nodes@list\foreach#1in}%
  \pgfutil@ifnextchar\bgroup\tikz@nodes@group\tikz@nodes@one%
}%
\def\tikz@nodes@one#1{%
  \expandafter\def\expandafter\tikz@nodes@list\expandafter{\tikz@nodes@list#1}%
  \pgfutil@ifnextchar f\tikz@nodes\tikz@nodes@scan%
}%
\def\tikz@nodes@group#1{%
  \expandafter\def\expandafter\tikz@nodes@list\expandafter{\tikz@nodes@list{#1}}%
  \pgfutil@ifnextchar f\tikz@nodes\tikz@nodes@scan%
}%
\def\tikz@nodes@scan{%
  \pgfutil@ifnextchar a{\tikz@nodes@at}%
  {\pgfutil@ifnextchar({\tikz@nodes@name}%
    {\pgfutil@ifnextchar[{\tikz@nodes@opt}%
      {\pgfutil@ifnextchar\bgroup{\tikz@nodes@main}%
      {\tikzerror{Nodes must have a (possibly empty) label text}%
        \tikz@fig@main{}}}}}}%}}%

% Look ahead whether the next character is a (.  If that is the case, we scan
% until ), otherwise we grab a single token and append.
\def\tikz@nodes@at at{\pgfutil@ifnextchar({\tikz@nodes@at@}{\tikz@nodes@at@@}}%
\def\tikz@nodes@at@#1){%
  \expandafter\def\expandafter\tikz@nodes@collect\expandafter{\tikz@nodes@collect at#1)}%
  \tikz@nodes@scan}%
\def\tikz@nodes@at@@#1{%
  \expandafter\def\expandafter\tikz@nodes@collect\expandafter{\tikz@nodes@collect at#1}%
  \tikz@nodes@scan}%

\def\tikz@nodes@name#1){%
  \expandafter\def\expandafter\tikz@nodes@collect\expandafter{\tikz@nodes@collect#1)}%
  \tikz@nodes@scan}%
\def\tikz@nodes@opt#1]{%
  \expandafter\def\expandafter\tikz@nodes@collect\expandafter{\tikz@nodes@collect#1]}%
  \tikz@nodes@scan}%
\def\tikz@nodes@main#1{%
  \iftikz@handle@active@nodes%
    \iftikz@node@is@pic%
      \expandafter\def\expandafter\tikz@nodes@collect\expandafter{\tikz@nodes@collect{#1}}%
    \else%
      \expandafter\def\expandafter\tikz@nodes@collect\expandafter{\tikz@nodes@collect{\scantokens{#1}}}%
    \fi%
  \else%
    \expandafter\def\expandafter\tikz@nodes@collect\expandafter{\tikz@nodes@collect{#1}}%
  \fi%
  % Ok, got everything.
  % Now, start building parse text.
  \global\setbox\tikz@tempbox=\box\tikz@figbox%
  \global\setbox\tikz@tempbox@bg=\box\tikz@figbox@bg%
  \tikz@nodes@list{%
    \setbox\tikz@figbox=\box\tikz@tempbox%
    \setbox\tikz@figbox@bg=\box\tikz@tempbox@bg%
    \expandafter\tikz@scan@next@command\tikz@nodes@collect\pgfextra\relax%
    \global\setbox\tikz@tempbox=\box\tikz@figbox%
    \global\setbox\tikz@tempbox@bg=\box\tikz@figbox@bg%
  }%
  \setbox\tikz@figbox=\box\tikz@tempbox%
  \setbox\tikz@figbox@bg=\box\tikz@tempbox@bg%
  \tikz@scan@next@command%
}%



%
% "late" options can be used to "redo" a node
%
\tikzset{late options/.code=\tikz@late@options{#1}}%
\def\tikz@late@options#1{%
  % Do a "virtual" node:
  \begingroup%
    \iftikz@shapeborder%
      \let\tikz@fig@name=\tikz@shapeborder@name%
    \else%
      \let\tikz@fig@name=\pgfutil@empty%
    \fi%
    \tikz@is@matrixfalse%
    \let\tikz@options=\pgfutil@empty%
    \tikz@clear@rdf@options%
    \let\tikz@after@path=\pgfutil@empty%
    \let\tikz@afternodepathoptions=\pgfutil@empty%
    \let\tikz@alias=\pgfutil@empty%
    \let\tikz@transform=\pgfutil@empty%
    \tikz@decoratepathfalse%
    \tikz@node@reset@hook%
    \tikz@enable@node@quotes%
    \tikzset{every node/.try,#1}%
    \ifx\tikz@fig@name\pgfutil@empty%
      \tikzerror{Late options must reference some existing node}%
    \fi%
    \tikz@node@begin@hook%
    \tikz@alias%
    \tikzgdlatenodeoptionacallback{\tikz@fig@name}%
    \global\let\tikz@last@fig@name=\tikz@fig@name%
    \global\let\tikz@after@path@smuggle=\tikz@after@path%
  \endgroup%
  \tikz@do@after@path@smuggle%
}%


% Auto placement

\def\tikz@auto@pre{%
  \begingroup
    \pgfresetnontranslationattimefalse
    \ifpgfslopedattime
      \pgfslopedattimefalse%
    \else
      \pgfslopedattimetrue%
    \fi
    \pgfallowupsidedownattimetrue%
    \tikz@timer%
    \pgf@x=\pgf@pt@aa pt%
    \pgf@y=\pgf@pt@ab pt%
    \pgfpointnormalised{}%
}%

\def\tikz@auto@post{%
    \global\let\tikz@anchor@smuggle=\tikz@anchor%
  \endgroup%
  \let\tikz@anchor=\tikz@anchor@smuggle%
}%

\def\tikz@auto@anchor{%
    \ifdim\pgf@x>0.05pt%
      \ifdim\pgf@y>0.05pt%
        \def\tikz@anchor{south east}%
      \else\ifdim\pgf@y<-0.05pt%
        \def\tikz@anchor{south west}%
      \else
        \def\tikz@anchor{south}%
      \fi\fi%
    \else\ifdim\pgf@x<-0.05pt%
      \ifdim\pgf@y>0.05pt%
        \def\tikz@anchor{north east}%
      \else\ifdim\pgf@y<-0.05pt%
        \def\tikz@anchor{north west}%
      \else
        \def\tikz@anchor{north}%
      \fi\fi%
    \else%
      \ifdim\pgf@y>0pt%
        \def\tikz@anchor{east}%
      \else%
        \def\tikz@anchor{west}%
      \fi%
    \fi\fi%
}%

\def\tikz@auto@anchor@prime{%
    \ifdim\pgf@x>0.05pt%
      \ifdim\pgf@y>0.05pt%
        \def\tikz@anchor{north west}%
      \else\ifdim\pgf@y<-0.05pt%
        \def\tikz@anchor{north east}%
      \else
        \def\tikz@anchor{north}%
      \fi\fi%
    \else\ifdim\pgf@x<-0.05pt%
      \ifdim\pgf@y>0.05pt%
        \def\tikz@anchor{south west}%
      \else\ifdim\pgf@y<-0.05pt%
        \def\tikz@anchor{south east}%
      \else
        \def\tikz@anchor{south}%
      \fi\fi%
    \else%
      \ifdim\pgf@y>0pt%
        \def\tikz@anchor{west}%
      \else%
        \def\tikz@anchor{east}%
      \fi%
    \fi\fi%
}%


%
% Callbacks: Please see the documentation of the graph drawing
% lib for info on these callbacks
%
\def\tikzgdeventcallback#1#2{}%
\def\tikzgdeventgroupcallback#1{}%
\def\tikzgdlatenodeoptionacallback#1{}%

% Syntax for trees:
% node {...} child [options] {...} child [options] {...} ...
% node {...} child [options] foreach \var in {list} [options] {...} ...

\def\tikz@children{%
  % Start collecting the children:
  \let\tikz@children@list=\pgfutil@empty%
  \tikznumberofchildren=0\relax%
  \tikz@collect@children c}%

\def\tikz@collect@children{\pgfutil@ifnextchar c{\tikz@collect@children@cchar}{\tikz@children@collected}}%
\def\tikz@collect@children@cchar c{\pgfutil@ifnextchar h{\tikz@collect@child}{\tikz@children@collected c}}%
\def\tikz@collect@child hild{\pgfutil@ifnextchar[{\tikz@collect@childA}{\tikz@collect@childA[]}}%}%
\def\tikz@collect@childA[#1]{\pgfutil@ifnextchar f{\tikz@collect@children@foreach[#1]}{\tikz@collect@childB[#1]}}%
\def\tikz@collect@childB[#1]{%
  \advance\tikznumberofchildren by1\relax
  \expandafter\def\expandafter\tikz@children@list\expandafter{\tikz@children@list \tikz@childnode[#1]}%
  \pgfutil@ifnextchar\bgroup{\tikz@collect@child@code}{\tikz@collect@child@code{}}}%
\def\tikz@collect@child@code#1{%
  \expandafter\def\expandafter\tikz@children@list\expandafter{\tikz@children@list{#1}}%
  \tikz@collect@children%
}%
\def\tikz@collect@children@foreach[#1]foreach#2in#3{%
  \pgfutil@ifnextchar\bgroup{\tikz@collect@children@foreachA{#1}{#2}{#3}}{\tikz@collect@children@foreachA{#1}{#2}{#3}{}}}%
\def\tikz@collect@children@foreachA#1#2#3#4{%
  \expandafter\def\expandafter\tikz@children@list\expandafter
    {\tikz@children@list\tikz@childrennodes[#1]{#2}{#3}{#4}}%
  \c@pgf@counta=\tikznumberofchildren%
  \foreach#2in{#3}%
  {%
    \global\advance\c@pgf@counta by1\relax%
  }%
  \tikznumberofchildren=\c@pgf@counta%
  \tikz@collect@children%
}%
\long\def\tikz@children@collected{%
  \begingroup%
    \advance\tikztreelevel by 1\relax%
    \tikzgdeventgroupcallback{descendants}%
    \let\tikz@options=\pgfutil@empty%
    \tikz@clear@rdf@options%
    \let\tikz@transform=\pgfutil@empty%
    \tikzset{level/.try=\the\tikztreelevel,level \the\tikztreelevel/.try}%
    \tikz@transform%
    \let\tikz@transform=\relax%
    \let\tikzparentnode=\tikz@last@fig@name%
    \ifx\tikz@grow\relax\else%
      % Transform to center of node
      \pgftransformshift{\pgfpointanchor{\tikzparentnode}{\tikz@growth@anchor}}%
    \fi%
    \tikznumberofcurrentchild=0\relax%
    \tikz@children@list%
    \global\setbox\tikz@tempbox=\box\tikz@figbox%
    \global\setbox\tikz@tempbox@bg=\box\tikz@figbox@bg%
  \endgroup%
  \setbox\tikz@figbox=\box\tikz@tempbox%
  \setbox\tikz@figbox@bg=\box\tikz@tempbox@bg%
  \tikz@scan@next@command%
}%

% Syntax for children:
%
% child [all children options] foreach \var in {values} [child options] {...}
\def\tikz@childrennodes[#1]#2#3#4{%
  \c@pgf@counta=\tikznumberofcurrentchild\relax%
  \setbox\tikz@tempbox=\box\tikz@figbox%
  \setbox\tikz@tempbox@bg=\box\tikz@figbox@bg%
  \foreach#2in{#3}{%
    \tikznumberofcurrentchild=\c@pgf@counta\relax%
    \setbox\tikz@figbox=\box\tikz@tempbox%
    \setbox\tikz@figbox@bg=\box\tikz@tempbox@bg%
    \tikz@childnode[#1]{#4}%
    % we must now make the current child number and the figbox survive
    % the group
    \global\c@pgf@counta=\tikznumberofcurrentchild\relax%
    \global\setbox\tikz@tempbox=\box\tikz@figbox%
    \global\setbox\tikz@tempbox@bg=\box\tikz@figbox@bg%
  }%
  \tikznumberofcurrentchild=\c@pgf@counta\relax%
  \setbox\tikz@figbox=\box\tikz@tempbox%
  \setbox\tikz@figbox@bg=\box\tikz@tempbox@bg%
}%


% Syntax for child:
%
% child
%
% child[options]
%
% child[options] {node (name) {child node text} ...
%   edge from parent[options] node {label text} node {label text}}

\def\tikz@childnode[#1]#2{%
  \advance\tikznumberofcurrentchild by1\relax%
  {\tikzset{every child/.try,#1}\expandafter}%
  \iftikz@child@missing%
    \tikzgdeventcallback{node}{}%
  \else%
  \setbox\tikz@whichbox=\hbox\bgroup%
    \unhbox\tikz@whichbox%
    \hbox\bgroup\bgroup%
        \pgfinterruptpath%
          \pgfscope%
            \let\tikz@transform=\pgfutil@empty%
            \tikzset{every child/.try,#1}%
            \tikz@options%
            \tikz@transform%
            \let\tikz@transform=\relax%
            \tikz@grow%
            % Typeset node:
            \edef\tikz@parent@node@name{[name=\tikzparentnode-\the\tikznumberofcurrentchild,style=every child node]}%
            \def\tikz@child@node@text{[shape=coordinate]{}}
            \tikz@parse@child@node#2\pgf@stop%
            \expandafter\expandafter\expandafter\node
            \expandafter\tikz@parent@node@name
              \tikz@child@node@text
              \pgfextra{\global\let\tikz@childnode@name=\tikz@last@fig@name};%
            \let\tikzchildnode=\tikz@childnode@name%
            {%
              \def\tikz@edge@to@parent@needed{edge from parent}
              \ifx\tikz@child@node@rest\pgfutil@empty%
                \path edge from parent;%
              \else%
                \path \tikz@child@node@rest \tikz@edge@to@parent@needed;%
              \fi%
            }%
        \endpgfscope%
      \endpgfinterruptpath%
    \egroup\egroup%
  \egroup%
  \fi%
}%

\def\tikz@parse@child@node{%
  \pgfutil@ifnextchar n{\tikz@parse@child@node@n}%
  {\pgfutil@ifnextchar c{\tikz@parse@child@node@c}%
    {\pgfutil@ifnextchar\pgf@stop\tikz@parse@child@node@rest\tikz@parse@child@node@expand}}}%
\def\tikz@parse@child@node@expand{%
  \advance\tikz@expandcount by-1\relax%
  \ifnum\tikz@expandcount<0\relax%
    \expandafter\tikz@parse@child@node@rest%
  \else%
    \expandafter\expandafter\expandafter\tikz@parse@child@node%
  \fi%
}%
\def\tikz@parse@child@node@rest#1\pgf@stop{\tikz@resetexpandcount\def\tikz@child@node@rest{#1}}%
\def\tikz@parse@child@node@c c{\tikz@resetexpandcount\pgfutil@ifnextchar o{\tikz@parse@child@node@co}{\tikz@parse@child@node@rest c}}%
\def\tikz@parse@child@node@co o{\pgfutil@ifnextchar o{\tikz@parse@child@node@coordinate}{\tikz@parse@child@node@rest co}}%
\def\tikz@parse@child@node@coordinate ordinate{%
  \pgfutil@ifnextchar ({\tikz@@parse@child@node@coordinate}{%
    \def\tikz@child@node@text{[shape=coordinate]{}}%
    \tikz@parse@child@node@rest}}%}%
\def\tikz@@parse@child@node@coordinate(#1){%
  \pgfutil@ifnextchar a{\tikz@p@c@n@c@at(#1)}{%
    \def\tikz@child@node@text{[shape=coordinate,name=#1]{}}%
    \tikz@parse@child@node@rest}}%
\def\tikz@p@c@n@c@at(#1)at#2({%
  \def\tikz@child@node@text@pre{[shape=coordinate,name=#1]at}%
  \tikz@scan@one@point\tikz@p@c@n@c@at@math(%
}%
\def\tikz@p@c@n@c@at@math#1{%
  \pgf@process{#1}%
  \edef\tikz@marshal{(\the\pgf@x,\the\pgf@y){}}%
  \expandafter\expandafter\expandafter\def%
  \expandafter\expandafter\expandafter\tikz@child@node@text%
  \expandafter\expandafter\expandafter{\expandafter\tikz@child@node@text@pre\tikz@marshal}%
  \tikz@parse@child@node@rest%
}%
\def\tikz@parse@child@node@n node{\tikz@resetexpandcount%
  \let\tikz@child@node@text=\pgfutil@empty%
  \tikz@p@c@s}%
\def\tikz@p@c@s}%
\def\tikz@p@c@s@at at#1({%
  \tikz@scan@one@point\tikz@p@c@s@at@math(%
}%
\def\tikz@p@c@s@at@math#1{%
  \pgf@process{#1}%
  \edef\tikz@marshal{ at(\the\pgf@x,\the\pgf@y)}%
  \expandafter\expandafter\expandafter\def%
  \expandafter\expandafter\expandafter\tikz@child@node@text%
  \expandafter\expandafter\expandafter{\expandafter\tikz@child@node@text\tikz@marshal}
  \tikz@p@c@s}%
\def\tikz@p@c@s@paran(#1){%
  \expandafter\def\expandafter\tikz@child@node@text\expandafter{\tikz@child@node@text(#1)}
  \tikz@p@c@s}%
\def\tikz@p@c@s@bra[#1]{%
  \expandafter\def\expandafter\tikz@child@node@text\expandafter{\tikz@child@node@text[#1]}
  \tikz@p@c@s}%
\def\tikz@p@c@s@group#1{%
  \iftikz@handle@active@nodes%
    \expandafter\def\expandafter\tikz@child@node@text\expandafter{\tikz@child@node@text{\scantokens{#1}}}%
  \else%
    \expandafter\def\expandafter\tikz@child@node@text\expandafter{\tikz@child@node@text{#1}}
  \fi%
  \tikz@parse@child@node@rest%
}%



%
% Syntax for decorated subpaths:
%
% decorate [option] { subpath }
%
\def\tikz@decoration ecorate{%
  \pgfutil@ifnextchar[{\tikz@lib@decoration}{\tikz@lib@decoration[]}%]
}%

\def\tikz@lib@decoration[#1]#2{\tikzerror{You need to load a decoration library}}%

% The decorate path command:
\def\tikz@lib@dec@decorate@path{\tikzerror{You need to load a decoration library}}%



%
% Syntax for let :
%
% let \p1 = (coordinate), \p2 = (coordinate),... in
%
\def\tikz@let@command et#1in{%
  \tikzerror{You need to say \string\usetikzlibrary{calc} to use the let command}%
  \tikz@scan@next@command%
}%


%
% Syntax for pictures:
%
% as for nodes, but with "pic" instead of "node"
%
\newif\iftikz@node@is@pic
\def\tikz@subpicture ic{\tikz@node@is@pictrue\tikz@scan@next@command node}%
\def\tikz@subpicture@handle#1{%
  \pgfkeys@spdef\tikz@temp{#1}%
  \expandafter\tikz@subpicture@handle@\expandafter{\tikz@temp}%
}%
\def\tikz@subpicture@handle@#1{
  \pgfkeys{/tikz/pics/.cd,#1}%
  \tikz@node@transformations%
  \let\tikz@transform=\relax%
  \let\tikz@picmode\tikz@mode%
  \tikzset{name prefix ../.style/.expanded={/tikz/name prefix=\pgfkeysvalueof{/tikz/name prefix}}}%
  \ifx\tikz@fig@name\pgfutil@empty\else%
    \tikzset{name prefix/.expanded=\tikz@fig@name}%
  \fi%
  \pgfkeysvalueof{/tikz/pics/setup code}%
  \pgfkeysgetvalue{/tikz/pics/code}{\tikz@pic@code}
  \ifx\tikz@pic@code\pgfutil@empty\else%
  \setbox\tikz@whichbox=\hbox\bgroup%
    \unhbox\tikz@whichbox%
      \hbox\bgroup
        \bgroup%
          \pgfinterruptpath%
            \pgfscope%
              \tikz@options%
              \setbox\tikz@figbox=\box\pgfutil@voidb@x%
              \setbox\tikz@figbox@bg=\box\pgfutil@voidb@x%
              \tikz@atbegin@scope%
              \scope[every pic/.try]%
                \tikz@pic@code%
              \endscope%
              \tikz@atend@scope%
            \endpgfscope%
          \endpgfinterruptpath%
        \egroup
      \egroup%
    \egroup%
  \fi%
  \pgfkeysgetvalue{/tikz/pics/foreground code}{\tikz@pic@code}
  \ifx\tikz@pic@code\pgfutil@empty\else%
  \setbox\tikz@figbox=\hbox\bgroup%
    \unhbox\tikz@figbox%
      \hbox\bgroup
        \bgroup%
          \pgfinterruptpath%
            \pgfscope%
              \tikz@options%
              \setbox\tikz@figbox=\box\pgfutil@voidb@x%
              \setbox\tikz@figbox@bg=\box\pgfutil@voidb@x%
              \tikz@atbegin@scope%
              \scope[every front pic/.try]%
                \tikz@pic@code%
              \endscope%
              \tikz@atend@scope%
            \endpgfscope%
          \endpgfinterruptpath%
        \egroup
      \egroup%
    \egroup%
  \fi%
  \pgfkeysgetvalue{/tikz/pics/background code}{\tikz@pic@code}
  \ifx\tikz@pic@code\pgfutil@empty\else%
  \setbox\tikz@figbox@bg=\hbox\bgroup%
    \unhbox\tikz@figbox@bg%
      \hbox\bgroup
        \bgroup%
          \pgfinterruptpath%
            \pgfscope%
              \tikz@options%
              \setbox\tikz@figbox=\box\pgfutil@voidb@x%
              \setbox\tikz@figbox@bg=\box\pgfutil@voidb@x%
              \tikz@atbegin@scope%
              \scope[every behind pic/.try]%
                \tikz@pic@code%
              \endscope%
              \tikz@atend@scope%
            \endpgfscope%
          \endpgfinterruptpath%
        \egroup
      \egroup%
    \egroup%
  \fi%
  \tikz@node@finish%
}%
\tikzset{
  pic actions/.code=\tikz@addmode{\tikz@picmode}
}%

% Setting up the picture codes:
\tikzset{
  pics/setup code/.initial=,
  pics/code/.initial=,
  pics/background code/.initial=,
  pics/foreground code/.initial=
}%

% Defining pictures:

\def\tikzdeclarepic#1#2{\pgfkeys{/tikz/#1/.cd,#2}}%

\pgfkeysdef{/handlers/.pic}{%
  \edef\pgf@temp{\pgfkeyscurrentpath}%
  \edef\pgf@temp{\expandafter\tikz@smuggle@pics@in\pgf@temp\pgf@stop}%
  \expandafter\pgfkeys\expandafter{\pgf@temp/.style={code={#1}}}%
}%
\def\tikz@smuggle@pics@in/tikz/#1\pgf@stop{/tikz/pics/#1}%

%
% Timers
%

\def\tikz@timer@line{%
  \pgftransformlineattime{\tikz@time}{\tikz@timer@start}{\tikz@timer@end}%
}%

\def\tikz@timer@vhline{%
  \ifdim\tikz@time pt<0.5pt% first half
    \pgf@process{\tikz@timer@start}%
    \pgf@xa=\pgf@x%
    \pgf@ya=\pgf@y%
    \pgf@process{\tikz@timer@end}%
    \pgf@xb=\tikz@time pt%
    \pgf@xb=2\pgf@xb%
    \edef\tikz@marshal{\noexpand\pgftransformlineattime{\pgf@sys@tonumber{\pgf@xb}}{\noexpand\tikz@timer@start}{%
        \noexpand\pgfqpoint{\the\pgf@xa}{\the\pgf@y}}}%
    \tikz@marshal%
  \else% second half
    \pgf@process{\tikz@timer@start}%
    \pgf@xa=\pgf@x%
    \pgf@ya=\pgf@y%
    \pgf@process{\tikz@timer@end}%
    \pgf@xb=\tikz@time pt%
    \pgf@xb=2\pgf@xb%
    \advance\pgf@xb by-1pt%
    \edef\tikz@marshal{\noexpand\pgftransformlineattime{\pgf@sys@tonumber{\pgf@xb}}%
      {\noexpand\pgfqpoint{\the\pgf@xa}{\the\pgf@y}}{\noexpand\tikz@timer@end}}%
    \tikz@marshal%
  \fi%
}%

\def\tikz@timer@hvline{%
  \ifdim\tikz@time pt<0.5pt% first half
    \pgf@process{\tikz@timer@start}%
    \pgf@xa=\pgf@x%
    \pgf@ya=\pgf@y%
    \pgf@process{\tikz@timer@end}%
    \pgf@xb=\tikz@time pt%
    \pgf@xb=2\pgf@xb%
    \edef\tikz@marshal{\noexpand\pgftransformlineattime{\pgf@sys@tonumber{\pgf@xb}}{\noexpand\tikz@timer@start}{%
        \noexpand\pgfqpoint{\the\pgf@x}{\the\pgf@ya}}}%
    \tikz@marshal%
  \else% second half
    \pgf@process{\tikz@timer@start}%
    \pgf@xa=\pgf@x%
    \pgf@ya=\pgf@y%
    \pgf@process{\tikz@timer@end}%
    \pgf@xb=\tikz@time pt%
    \pgf@xb=2\pgf@xb%
    \advance\pgf@xb by-1pt%
    \edef\tikz@marshal{\noexpand\pgftransformlineattime{\pgf@sys@tonumber{\pgf@xb}}%
      {\noexpand\pgfqpoint{\the\pgf@x}{\the\pgf@ya}}{\noexpand\tikz@timer@end}}%
    \tikz@marshal%
  \fi%
}%

\def\tikz@timer@curve{%
  \pgftransformcurveattime{\tikz@time}{\tikz@timer@start}{\tikz@timer@cont@one}{\tikz@timer@cont@two}{\tikz@timer@end}%
}%


\def\tikz@timer@arc{%
  \pgfmathcos@{\tikz@timer@start@angle}%
  \let\tikz@angle@cos\pgfmathresult%
  \pgfmathsin@{\tikz@timer@start@angle}%
  \let\tikz@angle@sin\pgfmathresult%
  \pgftransformarcaxesattime{\tikz@time}{%
    \pgfpointdiff{%
      \pgfpointadd{%
        \pgfpointscale{\tikz@angle@cos}{\tikz@timer@zero@axis}%
      }{%
        \pgfpointscale{\tikz@angle@sin}{\tikz@timer@ninety@axis}%
      }%
    }%
    {\tikz@timer@start}%
  }%
  {\tikz@timer@zero@axis}%
  {\tikz@timer@ninety@axis}%
  {\tikz@timer@start@angle}{\tikz@timer@end@angle}%
}%



%
% Coordinate systems
%

\def\tikzdeclarecoordinatesystem#1#2{%
  \expandafter\def\csname tikz@parse@cs@#1\endcsname(##1){%
    \pgf@process{%
      #2%
      \global\let\tikz@smubble@b=\tikz@shapeborder@name%
    }%
    \let\tikz@shapeborder@name=\tikz@smubble@b%
    \edef\tikz@return@coordinate{\noexpand\pgfqpoint{\the\pgf@x}{\the\pgf@y}}}%
}%
\def\tikzaliascoordinatesystem#1#2{%
  \edef\pgf@marshal{\noexpand\let\expandafter\noexpand\csname
    tikz@parse@cs@#1\endcsname=\expandafter\noexpand\csname
    tikz@parse@cs@#2\endcsname}%
  \pgf@marshal%
}%


% Default coordinate systems:

\tikzdeclarecoordinatesystem{canvas}
{%
  \tikzset{cs/.cd,x=0pt,y=0pt,#1}%
  \pgfpoint{\tikz@cs@x}{\tikz@cs@y}%
}%

\tikzdeclarecoordinatesystem{canvas polar}
{%
  \tikzset{cs/.cd,angle=0,radius=0cm,#1}%
  \pgfpointpolar{\tikz@cs@angle}{\tikz@cs@xradius and \tikz@cs@yradius}%
}%

\tikzdeclarecoordinatesystem{xyz}
{%
  \tikzset{cs/.cd,x=0,y=0,z=0,#1}%
  \pgfpointxyz{\tikz@cs@x}{\tikz@cs@y}{\tikz@cs@z}%
}%

\tikzdeclarecoordinatesystem{xyz polar}
{%
  \tikzset{cs/.cd,angle=0,radius=0,#1}%
  \pgfpointpolarxy{\tikz@cs@angle}{\tikz@cs@xradius and \tikz@cs@yradius}%
}%
\tikzaliascoordinatesystem{xy polar}{xyz polar}%


\tikzdeclarecoordinatesystem{node}
{%
  \tikzset{cs/.cd,name=,anchor=none,angle=none,#1}%
  \ifx\tikz@cs@anchor\tikz@nonetext%
    \ifx\tikz@cs@angle\tikz@nonetext%
      \expandafter\ifx\csname pgf@sh@ns@\tikz@cs@node\endcsname\tikz@coordinate@text%
      \else
        \aftergroup\tikz@shapebordertrue%
        \edef\tikz@shapeborder@name{\tikz@pp@name{\tikz@cs@node}}%
      \fi%
      \pgfpointanchor{\tikz@pp@name{\tikz@cs@node}}{center}%
    \else%
      \pgfpointanchor{\tikz@pp@name{\tikz@cs@node}}{\tikz@cs@angle}%
    \fi%
  \else%
    \pgfpointanchor{\tikz@pp@name{\tikz@cs@node}}{\tikz@cs@anchor}%
  \fi%
}%

% Intersection coordinates
\tikzset{cs/first line/.code=\def\tikz@cs@line@a{#1}\def\tikz@cs@type@a{line}}%
\tikzset{cs/second line/.code=\def\tikz@cs@line@b{#1}\def\tikz@cs@type@b{line}}%

\tikzset{cs/first node/.code=\tikz@cs@unpack{\tikz@cs@node@a}{\tikz@cs@type@a}{#1}}%
\tikzset{cs/second node/.code=\tikz@cs@unpack{\tikz@cs@node@b}{\tikz@cs@type@b}{#1}}%

\def\tikz@cs@unpack#1#2#3{%
  \expandafter\ifx\csname pgf@sh@ns@#3\endcsname\relax%
    \tikzerror{Undefined node ``#3''}%
  \else%
    \def#1{#3}%
    \edef#2{\csname pgf@sh@ns@#3\endcsname}%
  \fi%
}%

\tikzset{cs/solution/.initial=1}%

\tikzset{cs/horizontal line through/.store in=\tikz@cs@hori@line}%
\tikzset{cs/vertical line through/.store in=\tikz@cs@vert@line}%

\tikzdeclarecoordinatesystem{intersection}
{%
  \tikzset{cs/.cd,#1}%
  \expandafter\ifx\csname tikz@intersect@\tikz@cs@type@a @and@\tikz@cs@type@b\endcsname\relax%
    \tikzerror{I do not know how to compute the intersection
    of a \tikz@cs@type@a and a \tikz@cs@type@b. Try saying
    \string\usetikzlibrary{calc}}%
    \pgfpointorigin%
  \else%
    \csname tikz@intersect@\tikz@cs@type@a @and@\tikz@cs@type@b\endcsname%
  \fi%
}%

\def\tikz@intersect@line@and@line{%
  \expandafter\tikz@scan@one@point\expandafter\tikz@parse@line\tikz@cs@line@a%
  \pgf@xa=\pgf@xc%
  \pgf@ya=\pgf@yc%
  \pgf@xb=\pgf@x%
  \pgf@yb=\pgf@y%
  \expandafter\tikz@scan@one@point\expandafter\tikz@parse@line\tikz@cs@line@b%
  \edef\pgf@marshal{%
    {\noexpand\pgfpointintersectionoflines%
      {\noexpand\pgfqpoint{\the\pgf@xa}{\the\pgf@ya}}%
      {\noexpand\pgfqpoint{\the\pgf@xb}{\the\pgf@yb}}%
      {\noexpand\pgfqpoint{\the\pgf@xc}{\the\pgf@yc}}%
      {\noexpand\pgfqpoint{\the\pgf@x}{\the\pgf@y}}}}%
  \pgf@marshal%
}%

\def\tikz@parse@line#1--{%
  \pgf@process{#1}%
  \pgf@xc=\pgf@x%
  \pgf@yc=\pgf@y%
  \tikz@scan@one@point\pgf@process%
}%


\tikzdeclarecoordinatesystem{perpendicular}
{%
  \tikzset{cs/.cd,#1}%
  \expandafter\tikz@scan@one@point\expandafter\tikz@parse@intersection@a\tikz@cs@hori@line%
  \expandafter\tikz@scan@one@point\expandafter\tikz@parse@intersection@b\tikz@cs@vert@line%
  \pgfqpoint{\the\pgf@xb}{\the\pgf@ya}
}%

\tikzdeclarecoordinatesystem{barycentric}
{%
  {%
    \pgf@xa=0pt% point
    \pgf@ya=0pt%
    \pgf@xb=0pt% sum
    \tikz@bary@dolist#1,=,%
    \pgfmathparse{1/\the\pgf@xb}%
    \global\pgf@x=\pgfmathresult\pgf@xa%
    \global\pgf@y=\pgfmathresult\pgf@ya%
  }%
}%

\def\tikz@bary@dolist#1=#2,{%
  \def\tikz@temp{#1}%
  \ifx\tikz@temp\pgfutil@empty%
  \else
    \pgf@process{\pgfpointanchor{#1}{center}}%
    \pgfmathparse{#2}%
    \advance\pgf@xa by\pgfmathresult\pgf@x%
    \advance\pgf@ya by\pgfmathresult\pgf@y%
    \advance\pgf@xb by\pgfmathresult pt%
    \expandafter\tikz@bary@dolist%
  \fi%
}%

\tikzset{cs/x/.store in=\tikz@cs@x}%
\tikzset{cs/y/.store in=\tikz@cs@y}%
\tikzset{cs/z/.store in=\tikz@cs@z}%
\tikzset{cs/angle/.store in=\tikz@cs@angle}%
\tikzset{cs/x radius/.store in=\tikz@cs@xradius}%
\tikzset{cs/y radius/.store in=\tikz@cs@yradius}%
\tikzset{cs/radius/.style={/tikz/cs/x radius={#1},/tikz/cs/y radius={#1}}}%
\tikzset{cs/name/.store in=\tikz@cs@node}%
\tikzset{cs/anchor/.store in=\tikz@cs@anchor}%



%
% Coordinate management
%


% Last position visited
\def\tikz@last@position{\pgfqpoint{\tikz@lastx}{\tikz@lasty}}%
\def\tikz@last@position@saved{\pgfqpoint{\tikz@lastxsaved}{\tikz@lastysaved}}%

% Make given point the last position visited
\def\tikz@make@last@position#1{%
  \pgf@process{#1}%
  \tikz@lastx=\pgf@x\relax%
  \tikz@lasty=\pgf@y\relax%
  \iftikz@updatecurrent%
    \tikz@lastxsaved=\pgf@x\relax%
    \tikz@lastysaved=\pgf@y\relax%
  \fi%
  \iftikz@updatenext
    \tikz@updatecurrenttrue%
  \fi
}%

\newif\iftikz@updatecurrent
\tikz@updatecurrenttrue
\newif\iftikz@updatenext
\tikz@updatenexttrue



% Scanner: Scans a point or a relative point.
% It then calls the first parameter with the argument set to an
% appropriate pgf command representing that point.

\def\tikz@scan@one@point#1{%
  \let\tikz@to@use@whom=\tikz@to@use@last@coordinate%
  \tikz@shapeborderfalse%
  \pgfutil@ifnextchar+{\tikz@scan@relative#1}{\tikz@scan@absolute#1}}%
\def\tikz@scan@absolute#1{%
  \pgfutil@ifnextchar({\tikz@scan@@absolute#1}%)
  {%
    \advance\tikz@expandcount by -1
    \ifnum\tikz@expandcount<0\relax%
      \let\pgfutil@next=\tikz@@scangiveup%
    \else%
      \let\pgfutil@next=\tikz@@scanexpand%
    \fi%
    \pgfutil@next{#1}%
  }%
}%
\def\tikz@@scanexpand#1{\expandafter\tikz@scan@one@point\expandafter#1}%
\def\tikz@@scangiveup#1{\tikzerror{Cannot parse this coordinate}#1{\pgfpointorigin}}%
\def\tikz@scan@@absolute#1({%
  \pgfutil@ifnextchar[% uhoh... options!
  {\def\tikz@scan@point@recall{#1}\tikz@scan@options}%
  {\tikz@@@scan@@absolute#1(}%
}%

\def\tikz@scan@options[#1]#2{%
  \def\tikz@scan@point@options{#1}%
  \tikz@@@scan@@absolute\tikz@scan@handle@options(#2%
}%

\def\tikz@scan@handle@options#1{%
  {%
    % Ok, compute point with options set and zero transformation
    % matrix:
    \pgftransformreset%
    \let\tikz@transform=\pgfutil@empty%
    \expandafter\tikzset\expandafter{\tikz@scan@point@options}%
    \tikz@transform%
    \pgf@process{\pgfpointtransformed{#1}}%
    \xdef\tikz@marshal{\expandafter\noexpand\tikz@scan@point@recall{\noexpand\pgfqpoint{\the\pgf@x}{\the\pgf@y}}}%
  }%
  \tikz@marshal%
}%

\def\tikz@@@scan@@absolute#1({%
  \pgfutil@ifnextchar{$}%$
  {\tikz@parse@calculator#1(}
  {\tikz@scan@no@calculator#1(}%
}%
\def\tikz@scan@no@calculator#1(#2){%
  \edef\tikz@scan@point@coordinate{(#2)}%
  \expandafter\tikz@@scan@@no@calculator\expandafter#1\tikz@scan@point@coordinate%
}%
\def\tikz@@scan@@no@calculator#1(#2){%
  \pgfutil@in@{cs:}{#2}%
  \ifpgfutil@in@%
    \let\pgfutil@next\tikz@parse@coordinatesystem%
  \else%
    \pgfutil@in@{intersection }{#2}%
    \ifpgfutil@in@%
      \let\pgfutil@next\tikz@parse@intersection%
    \else%
      \pgfutil@in@|{#2}%
      \ifpgfutil@in@
        \pgfutil@in@{-|}{#2}%
        \ifpgfutil@in@
          \let\pgfutil@next\tikz@parse@hv%
        \else%
          \let\pgfutil@next\tikz@parse@vh%
        \fi%
      \else%
        \pgfutil@in@:{#2}%
        \ifpgfutil@in@
          \let\pgfutil@next\tikz@parse@polar%
        \else%
          \pgfutil@in@,{#2}%
          \ifpgfutil@in@%
            \let\pgfutil@next\tikz@parse@regular%
          \else%
            \let\pgfutil@next\tikz@parse@node%
          \fi%
        \fi%
      \fi%
    \fi%
  \fi%
  \pgfutil@next#1(#2)%
}%

\def\tikz@parse@calculator#1($#2$){%
  \tikzerror{You need to say \string\usetikzlibrary{calc} for coordinate calculation}%
  #1{\pgfpointorigin}%
}%

\def\tikz@parse@coordinatesystem#1(#2 cs:#3){%
  \let\tikz@return@coordinate=\pgfpointorigin%
  \pgfutil@ifundefined{tikz@parse@cs@#2}
  {\tikzerror{Unknown coordinate system '#2'}}
  {\csname tikz@parse@cs@#2\endcsname(#3)}%
  \expandafter#1\expandafter{\tikz@return@coordinate}%
}%


\newif\iftikz@isdimension
\def\tikz@checkunit#1{%
  \pgfmathparse{#1}%
  \let\iftikz@isdimension=\ifpgfmathunitsdeclared%
}%

\def\tikz@parse@polar#1(#2:#3){%
  \pgfutil@ifundefined{tikz@polar@dir@#2}
  {\tikz@@parse@polar#1({#2}:{#3})}
  {\tikz@@parse@polar#1(\csname tikz@polar@dir@#2\endcsname:{#3})}%
}%
\def\tikz@@parse@polar#1(#2:#3){%
  \pgfutil@in@{ and }{#3}%
  \ifpgfutil@in@%
    \edef\tikz@args{({#2}:#3)}%
  \else%
    \edef\tikz@args{({#2}:{#3} and {#3})}%
  \fi%
  \expandafter\tikz@@@parse@polar\expandafter#1\tikz@args%
}%
\def\tikz@@@parse@polar#1(#2:#3 and #4){%
  \tikz@checkunit{#3}%
  \iftikz@isdimension%
    \tikz@checkunit{#4}%
    \iftikz@isdimension%
      \def\tikz@next{#1{\pgfpointpolar{#2}{#3 and #4}}}%
    \else%
      \tikzerror{You cannot mix dimension and dimensionless values for polar coordinates}
      \def\tikz@next{#1{\pgfpointorigin}}%
    \fi%
  \else%
    \tikz@checkunit{#4}%
    \iftikz@isdimension%
      \tikzerror{You cannot mix dimension and dimensionless values for polar coordinates}
      \def\tikz@next{#1{\pgfpointorigin}}%
    \else%
      \def\tikz@next{#1{\pgfpointpolarxy{#2}{#3 and #4}}}%
    \fi%
  \fi%
  \tikz@next%
}%
\def\tikz@polar@dir@up{90}%
\def\tikz@polar@dir@down{-90}%
\def\tikz@polar@dir@left{180}%
\def\tikz@polar@dir@right{0}%
\def\tikz@polar@dir@north{90}%
\def\tikz@polar@dir@south{-90}%
\def\tikz@polar@dir@east{0}%
\def\tikz@polar@dir@west{180}%
\expandafter\def\csname tikz@polar@dir@north east\endcsname{45}%
\expandafter\def\csname tikz@polar@dir@north west\endcsname{135}%
\expandafter\def\csname tikz@polar@dir@south east\endcsname{-45}%
\expandafter\def\csname tikz@polar@dir@south west\endcsname{-135}%


% MW:
% Check to see if the y-coordinate is inside {}. If it is, scan it and
% reinsert it into the stream inside an extra group.
%
\def\tikz@parse@regular#1(#2,{%
    \pgfutil@ifnextchar\bgroup{\tikz@@parse@regular#1{#2}}{\tikz@@@parse@regular#1{#2}}%
}%
\def\tikz@@parse@regular#1#2#3{%
    \pgfutil@ifnextchar[{% Uh oh! An array index.
        \tikz@@@parse@regular#1{#2}{#3}}%
      {\tikz@@@parse@regular#1{#2}{{#3}}}}%

% Originally \def\tikz@parse@regular#1(#2,#3){%
%
\def\tikz@@@parse@regular#1#2#3){%
  \pgfutil@in@,{#3}%
  \ifpgfutil@in@%
    \tikz@parse@splitxyz{#1}{#2}#3,%
  \else%
    \tikz@checkunit{#2}%
    \iftikz@isdimension%
      \tikz@checkunit{#3}%
      \iftikz@isdimension%
        \def\pgfutil@next{#1{\pgfpoint{#2}{#3}}}%
      \else%
        \def\pgfutil@next{#1{\pgfpointadd{\pgfpoint{#2}{0pt}}{\pgfpointxy{0}{#3}}}}%
      \fi%
    \else%
      \tikz@checkunit{#3}%
      \iftikz@isdimension%
        \def\pgfutil@next{#1{\pgfpointadd{\pgfpoint{0pt}{#3}}{\pgfpointxy{#2}{0}}}}%
      \else%
        \def\pgfutil@next{#1{\pgfpointxy{#2}{#3}}}%
      \fi%
    \fi%
  \fi%
  \pgfutil@next%
}%

\def\tikz@parse@splitxyz#1#2#3,#4,{%
  \def\pgfutil@next{#1{\pgfpointxyz{#2}{#3}{#4}}}%
}%

\def\tikz@coordinate@text{coordinate}%

\def\tikz@parse@node#1(#2){%
  \pgfutil@in@.{#2}% Ok, flag this
  \ifpgfutil@in@
    \tikz@calc@anchor#2\tikz@stop%
  \else%
    \tikz@calc@anchor#2.center\tikz@stop% to be on the save side, in
                                % case iftikz@shapeborder is ignored...
    \ifcsname pgf@sh@ns@\tikz@pp@name{#2}\endcsname
      \expandafter\ifx\csname pgf@sh@ns@\tikz@pp@name{#2}\endcsname\tikz@coordinate@text%
      \else
        \tikz@shapebordertrue%
        \def\tikz@shapeborder@name{\tikz@pp@name{#2}}%
      \fi%
    \else\ifcsname pgf@sh@ns@#2\endcsname
      \expandafter\ifx\csname pgf@sh@ns@#2\endcsname\tikz@coordinate@text%
      \else
        \tikz@shapebordertrue%
        \def\tikz@shapeborder@name{#2}%
      \fi%
    \fi\fi
  \fi%
  \edef\tikz@marshal{\noexpand#1{\noexpand\pgfqpoint{\the\pgf@x}{\the\pgf@y}}}%
  \tikz@marshal%
}%

\def\tikz@calc@anchor#1.#2\tikz@stop{%
  % Check if a shape with name prefix exists, otherwise try the global name
  % without prefix.
  \ifcsname pgf@sh@ns@\tikz@pp@name{#1}\endcsname%
    \pgfpointanchor{\tikz@pp@name{#1}}{#2}%
  \else
    \pgfpointanchor{#1}{#2}%
  \fi
}%


\def\tikz@parse@hv#1(#2){%
  \pgfutil@in@{ -| }{#2}%
  \ifpgfutil@in@%
    \let\tikz@next=\tikz@parse@hvboth%
  \else%
    \pgfutil@in@{ -|}{#2}%
    \ifpgfutil@in@%
      \let\tikz@next=\tikz@parse@hvleft%
    \else%
      \pgfutil@in@{-| }{#2}%
      \ifpgfutil@in@%
        \let\tikz@next=\tikz@parse@hvright%
      \else%
        \let\tikz@next=\tikz@parse@hvdone%
      \fi%
    \fi%
  \fi%
  \tikz@next#1(#2)}%
\def\tikz@parse@hvboth#1(#2 -| #3){\tikz@parse@vhdone#1({#3}|-{#2})}%
\def\tikz@parse@hvleft#1(#2 -|#3){\tikz@parse@vhdone#1({#3}|-{#2})}%
\def\tikz@parse@hvright#1(#2-| #3){\tikz@parse@vhdone#1({#3}|-{#2})}%
\def\tikz@parse@hvdone#1(#2-|#3){\tikz@parse@vhdone#1({#3}|-{#2})}%

\def\tikz@parse@vh#1(#2){%
  \pgfutil@in@{ |- }{#2}%
  \ifpgfutil@in@%
    \let\tikz@next=\tikz@parse@vhboth%
  \else%
    \pgfutil@in@{ |-}{#2}%
    \ifpgfutil@in@%
      \let\tikz@next=\tikz@parse@vhleft%
    \else%
      \pgfutil@in@{|- }{#2}%
      \ifpgfutil@in@%
        \let\tikz@next=\tikz@parse@vhright%
      \else%
        \let\tikz@next=\tikz@parse@vhdone%
      \fi%
    \fi%
  \fi%
  \tikz@next#1(#2)}%
\def\tikz@parse@vhboth#1(#2 |- #3){\tikz@parse@vhdone#1({#2}|-{#3})}%
\def\tikz@parse@vhleft#1(#2 |-#3){\tikz@parse@vhdone#1({#2}|-{#3})}%
\def\tikz@parse@vhright#1(#2|- #3){\tikz@parse@vhdone#1({#2}|-{#3})}%
\def\tikz@parse@vhdone#1(#2|-#3){%
  {%
    \tikz@@@scan@@absolute\tikz@parse@vh@mid(#2)%
    \tikz@@@scan@@absolute\tikz@parse@vh@end(#3)%
    \xdef\tikz@marshal{\noexpand#1{\noexpand\pgfqpoint{\the\pgf@xa}{\the\pgf@ya}}}%
  }%
  \tikz@marshal%
}%
\def\tikz@parse@vh@mid#1{\pgf@process{#1}\pgf@xa=\pgf@x}%
\def\tikz@parse@vh@end#1{\pgf@process{#1}\pgf@ya=\pgf@y}%

\def\tikz@parse@intersection#1(intersection{%
  \pgfutil@ifnextchar o{%
    \tikz@parse@main@intersection#1 1%
  }{%
    \tikz@parse@main@intersection#1%
  }%
}%
\def\tikz@parse@main@intersection#1#2of #3 and #4){%
  \tikzset{cs/solution={#2}}%
  \pgfutil@in@{--}{#3}%
  \ifpgfutil@in@%
    \tikz@reparse@line{first}#3\pgf@stop%
  \else%
    \tikzset{cs/first node={#3}}%
  \fi%
  \pgfutil@in@{--}{#4}%
  \ifpgfutil@in@%
    \tikz@reparse@line{second}#4\pgf@stop%
  \else%
    \tikzset{cs/second node={#4}}%
  \fi%
  \tikz@parse@cs@intersection()% advanced hackery...
  \edef\pgf@marshal{\noexpand#1{\noexpand\pgfqpoint{\the\pgf@x}{\the\pgf@y}}}%
  \pgf@marshal%
}%
\def\tikz@reparse@line#1#2--#3\pgf@stop{%
  \tikzset{cs/#1 line={(#2)--(#3)}}%
}%


\def\tikz@parse@intersection@a#1{\pgf@process{#1}\pgf@xa=\pgf@x\pgf@ya=\pgf@y}%
\def\tikz@parse@intersection@b#1{\pgf@process{#1}\pgf@xb=\pgf@x\pgf@yb=\pgf@y}%

\def\tikz@scan@relative#1+{%
  \pgfutil@ifnextchar+{\tikz@scan@plusplus#1}{\tikz@scan@oneplus#1}}%

\def\tikz@scan@plusplus#1+{%
  \def\tikz@doafter{#1}%
  \tikz@scan@absolute\tikz@add%
}%
\def\tikz@add#1{%
  \tikz@doafter{\pgfpointadd{#1}{\tikz@last@position@saved}}%
}%
\def\tikz@scan@oneplus#1{%
  \def\tikz@doafter{#1}%
  \tikz@updatecurrentfalse%
  \tikz@scan@absolute\tikz@add%
}%



%
% Quote handling
%

\let\tikz@enable@node@quotes\relax
\let\tikz@enable@edge@quotes\relax
\let\tikz@enable@pic@quotes\relax



% Loading further libraries

% Include a library file.
%
% #1 = List of names of library file.
%
% Description:
%
% This command includes a list of TikZ library files. For each file X in the
% list, the file tikzlibraryX.code.tex is included, provided this has
% not been done earlier.
%
% For the convenience of Context users, both round and square brackets
% are possible for the argument.
%
% If no file tikzlibraryX.code.tex exists, the file
% pgflibraryX.code.tex is tried instead. If this file, also, does not
% exist, an error message is printed.
%
% Example:
%
% \usetikzlibrary{arrows}
% \usetikzlibrary[patterns,topaths]

\def\usetikzlibrary{\pgfutil@ifnextchar[{\use@tikzlibrary}{\use@@tikzlibrary}}%}%
\def\use@tikzlibrary[#1]{\use@@tikzlibrary{#1}}%
\def\use@@tikzlibrary#1{%
  \edef\pgf@list{#1}%
  \pgfutil@for\pgf@temp:=\pgf@list\do{%
    \expandafter\pgfkeys@spdef\expandafter\pgf@temp\expandafter{\pgf@temp}%
    \ifx\pgf@temp\pgfutil@empty
    \else
      \expandafter\ifx\csname tikz@library@\pgf@temp @loaded\endcsname\relax%
      \expandafter\global\expandafter\let\csname tikz@library@\pgf@temp @loaded\endcsname=\pgfutil@empty%
      \expandafter\edef\csname tikz@library@#1@atcode\endcsname{\the\catcode`\@}
      \expandafter\edef\csname tikz@library@#1@barcode\endcsname{\the\catcode`\|}
      \expandafter\edef\csname tikz@library@#1@dollarcode\endcsname{\the\catcode`\$}
      \catcode`\@=11
      \catcode`\|=12
      \catcode`\$=3
      \pgfutil@InputIfFileExists{tikzlibrary\pgf@temp.code.tex}{}{
        \pgfutil@IfFileExists{pgflibrary\pgf@temp.code.tex}{%
          \expandafter\usepgflibrary\expandafter{\pgf@temp}%
        }{%
          \tikzerror{I did not find the tikz library
            '\pgf@temp'. I looked for files named
            tikzlibrary\pgf@temp.code.tex and
            pgflibrary\pgf@temp.code.tex, but neither
            could be found in the current texmf trees.}
        }}%
      \catcode`\@=\csname tikz@library@#1@atcode\endcsname
      \catcode`\|=\csname tikz@library@#1@barcode\endcsname
      \catcode`\$=\csname tikz@library@#1@dollarcode\endcsname
      \fi%
    \fi
  }%
}%


% Always-present libraries:

\usetikzlibrary{topaths}%




\endinput
